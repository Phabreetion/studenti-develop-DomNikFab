%%%%%%%%%%%%%%%%%%%%%%%%%%%%%%%%%%%%%h%%%%%%%%%%%%%%%%%%%%%%

\chapter{Dominio del problema}
\label{ref:Introduzione}
Sezione comune a tutti i gruppi: la curerà qualcuno di noi. Non rivolta ai singoli gruppi.
%%%%%%%%%%%%%%%%%%%%%%%%%%%%%%%%%%%%%%%%%%%%%%%%%%%%%%%%%%%

\section{Introduzione al RAD}
Questo \textit{Requirement Analysis Document} ha lo scopo di descrivere la fase di raccolta e analisi dei requisiti funzionali, non funzionali e pseudo-requisiti dell'applicazione \textit{Studenti Unimol} quale sistema da sviluppare per gli esami di \textit{Ingegneria del software e laboratorio} e \textit{Gestione progetti software} previsti dai piani di studio del \textit{Corso di studio unificato in Informatica}. Tale documento funge da contratto tra il prof. Fausto Fasano che ha commissionato il sistema e i team di progettazione e sviluppo che lo realizzeranno. Esso fornisce una panoramica astratta del sistema che gli studenti si accingeranno ad implementare.

\paragraph{}
Lorem ipsum dolor sit amet...

\section{Scope}

\paragraph{}
Lorem ipsum dolor sit amet...

\section{Contesto e panoramica del sistema}
Il corso di \textit{Ingegneria del software e laboratorio} all'\textit{Università degi Studi del Molise} prevede la suddivisione degli studenti in gruppi di lavoro ai quali è chiesto di progettare, documentare e sviluppare alcune funzinalità interne all'applicazione \textit{Studenti Unimol} utilizzata dagli studenti per la gestione semplificata della loro carriera accademica. Sarà rilasciata una versione aggiornata di quella attuale previa revisione della versione già esistente. Essa sarà resa disponibile a tutti gli studenti regolarmente iscritti all'\textit{Università degli Studi del Molise}.

\paragraph{}


\section{Manager di progetto e sviluppatori}

\paragraph{}
L'app \textit{Studenti Unimol} è stata documentata e implementata nel 2017 dagli studenti del corso di \textit{Ingegneria del Software}, coordinati da gruppi di manager del corso magistrale in \textit{Sicurezza dei sistemi software}, operanti nell'ambito dell'insegnamento di \text{Gestione progetti software}. Successivamente l'applicazione è stata manutenuta dal prof. Fausto Fasano, il quale, nel 2019, ha chiesto agli studenti di triennale e magistrale di manutenere e far evolvere l'attuale applicazione per produrre la versione 3.0. Ogni team è supervisionato da alcuni manager che si occuperanno della comunicazione e dell'organizzazione dell'intero progetto, coordinando i gruppi a loro assegnati e gestendo le relazioni orizzontali tra i gruppi che lavorano in parallelo su funzionalità diverse dell'app.

//TO DO: descrizione dei gruppi e allocazione sulle funzionalità.

\section{Glossario}

\paragraph{}
Di seguito è riportata una tabella degli acronimi e dei termini tecnici utilizzati nel documento:

\begin{table}
\begin{tabular}{p{2in}|p{4in}} \\
	{\bf Termine o sigla} & Descrizione\\ \hline
	RAD & Requirement Analysis Document, documento di analisi dei requisiti (trattandosi di una sigla, conviene scrivere il corrispondente esatto dell’acronimo inglese e dopo la traduzione in italiano) \\
	TO DO 2 &  TO DO \\
	TO DO 3 & TO DO \\
\end{tabular}
\end{table}
\newpage