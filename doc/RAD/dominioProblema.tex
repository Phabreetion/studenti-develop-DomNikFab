%%%%%%%%%%%%%%%%%%%%%%%%%%%%%%%%%%%%%h%%%%%%%%%%%%%%%%%%%%%%

\chapter{Dominio del problema}
\label{ref:Introduzione}
Sezione comune a tutti i gruppi: la curerà qualcuno di noi. Non metteteci le mani!
%%%%%%%%%%%%%%%%%%%%%%%%%%%%%%%%%%%%%%%%%%%%%%%%%%%%%%%%%%%

\section{Introduzione al RAD}

\paragraph{}
Lorem ipsum dolor sit amet...

\section{Scope}

\paragraph{}
Lorem ipsum dolor sit amet...

\section{Contesto e panoramica del sistema}

\paragraph{}
L’app \textbf{\textit{Studenti Unimol}} è stata documentata e implementata nel 2017 dagli studenti del corso di {\textit{Ingegneria del Software}, coordinati da un gruppo di manager del corso magistrale in {\textit{Sicurezza dei Sistemi Software}. Successivamente l’applicazione è stata manutenuta dal prof. Fausto Fasano, il quale nel 2019 ha  chiesto agli studenti di triennale e magistrale di manutenere e fare evolvere l’attuale applicazione per produrre una versione 3.0: nello specifico agli studenti è stato chiesto di risolvere alcuni \textit{bug} attualmente presenti nell'applicazione e di migliorare alcune funzionalità. Ogni team è supervisionato da manager che si occuperanno della comunicazione e dell’organizzazione dell’intero progetto, coordinandosi tra di loro e coordinando i gruppi a loro assegnati, per realizzare una comunicazione completa ed efficace \textit{intergruppo} e \textit{intragruppo}. 

\section{Manager di progetto e sviluppatori}

\paragraph{}
Lorem ipsum dolor sit amet...

\section{Glossario}

\paragraph{}
Di seguito è riportata una tabella degli acronimi e dei termini tecnici utilizzati nel documento:

\begin{table}
\begin{tabular}{p{1.5in}|p{4in}} \\
	{\bf Termine o sigla} & {\bf Descrizione} \\ \hline
	RAD & Requirement Analysis Document, documento di analisi dei requisiti (trattandosi di una sigla, conviene scrivere il corrispondente esatto dell’acronimo inglese e dopo la traduzione in italiano) \\
	Azienda USL 2 & { 216546} \\
	Azienda USL 3 & { 269265} \\
\end{tabular}
\end{table}
\clearpage