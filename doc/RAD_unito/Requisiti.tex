%%%%%%%%%%%%%%%%%%%%%%%%%%%%%%%%%%%%%%%%%%%%%%%%%%%%%%%%%%%

\chapter{Analisi dei requisiti}
\label{ref:requisiti1}

\section{Requisiti funzionali}

\subsection{Gestione Piano di Studio}
\paragraph{} 
L’app dovrà mostrare i corsi previsti dal piano di studio dello studente permettendo la visualizzazione di tutti i corsi ad esso afferenti evidenziando quelli per cui l’esame è stato sostenuto e quelli per cui l’esame è da sostenere. Per ogni corso saranno inoltre visualizzati i relativi dettagli, come il numero di CFU, la valutazione in trentesimi oppure eventuale idoneità. Lo studente potrà effettuare le operazioni di ricerca, filtro e ordinamento dell'elenco dei corsi e potrà scegliere se memorizzare nello \textit{storage} le sue preferenze oppure resettarle. Il sistema richiederà i dati aggiornati al sincronizzatore, il quale si occuperà di salvarli nello \textit{storage} dell’app.
Il relativo diagramma dei casi d'uso è visibile al punto \ref{diag:gestionePianoStudio}.

\subsection{Visualizza dettagli corso}
\paragraph{} 
L’app dovrà mostrare informazioni relative a ciascun corso cliccando sullo stesso nella sezione carriera. Selezionando la voce \textit{dettagli}, l’app mostrerà allo studente, per ogni corso, il/i docente/i responsabile/i dell’insegnamento, il numero di CFU, l’anno accademico in cui viene frequentato il corso ed i suoi contenuti. Da tale sezione, inoltre, si potranno raggiungere le interfacce del materiale didattico e dell’elenco appelli. Selezionando un corso il cui esame è già stato sostenuto sarà, invece, possibile ottenere anche informazioni relative alla data in cui è stato svolto, alla data in cui è stato verbalizzato ed il voto ottenuto. Il sistema richiederà i dati aggiornati al sincronizzatore, il quale si occuperà di salvarli nello \textit{storage} dell’app.
Il relativo diagramma dei casi d'uso è visibile al punto \ref{diag:visualizzaDettCorso}.

\subsection{Gestione materiale didattico}
\paragraph{} 
L’app mostrerà l’elenco dei file relativi ad un corso selezionato, dopo aver richiesto i dati al sincronizzatore, averli ricevuti e salvati nello \textit{storage} dell’applicazione. Per visualizzare un file selezionato dalla lista, questo dovrà essere prima scaricato all’interno dello \textit{storage} dell’app: in tal caso, il file potrà essere aperto e visualizzato. Nel caso in cui il file selezionato fosse stato precedentemente scaricato sarà possibile eliminarlo, altrimenti quest’ultima notificherà allo studente l’assenza del file nello \textit{storage} e gli chiederà se è intenzionato a scaricarlo. Se lo studente sceglierà di scaricare un file, il sistema richiederà i dati aggiornati al sincronizzatore, il quale si occuperà di salvarli nello \textit{storage} dell’app.
Il relativo diagramma dei casi d'uso è visibile al punto \ref{diag:gestioneMatDidattico}.

\subsection{Gestione appelli}
\paragraph{} 
L’app dovrà mostrare l’elenco degli appelli disponibili richiedendoli al sincronizzatore e salvando nello \textit{storage} i dati ricevuti: lo studente potrà prenotarsi a uno specifico appello tra quelli visualizzati, che saranno solo quelli prenotabili. Lo studente potrà effettuare le operazioni di ricerca, filtro con parola chiave e ordinamento dall'elenco di appelli disponibili e scegliere se memorizzare nello \textit{storage} le sue preferenze oppure resettarle. Il sistema richiederà i dati aggiornati al sincronizzatore, il quale si occuperà di salvarli nello \textit{storage} dell’app. Lo studente, selezionando una data di appello, potrà effettuare una prenotazione, la quale verrà inserita dal sistema nell’elenco degli appelli prenotati. La prenotazione potrà essere annullata fino a cinque giorni prima della data di esame.
Il relativo diagramma dei casi d'uso è visibile al punto \ref{diag:gestioneAppelli}.

\subsection{Resquisiti chat}
\subsubsection{Chat studenti}
Il sistema sarà in grado di mostrare le \emph{chat} con cui è possibile interagire. Le funzionalità saranno:

\paragraph{Requisito 1: selezione\\} 
Sarà possibile selezionare la \emph{chat} desiderata tra quelle in elenco. Inoltre al momento della selezione saranno mostrati i messaggi non letti.

\paragraph{Requisito 2: visualizzazione\\}
Verranno mostrati i messaggi e gli allegati contenuti all’interno della \emph{chat} selezionata, evidenziando i messaggi non letti dagli altri.

\paragraph{Requisito 3: messaggi in evidenza\\}
Vi sarà una differenziazione visiva dei messaggi in base al tipo di utente.\\
\\
Il sistema permetterà di interagire con la \emph{chat} selezionata:

\paragraph{Requisito 4: invio messaggio\\}
Sarà possibile inviare messaggi di testo, con l’aggiunta di \emph{emoji}.

\paragraph{Requisito 5: invio allegato\\}
Sarà possibile inviare allegati nella \emph{chat} dell’anno accademico.

\paragraph{Requisito 6: risposta a singolo messaggio\\}
Sarà possibile rispondere ad un messaggio precedentemente inviato, selezionando quest’ultimo, al fine di garantire una migliore comprensione della conversazione della \emph{chat}.

\paragraph{Requisito 7: tag membro in messaggio\\}
Il sistema permetterà di citare un altro membro all’interno di un messaggio di testo avvisando con una notifica diretta il membro selezionato.

\paragraph{Requisito 8: selezione canale d'interesse\\}
Vi sarà la possibilità di selezionare un canale di comunicazione interno alla \emph{chat}, per suddividere la \emph{chat} principale in base a temi differenti e quindi garantire una suddivisione logica delle conversazioni.

\paragraph{Requisito 9: segnalazione messaggio\\}
Sarà possibile segnalare eventuali messaggi con contenuti moralmente inadatti e inadeguati alla \emph{chat} selezionata.

\paragraph{Requisito 10: ricerca testo nella \emph{chat}\\}
Il sistema permetterà la ricerca di caratteri o parole contenute nei messaggi testuali presenti all’interno della \emph{chat}.

\paragraph{Requisito 11: gestione allegato\\}
L’applicazione permetterà di selezionare uno o più contenuti multimediali presenti all’interno del dispositivo in uso per poterli successivamente inviare all’interno della \emph{chat}.\\
\\
Il sistema, inoltre, permetterà di:

\paragraph{Requisito 12: gestione notifiche \emph{chat}\\}
Il sistema imposterà automaticamente l’abilitazione a ricevere notifiche, sarà comunque possibile disabilitarle.\\
\\

Il sistema permetterà alcune funzionalità offline:

\paragraph{Requisito 13: visualizzazione messaggi\\}
Il sistema sarà capace di visualizzare i messaggi ricevuti fino all’ultima sessione di connessione.

\paragraph{Requisito 14: coda d'invio\\}
Il sistema consentirà di mettere i messaggi in coda di invio in caso di connessione assente. Sarà prevista quindi la capacità del sistema di provvedere all’invio effettivo dei messaggi messi in coda durante l’assenza di connessione.

\subsubsection{Chat docenti}
Il \emph{Docente} avrà tutte le funzionalità dello \emph{Studente} con l’aggiunta di:

\paragraph{Requisito 1: gestione chat\\}
Di default le \emph{chat} saranno disabilitate, sarà comunque possibile abilitare quelle desiderate.

\paragraph{Requisito 2: gestione canale\\}
Vi sarà la possibilità di creare un canale di comunicazione interno alla \emph{chat}, per suddividere la \emph{chat} principale in base a temi differenti e quindi garantire una suddivisione logica delle conversazioni. Sarà inoltre possibile aggiungere o eliminare membri e, eventualmente, eliminare l’intero canale.

\paragraph{Requisito 3: blocco utente\\}
Sarà possibile impedire l’invio di messaggi per un tempo determinato ad utenti specifici.

\paragraph{Requisito 4: invio allegato\\}
Sarà possibile inviare allegati in tutte le \emph{chat}.


\subsubsection{Pannello di amministrazione}
Il sistema consentirà di accedere al pannello amministrazione mediante un autenticazione tramite credenziali.\\
\emph{L’Amministratore} effettuato l’accesso al sistema, e visualizzata la schermata iniziale potrà:

\paragraph{Requisito 1: selezionare categoria \emph{chat}\\}
Il sistema permetterà di scegliere a quale categoria di \emph{chat} accedere distinguendo: 
\begin{itemize}
	\item \emph{Chat} degli studenti: visualizzando dopo una ricerca una lista delle \emph{chat} o una singola \emph{chat} degli studenti ricercata.
	\item \emph{Chat} dei corsi: queste ultime a loro volta suddivise in \emph{chat} attive e \emph{chat} non attive.
\end{itemize}

\paragraph{Requisito 2: visualizzare e selezionare singola \emph{chat}\\}
Selezionata una delle tipologie presenti, al fine di agevolare la visualizzazione di una specifica \emph{chat} o gruppo di \emph{chat}, sarà possibile effettuare una ricerca all’interno della tipologia selezionata. Tale ricerca sarà facilitata mediante l’uso di filtri messi a disposizione dal sistema.\\
Sarà possibile, una volta terminata la ricerca, visualizzare la singola \emph{chat} ricercata o l’elenco di \emph{chat} relative ai filtri applicati.
Inoltre sarà possibile, in ogni \emph{chat}, visualizzare i messaggi scambiati all’interno dagli utenti e anche le possibili azioni attuabili.\\
Non si avrà la possibilità di interagire nella conversazione.

\paragraph{Requisito 3: inviare notifiche\\} 
Sarà possibile accedere dalla schermata iniziale alla sezione dedicata all’invio delle notifiche personalizzate. Nello specifico si potrà scrivere una nuova \emph{news} e, attraverso una selezione, decidere a quale \emph{chat} indirizzarla.\\
Sarà possibile distinguere tra: 
\begin{itemize}
	\item Dipartimenti;
	\item Corsi di studio;
	\item Singoli corsi. 
\end{itemize}

\paragraph{Requisito 4: gestione messaggi inopportuni\\}
Dalla schermata iniziale sarà possibile accedere alla sezione dedicata ai messaggi segnalati come inopportuni.
In particolare si visualizzerà una lista delle \emph{chat} dove è presente almeno una segnalazione di uno o più messaggi offensivi, riportandone anche il numero. Cliccando su una \emph{chat}, \emph{l’Amministratore} avrà la possibilità di visualizzare tutti i messaggi relativi alla \emph{chat}, evidenziando i messaggi segnalati come inopportuni. Avrà dunque la possibilità di nascondere o meno il messaggio segnalato e di silenziare l’utente, qualora si sia accertata l’inadeguatezza del contenuto.

\paragraph{Requisito 5: log-out\\}
Il sistema permetterà di uscire con sicurezza dal pannello di gestione delle \emph{chat}.\\
\\
Il sistema permetterà di amministrare in modo efficace gli utenti che potranno utilizzare la \emph{chat}. Nello specifico permetterà di:

\paragraph{Requisito 6: nominare \emph{amministratore}\\}
All’interno di una \emph{chat} il sistema permetterà di nominare uno o più \emph{Amministratori} che potranno gestire la \emph{chat} e i canali aggiungendo o rimuovendo utenti dai canali e nominando \emph{Amministratori} o rendendo non più \emph{Amministratori} altri utenti. L’operazione è reversibile.

\paragraph{Requisito 7: modificare permessi utente\\}
Si potrà modificare il ruolo che avrà un utente in un canale di una \emph{chat}, ossia se sarà un \emph{Amministratore} o un semplice utente del canale. Il ruolo che avrà il singolo utente in una canale della \emph{chat} non sarà necessariamente lo stesso all’interno di tutti i canali di una \emph{chat}.\\ 
L’operazione è reversibile.

\paragraph{Requisito 8: silenziare utente\\}
Un \emph{Amministratore} del sistema potrà silenziare un utente all’interno di un canale. Avrà in seguito la possibilità di reintegrare l’utente precedentemente silenziato.\\
\\
Il sistema permetterà l’interazione con le \emph{chat}, attraverso le seguenti funzionalità:

\paragraph{Requisito 9: gestione \emph{chat} disponibili\\}
Le \emph{chat} verranno rese disponibili dal sistema e sarà possibile, in seguito ad una richiesta, attivare le \emph{chat} dei vari corsi con i relativi docenti.\\ 
Una singola \emph{chat} potrà inoltre essere disabilitata in qualsiasi momento.

\paragraph{Requisito 10: gestione \emph{chat} attive\\}
All’interno di una singola \emph{chat} sarà possibile aggiungere un utente, qualora lo stesso ne richieda l’inserimento.\\
Permetterà anche di creare canali nella \emph{chat} per una migliore gestione della \emph{chat}. 
In ogni \emph{chat} di corso sarà possibile nominare un utente \emph{Amministratore} dell’intera \emph{chat} e nominare un utente \emph{Amministratore} di ogni canale nella \emph{chat}.

\paragraph{Requisito 11: gestione \emph{chat} attive\\}
Al momento della creazione del nuovo canale sarà impostato e visualizzato come canale di default il primo, in seguito sarà possibile scegliere il canale di default da visualizzare all’apertura della \emph{chat}. All’interno di un canale si potranno inserire o rimuovere utenti. Sarà possibile eliminare un canale all’interno della \emph{chat} qualora non si ritenga più necessari.

\subsection {Assistenza}

\paragraph{Requisito 1: richiedi assistenza\\}
Il sistema, per risolvere tutte le problematiche degli utenti e per andare in contro ad ogni loro esigenza, prevederà la possibilità di comunicare con un gruppo di esperti che si occuperà della risoluzione delle problematiche.

\paragraph{Requisito 2: comunicazione problematiche\\}
Sarà disponibile per gli utenti una \emph{chat} di assistenza, disabilitata fino all’accettazione della prima richiesta di supporto, in cui sarà possibile descrivere il problema e comunicare con un gruppo di assistenti;

\paragraph{Requisito 3: gruppo \emph{supporter}\\}
L’assistenza sarà composta da un gruppo di utenti esperti con una conoscenza dell’app \emph{Studenti Unimol} tale da poter comprendere la natura delle problematiche e fornire istruzioni adeguate agli utenti richiedenti assistenza. 

\paragraph{Requisito 4: \emph{chat} \emph{supporter}\\}
Il gruppo di assistenti visualizzerà la \emph{chat} di assistenza diversamente dagli altri utenti, sarà presente un canale che conterrà le problematiche risolte e uno con quelle ancora irrisolte.

\paragraph{Requisito 5: risoluzione problematiche\\}
Per ogni problematica qualunque esperto sarà capace di comunicare direttamente con il richiedente assistenza e potrà inoltre aggiungere la problematica tra quelle risolte. 

\subsection{Visualizza Rubrica}
\paragraph{}
La funzionalità permette di visualizzare una lista contatti aggiornata, contenente tutte le informazioni relative al personale Unimol. L’accesso alla rubrica è garantito agli studenti
registrati e che hanno effettuato l'accesso al sistema.
Lo studente scrive nella barra di ricerca il docente o il personale tecnico-amministartivo del quale vuole ottenere le informazioni di contatto, oppure, in alternativa, la sede didattica di cui vuole visualizzare i relativi contatti.
Se l'informazione non è presente nel database, verrà mostrato
il messaggio di mancato riscontro, se così non fosse saranno
mostrate le informazioni richieste dall'utente.

\subsection{Visualizza Contatto}
\paragraph{}
La funzionalità permette di visualizzare le informazioni relative al contatto cercato: nome e cognome, dipartimento, sede didattica, email e numero telefonico.

\subsection{Requisiti Previsione Media}

La funzionalità consente allo studente di prevedere le medie, aritmetica e ponderata, e la base di laurea aggiornata, simulando il conseguimento di un esame con una determinata valutazione. La stima si basa sulle medie precedenti e sul numero di CFU conseguiti (idoneità escluse). Il sistema inoltre salverà la simulazione effettuata (se lo studente lo desidererà).

\subsubsection{Visualizza previsione media}

Lo studente, accede alla funzionalità previsione media, selezionando gli esami di suo interesse, in modo tale da effettuare una simulazione per prevedere un'ipotetica base di laurea, con relativa media aritmetica e ponderata.

\section{Requisiti non funzionali}

Facendo riferimento al modello FURPS+ sono stati individuati i seguenti requisiti
non funzionali.

\paragraph{Funzionalità\\} 
Il sistema sarà in grado di fruire tutte le funzionalità richieste con la massima completezza dell’applicazione.

\paragraph{Usabilità\\} 
Il sistema presenterà un’interfaccia semplice ed essenziale che garantirà la completezza e la comprensibilità per un utilizzo facile ed intuitivo per il monitoraggio della propria carriera universitaria. Al primo avvio l’utente, tramite un tutorial introduttivo, sarà guidato nell’utilizzo dell’app \emph{Studenti Unimol}.\\ 
Le informazioni presentate sullo schermo sono in grado di indirizzare l’utente verso le funzionalità a cui desidera accedere, cercando di volta in volta di isolare soltanto le informazioni necessarie.\\
L'interfaccia dell' \textit{app Studenti} mostrerà messaggi di errore con una grafica piacevole.

\paragraph{Robustezza\\} 
Ogni volta che un’operazione dell’utente sul sistema presenta un insuccesso, si invia all’utente una notifica con un messaggio di errore dando la possibilità di riprovare ad effettuare l’operazione. In caso contrario il sistema notifica il successo dell’operazione. 
Il sistema sarà anche in grado di riconoscere le parole offensive contenute all’interno di un messaggio e di non far visualizzare quest’ultimo all’interno della \emph{chat} a cui era destinato, in modo da rendere quanto più decorosa la conversazione. 

\paragraph{Performance\\} 
Il sistema dovrà ridurre al minimo i tempi di risposta in modo da rendere più piacevole e fluido l’utilizzo da parte dell’utente. Inoltre deve essere in grado di gestire più richieste da parte di diversi utenti contemporaneamente.

\paragraph{Supportabilità\\} 
Il sistema dovrà garantire semplicità nelle attività di manutenzione, risoluzione degli errori, evoluzione e aggiunta di nuove funzionalità o modifica di quelle esistenti.
Queste attività saranno garantite da aggiornamenti del \emph{software} che riguarderanno sia l’aggiunta di nuove funzionalità che l'utilizzo di nuove tecnologie per far fronte a difetti del sistema.

\paragraph{Affidabilità\\} 
Gran parte delle operazioni eseguibili sul sistema devono essere disponibili anche senza connessione \emph{Internet}. Tutto il sistema si basa sugli ultimi dati salvati nello \textit{storage}, aggiornati periodicamente.Tuttavia, il sistema richiede la connessione \emph{internet} per alcune funzionalità.

\paragraph{Sicurezza e privacy\\} 
Il sistema permetterà agli utenti del sistema l’accesso tramite credenziali. Le aree riservate e i dati sensibili saranno memorizzati sul dispositivo rispettando la normativa vigente in materia di protezione dei dati personali(\textit{GDPR}). Nella fattispecie, il sistema sarà in grado di proteggere i dati sensibili contenuti al suo interno e proteggersi dalle vulnerabilità presenti nel codice.

\paragraph{Tracciabilità\\} 
Il sistema garantirà un \emph{report} basato sull’orario di invio e conserverà informazioni delle conversazioni relative alle \emph{chat} dei corsi, al fine di poter effettuare una eventuale verifica futura.\\

\section{Pseudorequisiti e vincoli}
%%%%%%%%%%%%% 7.3.1 Implementazione %%%%%%%%%%%%%
\subsection{Implementazione}
\paragraph{}
L’app dovrà essere realizzata tramite l’utilizzo del framework \textit{Ionic}, versione 4, che comprende tecnologie come \emph{TypeScript}, \emph{Angular} e \emph{CSS}.
Per il \emph{back-end} sarà utilizzato \emph{PHP}.\\
Per il pannello di amministrazione, l'applicativo software sarà implementato con:\\
\emph{HTML}, \emph{CSS}, \emph{JavaScript} per il \emph{front-end} e \emph{PHP} per il \emph{back-end}.


%%%%%%%%%%%%% 7.3.2 Interafccia %%%%%%%%%%%%%
\subsection{Interfaccia}
\paragraph{}
L’app si appoggerà a servizi esterni offerti dall’Università utilizzando un suo sincronizzatore costruito \textit{ad hoc} per le chiamate a diversi portali.

%%%%%%%%%%%%% 7.3.3 Packaging %%%%%%%%%%%%%
\subsection{Packaging}
\paragraph{}
L’app potrà essere installata ed eseguita su tutti i dispositivi \textit{Andorid 4.4+} e \textit{iOS 9+}.

%%%%%%%%%%%%% 7.3.4 Legali %%%%%%%%%%%%%
\subsection{Legali}
\paragraph{}
L’utilizzo dell’app non comporterà il pagamento di alcuna royalty da parte dello studente.


\clearpage