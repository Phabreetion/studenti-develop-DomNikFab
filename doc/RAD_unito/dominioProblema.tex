%%%%%%%%%%%%%%%%%%%%%%%%%%%%%%%%%%%%%h%%%%%%%%%%%%%%%%%%%%%%

\chapter{Dominio del problema}
\label{ref:Introduzione}
Sezione comune a tutti i gruppi: la curerà qualcuno di noi. Non rivolta ai singoli gruppi.
%%%%%%%%%%%%%%%%%%%%%%%%%%%%%%%%%%%%%%%%%%%%%%%%%%%%%%%%%%%

\section{Introduzione al RAD}

\paragraph{}
Questo \textit{Requirement Analysis Document} ha lo scopo di descrivere la fase di raccolta e analisi dei requisiti funzionali, non funzionali e pseudo-requisiti dell'applicazione \textit{Studenti Unimol} quale sistema da sviluppare per gli esami di \textit{Ingegneria del software e laboratorio} e \textit{Gestione progetti software} previsti dai piani di studio del \textit{Corso di studio unificato in Informatica}. Tale documento funge da contratto tra il prof. Fausto Fasano che ha commissionato il sistema e i team di progettazione e sviluppo che lo realizzeranno. Esso fornisce una panoramica astratta del sistema che gli studenti si accingeranno ad implementare.

\section{Scope}

\paragraph{}
Questo progetto ha come scopo quello di realizzare una App ufficiale per l’Università degli Studi del Molise. L’obiettivo è quello di rendere disponibile, agli studenti universitari, i principali servizi di Ateneo attraverso un’applicazione completa ed esaustiva.
In primo luogo, dal punto di vista informativo ed organizzativo, il sistema deve essere capace di focalizzarsi sull’esigenza comune agli studenti di accedere velocemente alle informazioni e ai dati afferenti al proprio Piano di Studio. In tal modo, lo studente può monitorare in modo agevole ed intuitivo tutto ciò che attiene al proprio percorso universitario e alle principali attività svolte all'interno dell'Ateneo.

La tecnologia usata facilita estremamente l'utilizzo dell'applicazione. Si tratta di un’app ibrida multipiattaforma, che permette di avere alte performance su ciascun dispositivo mobile, in particolar modo su smartphone \textit{iOS} e \textit{Android}.  L’app supporta funzionalità di base ma anche funzionalità avanzate, le quali garantiscono un servizio aggiuntivo per gli studenti che decidono di utilizzarla. Lo studente può visualizzare tutti i suoi dati universitari e compiere una serie di attività, accedendo con le sue credenziali del portale \textit{Esse3} alla propria area personale. Tra le funzionalità avanzate, introdotta nell'anno accademico 2018/2019, è presente un servizio di messagistica instantaneo.
L'idea di base è quella di fornire un sistema di comunicazione semplice ed affidabile, evitando la trasmissione di informazioni personali, quali i contatti personali, come numero di telefono ed email, di docenti e studenti.
Le chat sono create automaticamente in base alla \textit{Coorte accademica} e ai relativi \textit{corsi} e sono manutenute per un periodo limitato di tempo. La chat verrà abilitata per ciascuno studente o docente nel momento in cui egli fa il login nell’\textit{app}.


\section{Contesto e panoramica del sistema}
Il corso di \textit{Ingegneria del software e laboratorio} all'\textit{Università degli Studi del Molise} prevede la suddivisione degli studenti in gruppi di lavoro ai quali è chiesto di progettare, documentare e sviluppare alcune funzionalità interne all'applicazione \textit{Studenti Unimol} utilizzata dagli studenti per la gestione semplificata della loro carriera accademica. Sarà rilasciata una versione aggiornata di quella attuale previa revisione della versione già esistente. Essa sarà resa disponibile a tutti gli studenti regolarmente iscritti all'\textit{Università degli Studi del Molise}.

\section{Manager di progetto e sviluppatori}

\paragraph{}
L'app \textit{Studenti Unimol} è stata documentata e implementata nel 2017 dagli studenti del corso di \textit{Ingegneria del Software}, coordinati da gruppi di manager del corso magistrale in \textit{Sicurezza dei sistemi software}, operanti nell'ambito dell'insegnamento di \text{Gestione progetti software}. Successivamente l'applicazione è stata manutenuta dal prof. Fausto Fasano, il quale, nel 2019, ha chiesto agli studenti di triennale e magistrale di manutenere e far evolvere l'attuale applicazione per produrre la versione 3.0. Ogni team è supervisionato da alcuni manager che si occuperanno della comunicazione e dell'organizzazione dell'intero progetto, coordinando i gruppi a loro assegnati e gestendo le relazioni orizzontali tra i gruppi che lavorano in parallelo su funzionalità diverse dell'app. \\

\textbf {Gruppo 1 - Funzionalità: piano di studi, appelli, materiale didattico} \\ \\
\textbf{Manager} \\
Fantini Martina (PM), Fierro Fabiana (PM), Varriano Giulia (QM), Mastropaolo Antonio (SM). \\
\textbf{Ingegneri del software} \\
Caserio Walter, Ciaramella Giovanni, Daniele Raffaele, De Turris Antonio, Discenza Christian, Fagnano Stefano, Iannotti Carmine,  Muccigrosso Marco, Russodivito Marco, Tata Giancarlo. \\

\textbf{Gruppo 2 - Funzionalità: gestione notifiche e gestione orario} \\ \\
\textbf{Manager} \\
Piedimonte Massimo (PM), Placella Davide (PM), La Rocca Piera Elena (QM), Polisena Alessandro Bruno (QM), Di Tommaso Fabio (SM). \\
\textbf{Ingegneri del software} \\
Armenti Carmen, Buro Martina, Cocozza Nicola, Discenza Silvia, Lucchetti Roberto, Mazzocco Giuseppina, Schiavone Raffaele, Siravo Luca, Spina Chiara, Varrati Angelo Gino. \\

\textbf {Gruppo 3 - Funzionalità: previsione della media} \\ \\
\textbf{Manager} \\Crincoli Giuseppe (PM), Fabrizio Emilio (PM), D'Ercole Giovanna (QM), Di Cristino Gianluca (SM). \\
\textbf{Ingegneri del software} \\Cancelliere Alessandro, Di Pilla Francesco, Maglioli Raffaele, Marzullo Alessio, Placella Andrea, Stefanelli Vincenzo, Venditti Giorgio.\\

\textbf {Gruppo 4 - Funzionalità: news} \\ \\
\textbf{Manager} \\
Fausto Fasano. \\ \\ \\ \\

\textbf {Gruppo 5 - Funzionalità: rubrica} \\ \\
\textbf{Manager} \\
Carnevale Filippo (PM), Marinaro Tiziano (QM), Tortola Domenico (SM). \\
\textbf{Ingegneri del software} \\
Ciardiello Gabriele, Pizzi Mario, Vitiello Giò. \\

\textbf{Gruppo 6 - Funzionalità: chat per studenti e docenti con annesso pannello di amministrazione} \\ \\
\textbf{Manager} \\
\textbf{} 
Giovanni Rosa (PM), Michele Guerra (QM), Angelo Iallonardi (SM). \\
\textbf{Ingegneri del software } \\
Daniele Albanese, Antonio Antenucci, Mattia Ciccaglione, Andrea D'Aguanno, Antonio De Santis,  Francesco Di Rito, Gianluca Farinaro, Antonio Fratianni, Emanuela Guglielmi, Gaia Iannone, Aldo Palombo, Marica Principe, Federico Zappone.

\clearpage
\section{Glossario}

\paragraph{}
Di seguito è riportata una tabella degli acronimi e dei termini tecnici utilizzati nel documento:

\begin{table}[!h]
\begin{tabular}{p{1.5in}|p{4in}} \\
	{\bf Termine o sigla} & {\bf Descrizione} \\ \hline
	\textbf{RAD} & \textit{Requirement Analysis Document}, documento di analisi dei requisiti. \\
	\textbf{Attore} &  Entità esterna al sistema che interagisce con
	esso. In questo documento la definizione degli attori viene trattata al paragrafo \ref{sec:attori} \\
	\textbf{Modello di workflow} & Modello del flusso di lavoro: indica il modello di sistema che viene utilizzato dagli sviluppatori durante la fase di analisi dei requisiti e implementazione del sistema. \\
	\textbf{Scenario} & Singola istanza di un caso d’uso. Descrive concretamente una situazione ipotetica che potrebbe verificarsi all'avverarsi di un caso d’uso. \\
	\textbf{Caso d'uso} & Flusso di eventi che coinvolge il sistema e alcuni attori che si genera quando  un attore interagisce con il sistema e, poiché sono soddisfatte delle condizioni di ingresso e dei vincoli, vengono eseguite alcune azioni affinché il flusso termini con una o più condizioni d’uscita. \\
	\textbf{Diagramma di sequenza} & Rappresenta un diagramma che serve per descrivere uno scenario in cui le scelte o i flussi alternativi sono rappresentati tramite un pannello di "alt". In particolare servono per dare una visione specifica del flusso degli eventi a livello applicativo. \\
	\textbf{Screen mockup} & Rappresenta un prototipo dell’interfaccia che il sistema presenterà allo studente e permette al lettore di immaginare la veste grafica dell’applicazione e l’implementazione di uno specifico caso d’uso. \\
\end{tabular}
\end{table}
\clearpage