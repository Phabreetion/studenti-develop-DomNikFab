%%%%%%%%%%%%%%%%%%%%%%%%%%%%%%%%%%%%%%%%%%%%%%%%%%%%%%%%%%%

\chapter{Modello del sistema}
\label{ref:modSistema}

%%% Il gruppo 1 scriverà il suo modello del sistema. Esso dovrà includere: attori, casi d'uso (descrizione e tabella), scenari, diagrammi dei casi d'uso, diagrammi di sequenza, diagramma delle attività, screen mockups della funzionalità %%%

\section{Attori}
Descrivere gli attori che partecipano ai casi d'uso. Se un attore partecipa solo ad alcuni casi d'uso, specificare a quali partecipa, inserendo anche una referenza con \ref{nome_label} qui e con \label{nome_label} sui paragrafi che descrivono quel caso d'uso. 

//Per la chat aggiungere alla descrizione di Studente la visione dello studente che ha il gruppo chat (gruppo 6), inserendo la referenza alla loro parte.

\section{Scenari}
Inserire qui gli scenari di tutti i gruppi.

\section{Casi d'uso}
Inserire casi d'uso di tutti i gruppi.

\paragraph{Caso d'uso 1 (sostituire con nome caso d'uso) \\} 
Lorem ipsum dolor sit amet... (sostituire con descrizione caso d'uso)

\begin{table}[tb]
%\normalsize % Dimensione testo normale
\small % Dimensione testo piccola
%\footnotesize % Dimensione testo piccolissima
%\scriptsize % Dimensione del testo ulteriormente più piccola
%\caption{} % Didascalia tabella
%\label{} % Etichetta per riferimenti incrociati
\begin{tabular}{| p{\useCaseLeft} | p{\useCaseNum} | p{\useCaseTwoCol} | p{\useCaseTwoCol} |}
	\hline
	\textbf{Nome caso d'uso} & \multicolumn{3}{p{\useCaseMulticol} |}{\textbf{Login}} \\
	\hline
	\textbf{Attori partecipanti} & \multicolumn{3}{p{\useCaseMulticol} |}{Inizializzato da \textbf{Utente}.} \\
	\hline
	\textbf{Condizioni d'ingresso} & \multicolumn{3}{p{\useCaseMulticol} |}{L'utente ha cliccato sul bottone di login.} \\
	\hline
	\textbf{Flusso degli eventi} & \textbf{\#} & \textbf{Utente} & \textbf{Sistema} \\
	\hline
	\textbf{} & \textbf{1} & \textbf{} & Propone una schermata per l'inserimento dei dati necessari per il login, e-mail e password dell'utente \\
	\hline
	\textbf{} & \textbf{2} & Inserisce i dati e sottomette la richiesta & \textbf{} \\
	\hline
	\textbf{} & \textbf{3} & \textbf{} & Controlla che siano stati inseriti entrambi i campi e avvia le operazioni di visualizzazione \\
	\hline
	\textbf{Eccezioni} & \multicolumn{3}{p{\useCaseMulticol} |}{3.1 Uno o entrambi i campi sono vuoti.\newline 3.2 Le credenziali inserite non sono valide (una o entrambe).} \\
	\hline
	\textbf{Condizioni d'uscita} & \multicolumn{3}{p{\useCaseMulticol} |}{Il sistema completa la login e dà accesso all'app o, in caso contrario, visualizza un messaggio di errore se non sono stati inseriti tutti i dati obbligatori, se le credenziali non sono corrette o se si verifica un insuccesso dell'operazione.} \\
	\hline
\end{tabular}
\end{table}

\section{Diagramma dei casi d'uso}

Inserire diagramma casi d'uso di tutti i gruppi.

\begin{figure}
	\centering
	\includegraphics[height=3in]{imgs/file-comuni-ai-gruppi/useCaseEsempio.png}
	\caption{Inserire descrizione}
	\label{fig:prova}
\end{figure}

\section{Diagramma di sequenza}

Inserire diagrammi di sequenza di tutti i gruppi.

\begin{figure}
	\centering
	\includegraphics[height=3in,width=5in]{imgs/file-comuni-ai-gruppi/SequenceDgEsempio.png}
	\caption{Inserire descrizione}
	\label{fig:prova}
\end{figure}

\section{Diagramma delle attività}

Inserire diagrammi di activity di tutti i gruppi.

\begin{figure}
	\centering
	\includegraphics[height=3in,width=5in]{imgs/file-comuni-ai-gruppi/ActivityDgEsempio.png}
	\caption{Inserire descrizione}
	\label{fig:prova}
\end{figure}

\clearpage