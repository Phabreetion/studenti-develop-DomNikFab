%%%%%%%%%%%%%%%%%%%%%%%%%%%%%%%%%%%%%%%%%%%%%%%%%%%%%%%%%%%

\chapter{Analisi dei requisiti - gruppo 1}
\label{ref:requisiti1}

%%% Il gruppo 1 scriverà qui i suoi requisiti funzionali e non funzionali %%%

\section{Requisiti funzionali}

%%%%%%%%%%%%%%7.1.1 Gestione piano di studio%%%%%%%%%%%%%%%%%
\subsection{Gestione piano di studio}
\paragraph{} 
L’app dovrà mostrare i corsi previsti dal piano di studio dello studente permettendo la visualizzazione di tutti i corsi ad esso afferenti evidenziando quelli per cui l’esame è stato sostenuto e quelli per cui l’esame è da sostenere. Per ogni corso saranno inoltre visualizzati i relativi dettagli, come il numero di CFU, la valutazione in trentesimi oppure eventuale idoneità. Lo studente potrà effettuare le operazioni di ricerca, filtro e ordinamento dell'elenco dei corsi e potrà scegliere se memorizzare nello \textit{storage} le sue preferenze oppure resettarle. Il sistema richiederà i dati aggiornati al sincronizzatore, il quale si occuperà di salvarli nello \textit{storage} dell’app.
%%%%%%%%%%%7.1.2 Visualizza dettagli corso%%%%%%%%%%%%%%%%%%%
\subsection{Visualizza dettagli corso}
\paragraph{} 
L’app dovrà mostrare informazioni relative a ciascun corso cliccando sullo stesso nella sezione carriera. Selezionando la voce \textit{dettagli}, l’app mostrerà allo studente, per ogni corso, il/i docente/i responsabile/i dell’insegnamento, il numero di CFU, l’anno accademico in cui viene frequentato il corso ed i suoi contenuti. Da tale sezione, inoltre, si potranno raggiungere le interfacce del materiale didattico e dell’elenco appelli. Selezionando un corso il cui esame è già stato sostenuto sarà, invece, possibile ottenere anche informazioni relative alla data in cui è stato svolto, alla data in cui è stato verbalizzato ed il voto ottenuto. Il sistema richiederà i dati aggiornati al sincronizzatore, il quale si occuperà di salvarli nello \textit{storage} dell’app.
%%%%%%%%%%%7.1.3 Gestione materiale didattico %%%%%%%%%%%%%
\subsection{Gestione materiale didattico}
\paragraph{} 
L’app mostrerà l’elenco dei file relativi ad un corso selezionato, dopo aver richiesto i dati al sincronizzatore, averli ricevuti e salvati nello \textit{storage} dell’applicazione. Per visualizzare un file selezionato dalla lista, questo dovrà essere prima scaricato all’interno dello \textit{storage} dell’app: in tal caso, il file potrà essere aperto e visualizzato. Nel caso in cui il file selezionato fosse stato precedentemente scaricato sarà possibile eliminarlo, altrimenti quest’ultima notificherà allo studente l’assenza del file nello \textit{storage} e gli chiederà se è intenzionato a scaricarlo. Se lo studente sceglierà di scaricare un file, il sistema richiederà i dati aggiornati al sincronizzatore, il quale si occuperà di salvarli nello \textit{storage} dell’app.
%%%%%%%%%%%%%%%%%% 7.1.4 Gestione appelli %%%%%%%%%%%%%%%%%%%
\subsection{Gestione appelli}
\paragraph{} 
L’app dovrà mostrare l’elenco degli appelli disponibili richiedendoli al sincronizzatore e salvando nello \textit{storage} i dati ricevuti: lo studente potrà prenotarsi a uno specifico appello tra quelli visualizzati, che saranno solo quelli prenotabili. Lo studente potrà effettuare le operazioni di ricerca, filtro con parola chiave e ordinamento dall'elenco di appelli disponibili e scegliere se memorizzare nello \textit{storage} le sue preferenze oppure resettarle. Il sistema richiederà i dati aggiornati al sincronizzatore, il quale si occuperà di salvarli nello \textit{storage} dell’app. Lo studente, selezionando una data di appello, potrà effettuare una prenotazione, la quale verrà inserita dal sistema nell’elenco degli appelli prenotati. La prenotazione potrà essere annullata fino a cinque giorni prima della data di esame.


\section{Requisiti non funzionali}

%%%%%%%%%%%%% 7.2.1 Usabilità e user experience %%%%%%%%%%%%%
\subsection{Usabilità e \textit{user experience}}
\paragraph{} 
L’app dovrà essere semplice da utilizzare per agevolare ogni studente dell’\textit{Università degli Studi del Molise} nel monitoraggio della propria carriera universitaria. L’applicazione, inoltre, sarà dotata di un’interfaccia \textit{user friendly} per essere utilizzata e compresa con facilità. A tal proposito la prima volta che l’app sarà aperta verrà mostrato un breve tutorial che offrirà una panoramica veloce riguardante le funzionalità principali. L’app mostrerà messaggi di errore utilizzando una grafica piacevole.

\section{Pseudorequisiti e vincoli}
\paragraph{Sostituire con gli pseudorequisiti e vincoli da rispettare \\}
Lorem ipsum dolor sit amet

\clearpage
