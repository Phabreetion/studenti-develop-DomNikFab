%%%%%%%%%%%%%%%%%%%%%%%%%%%%%%%%%%%%%%%%%%%%%%%%%%%%%%%%%%%

\chapter{Analisi dei requisiti - gruppo 1}
\label{ref:requisiti1}

%%% Il gruppo 1 scriverà qui i suoi requisiti funzionali e non funzionali %%%

\section{Requisiti funzionali}

%%%%%%%%%%%%%%%%%%%%%%%%%%%%%%%%%%%%%%%%%%%%%%%%%%%%%%%%%%%%
\subsection{Gestione piano di studio}
\paragraph{} 
L’app dovrà mostrare i corsi previsti dal piano di studio dello studente permettendo la visualizzazione di tutti i corsi ad esso afferenti evidenziando quelli per cui l’esame è stato sostenuto e quelli per cui l’esame è da sostenere. Per ogni corso saranno inoltre visualizzati i relativi dettagli, come il numero di CFU, la valutazione in trentesimi oppure eventuale idoneità. Lo studente potrà effettuare le operazioni di ricerca, filtro e ordinamento dell'elenco dei corsi e potrà scegliere se memorizzare nello \textit{storage} le sue preferenze oppure resettarle. Il sistema richiederà i dati aggiornati al sincronizzatore, il quale si occuperà di salvarli nello \textit{storage} dell’app.
%%%%%%%%%%%%%%%%%%%%%%%%%%%%%%%%%%%%%%%%%%%%%%%%%%%%%%%%%%%%
\subsection{Visualizza dettagli corso}
\paragraph{} 
L’app dovrà mostrare informazioni relative a ciascun corso cliccando sullo stesso nella sezione carriera. Selezionando la voce \textit{dettagli}, l’app mostrerà allo studente, per ogni corso, il/i docente/i responsabile/i dell’insegnamento, il numero di CFU, l’anno accademico in cui viene frequentato il corso ed i suoi contenuti. Da tale sezione, inoltre, si potranno raggiungere le interfacce del materiale didattico e dell’elenco appelli. Selezionando un corso il cui esame è già stato sostenuto sarà, invece, possibile ottenere anche informazioni relative alla data in cui è stato svolto, alla data in cui è stato verbalizzato ed il voto ottenuto. Il sistema richiederà i dati aggiornati al sincronizzatore, il quale si occuperà di salvarli nello \textit{storage} dell’app.


\section{Requisiti non funzionali}

\paragraph{Requisito 1 (sostituire con nome requisito) \\} 
Lorem ipsum dolor sit amet

\section{Pseudorequisiti e vincoli}
\paragraph{Sostituire con gli pseudorequisiti e vincoli da rispettare \\}
Lorem ipsum dolor sit amet

\clearpage
