%%%%%%%%%%%%%%%%%%%%%%%%%%%%%%%%%%%%%%%%%%%%%%%%%%%%%%%%%%%

\chapter{Analisi dei requisiti - gruppo 1}
\label{ref:requisiti1}

%%% Il gruppo 1 scriverà qui i suoi requisiti funzionali e non funzionali %%%

\section{Requisiti funzionali}
\subsection{Gestione piano di studio}
\paragraph{} 
L’app dovrà mostrare i corsi previsti dal piano di studio dello studente permettendo la visualizzazione di tutti i corsi ad esso afferenti evidenziando quelli per cui l’esame è stato sostenuto e quelli per cui l’esame è da sostenere. Per ogni corso saranno inoltre visualizzati i relativi dettagli, come il numero di CFU, la valutazione in trentesimi oppure eventuale idoneità. Lo studente potrà effettuare le operazioni di ricerca, filtro e ordinamento dell'elenco dei corsi e potrà scegliere se memorizzare nello \textit{storage} le sue preferenze oppure resettarle. Il sistema richiederà i dati aggiornati al sincronizzatore, il quale si occuperà di salvarli nello \textit{storage} dell’app.


\section{Requisiti non funzionali}

\paragraph{Requisito 1 (sostituire con nome requisito) \\} 
Lorem ipsum dolor sit amet

\section{Pseudorequisiti e vincoli}
\paragraph{Sostituire con gli pseudorequisiti e vincoli da rispettare \\}
Lorem ipsum dolor sit amet

\clearpage
