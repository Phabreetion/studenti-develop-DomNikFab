%%%%%%%%%%%%%%%%%%%%%%%%%%%%%%%%%%%%%h%%%%%%%%%%%%%%%%%%%%%%

\chapter{Dominio del problema}
\label{ref:Introduzione}
Sezione comune a tutti i gruppi: la curerà qualcuno di noi. Non rivolta ai singoli gruppi.
%%%%%%%%%%%%%%%%%%%%%%%%%%%%%%%%%%%%%%%%%%%%%%%%%%%%%%%%%%%

\section{Introduzione al RAD}

\paragraph{}
Questo \textit{Requirement Analysis Document} ha lo scopo di descrivere la fase di raccolta e analisi dei requisiti funzionali, non funzionali e pseudo-requisiti dell'applicazione \textit{Studenti Unimol} quale sistema da sviluppare per gli esami di \textit{Ingegneria del software e laboratorio} e \textit{Gestione progetti software} previsti dai piani di studio del \textit{Corso di studio unificato in Informatica}. Tale documento funge da contratto tra il prof. Fausto Fasano che ha commissionato il sistema e i team di progettazione e sviluppo che lo realizzeranno. Esso fornisce una panoramica astratta del sistema che gli studenti si accingeranno ad implementare.

\section{Scope}

\paragraph{}
Lorem ipsum dolor sit amet... //TO DO!

\section{Contesto e panoramica del sistema}
Il corso di \textit{Ingegneria del software e laboratorio} all'\textit{Università degi Studi del Molise} prevede la suddivisione degli studenti in gruppi di lavoro ai quali è chiesto di progettare, documentare e sviluppare alcune funzinalità interne all'applicazione \textit{Studenti Unimol} utilizzata dagli studenti per la gestione semplificata della loro carriera accademica. Sarà rilasciata una versione aggiornata di quella attuale previa revisione della versione già esistente. Essa sarà resa disponibile a tutti gli studenti regolarmente iscritti all'\textit{Università degli Studi del Molise}.

\section{Manager di progetto e sviluppatori}

\paragraph{}
L'app \textit{Studenti Unimol} è stata documentata e implementata nel 2017 dagli studenti del corso di \textit{Ingegneria del Software}, coordinati da gruppi di manager del corso magistrale in \textit{Sicurezza dei sistemi software}, operanti nell'ambito dell'insegnamento di \text{Gestione progetti software}. Successivamente l'applicazione è stata manutenuta dal prof. Fausto Fasano, il quale, nel 2019, ha chiesto agli studenti di triennale e magistrale di manutenere e far evolvere l'attuale applicazione per produrre la versione 3.0. Ogni team è supervisionato da alcuni manager che si occuperanno della comunicazione e dell'organizzazione dell'intero progetto, coordinando i gruppi a loro assegnati e gestendo le relazioni orizzontali tra i gruppi che lavorano in parallelo su funzionalità diverse dell'app.

Di seguito saranno elencati i componenti dei singoli gruppi e ogni gruppo sarà associato alle funzionalità che documenta e implementa. Nello specifico, per ogni gruppo saranno elencati i manager e gli Ingegneri del Software. Saranno adottate le seguenti sigle per classificare i ruoli dei manager: PM per i \textit{Project manager}, QM per i \textit{Quality manager} e SM per i \textit{Security manager}. \newline

//TO DO: descrizione dei gruppi e allocazione sulle funzionalità.

//OGNI GRUPPO INSERISCA DI SEGUITO I SUOI NOMI, COME FATTO DAL GRUPPO 1 \\

\textbf {Gruppo 1 - Funzionalità: piano di studi, appelli, materiale didattico} \\ \\
\textbf{Manager} \\
Fantini Martina (PM), Fierro Fabiana (PM), Varriano Giulia (QM), Mastropaolo Antonio (SM). \\
\textbf{Ingegneri del software} \\
Caserio Walter, Ciaramella Giovanni, Daniele Raffaele, De Turris Antonio, Discenza Christian, Fagnano Stefano, Iannotti Carmine,  Muccigrosso Marco, Russodivito Marco, Tata Giancarlo.

\textbf{Gruppo 6 - Funzionalità: Chat per studenti e docenti con annesso pannello di amministrazione} \\ \\
\textbf{Manager:} \\
Giovanni Rosa (PM),
Michele Guerra (QM),
Angelo Iallonardi (SM). \\
\textbf{Ingegneri del software:} \\
Aldo Palombo,
Andrea D'Aguanno,
Antonio De Santis,
Antonio Antenucci,
Antonio Fratianni,
Daniele Albanese,
Emanuela Guglielmi,
Federico Zappone,
Francesco di Rito,
Gaia Iannone,
Gianluca Farinaro,
Marica Principe,
Mattia Ciccaglione.

\textbf {Gruppo 2 - Funzionalità: gestione notifiche e gestione orario} \\ \\
\textbf{Manager} \\
Di Tommaso Fabio, La Rocca Piera Elena, Placella Davide,Piedimonte Massimo, Polisena Alessandro. \\
\textbf{Ingegneri del software} \\
Armenti Carmen, Buro Martina, Cocozza Nicola, Discenza Silvia, Lucchetti Roberto, Mazzocco Giuseppina, Schiavone Raffaele, Siravo Luca, Spina Chiara, Varrati Angelo Gino.

\section{Glossario}

\paragraph{}
Di seguito è riportata una tabella degli acronimi e dei termini tecnici utilizzati nel documento:

\begin{table}
\begin{tabular}{p{1.5in}|p{4in}} \\
	{\bf Termine o sigla} & {\bf Descrizione} \\ \hline
	\textbf{RAD} & \textit{Requirement Analysis Document}, documento di analisi dei requisiti. \\
	\textbf{Attore} &  Entità esterna al sistema che interagisce con
	esso. In questo documento la definizione degli attori viene trattata al paragrafo ... [TO DO: INSERIRE REFERENZA AL PARAGRAFO] \\
	\textbf{Modello di workflow} & Modello del flusso di lavoro: indica il modello di sistema che viene utilizzato dagli sviluppatori durante la fase di analisi dei requisiti e implementazione del sistema. \\
	\textbf{Caso d'uso} & Flusso di eventi che coinvolge il sistema e alcuni attori che si genera quando  un attore interagisce con il sistema e, poiché sono soddisfatte delle condizioni di ingresso e dei vincoli, vengono eseguite alcune azioni affinché il flusso termini con una o più condizioni d’uscita. \\
	\textbf{Scenario} & Singola istanza di un caso d’uso. Descrive
	concretamente una situazione ipotetica che potrebbe verificarsi all'avverarsi di un caso d’uso. \\
	\textbf{Screen mockup} & Rappresenta un prototipo dell’interfaccia che il sistema presenterà allo studente e permette al lettore di immaginare la veste grafica dell’applicazione e l’implementazione di uno specifico caso d’uso. \\
\end{tabular}
\end{table}
\clearpage