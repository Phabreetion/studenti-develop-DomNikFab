%%%%%%%%%%%%%%%%%%%%%%%%%%%%%%%%%%%%%%%%%%%%%%%%%%%%%%%%%%%

\chapter{Analisi dei requisiti - gruppo 2 - orario}
\label{ref:requisiti2-orario}

La funzionalità “Gestione orario” consentirà allo studente di visualizzare l’orario costituito dai corsi previsti dal Piano di Studio a seconda dell’anno accademico e dell’aula in cui questi si svolgeranno.  
Lo studente avrà la possibilità di gestire l’organizzazione del proprio orario a seconda dei corsi che deciderà di frequentare. Sarà possibile infatti eliminare dall’orario relativo al proprio anno accademico i corsi che lo studente non seguirà e aggiungere, eventualmente, insegnamenti relativi ad anni precedenti. 


\section{Requisiti funzionali}

\subsection{Gestione primo avvio}
\paragraph{} 
La funzionalità consente allo studente di selezionare i corsi relativi al proprio Corso di Studi e l’aula in cui si svolgeranno i corsi. 

\subsection{Modifica orario } 
\paragraph{}
La funzionalità consente allo studente di inserire ed eliminare dall’orario i corsi che non si vuole visualizzare. 

\subsection{Visualizza orario }
\paragraph{}
La funzionalità consente allo studente di visualizzare l’orario dei corsi da lui selezionati. 

\section{Requisiti non funzionali}

\subsection{Prestazioni del sistema }
\paragraph{} 
Il sistema non richiede un particolare livello di prestazioni, esso deve ridurre al minimo i tempi di risposta e deve essere in grado di gestire più richieste da parte di diversi utenti contemporaneamente. 

\subsection{Affidabilità }
\paragraph{}
Gran parte delle operazioni eseguibili sul sistema devono essere disponibili anche senza connessione Internet. Questo risultato si ottiene basandosi su file locali che vengono aggiornati periodicamente e ad ogni login. Il sistema richiede tuttavia il collegamento Internet per alcune funzionalità che richiedono l’invio o la ricezione di dati per essere completate con successo. 

\subsection{Gestione degli errori e tolleranza ai guasti }
\paragraph{}
Ogni volta che un’operazione dell’utente sul sistema presenta un insuccesso, si invia all’utente una notifica con un messaggio di errore dando la possibilità di riprovare ad effettuare l’operazione. In caso contrario il sistema notifica il successo dell’operazione. 

\subsection{Usabilità}
\paragraph{} 
Il sistema è stato progettato con un’interfaccia semplice ed intuitiva per rendere l’utilizzo fruibile a qualsiasi utente. 

\subsection{Protezione }
\paragraph{} 
Il sistema è stato progettato per influire il meno possibile sull’hardware rimandando l’esecuzione di alcune funzionalità al server. 


\section{Pseudorequisiti e vincoli}
\paragraph{Sostituire con gli pseudorequisiti e vincoli da rispettare \\}


\clearpage