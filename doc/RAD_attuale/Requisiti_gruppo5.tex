%%%%%%%%%%%%%%%%%%%%%%%%%%%%%%%%%%%%%%%%%%%%%%%%%%%%%%%%%%%

\chapter{Analisi dei requisiti - gruppo 5}
\label{ref:requisiti5}

%%% Il gruppo 3 scriverà qui i suoi requisiti funzionali e non funzionali %%%

\section{Requisiti funzionali}

\paragraph{Visualizza Rubrica \\} 
La funzionalità permette di visualizzare una lista contatti aggiornata, contenente tutte le informazioni relative al personale Unimol. L’accesso alla rubrica è garantito agli studenti
registrati e che hanno effettuato l'accesso al sistema.
Lo studente scrive nella barra di ricerca il docente o il personale tecnico-amministartivo del quale vuole ottenere le informazioni di contatto, oppure, in alternativa, la sede didattica di cui vuole visualizzare i relativi contatti.
Se l'informazione non è presente nel database, verrà mostrato
il messaggio di mancato riscontro, se così non fosse saranno
mostrate le informazioni richieste dall'utente.

\paragraph{Visualizza Contatto \\}
La funzionalità permette di visualizzare le informazioni relative al contatto cercato: nome e cognome, dipartimento, sede didattica, email e numero telefonico.

\section{Requisiti non funzionali}

\paragraph{Usabilità \\} 
Il sistema fornisce un’interfaccia molto facile e molto intuitiva per agevolare e per velocizzare le operazioni.

\paragraph{Sicurezza \\} 
Il sistema garantisce la veridicità all’autoautenticazione dello studente tramite l’applicazione. La gestione di questa è garantita dall’accesso univoco tramite la matricola e la password all’applicazione che ogni utente scaricherà e istallerà sul proprio dispositivo.

\paragraph{Performance \\} 
Il sistema deve essere in grado di riconoscere, all’incirca in 5 secondi, se sono presenti nel database le informazioni di interesse.

\paragraph{Flessibilità \\}
Il sistema deve poter essere utilizzato dal più esperto al meno esperto nel settore.

\paragraph{Modificabilità \\}
Il sistema viene concepito secondo un approccio modulare e scalabile del sorgente, garantendone la massima fluidità nell'applicare eventuali modifiche e miglioramenti.

\paragraph{Privacy \\}
Il sistema gestisce i dati raccolti riguardo gli utenti nel pieno rispetto delle normative vigenti in materia di trattamento dei dati personali.

\section{Pseudorequisiti e vincoli}

\paragraph{Implementazione \\}
Il sistema deve essere sviluppato in ambiente Ionic, una piattaforma di sviluppo di applicazioni ibride.


\clearpage