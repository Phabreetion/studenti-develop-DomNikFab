%%%%%%%%%%%%%%%%%%%%%%%%%%%%%%%%%%%%%%%%%%%%%%%%%%%%%%%%%%%

\chapter{Analisi dei requisiti - gruppo 3}
\label{ref:requisiti3}
La funzionalità consente allo studente di prevedere le medie, aritmetica e ponderata, e la base di laurea aggiornata, simulando il conseguimento di un esame con una determinata valutazione. La stima si basa sulle medie precedenti e sul numero di CFU conseguiti (idoneità escluse). Il sistema inoltre salverà la simulazione effettuata (se lo studente lo desidererà).

\section{Requisiti funzionali}

Lo studente, accede alla funzionalità previsione media, selezionando gli esami di suo interesse, in modo tale da effettuare una simulazione per prevedere un'ipotetica base di laurea, con relativa media aritmetica e ponderata.

\section{Requisiti non funzionali}

\subsection{Prestazioni del sistema} 

Il sistema non richiede un particolare livello di prestazioni, esso deve ridurre al minimo i tempi di risposta e deve essere in grado di gestire più richieste da parte di diversi utenti contemporaneamente.
\subsection{Gestione degli errori e tolleranza ai guasti}

Ogni volta che un’operazione dell’utente sul sistema presenta un insuccesso, si invia all’utente una notifica con un messaggio di errore dando la possibilità di riprovare ad effettuare l’operazione. In caso contrario il sistema notifica il successo dell’operazione.

\subsection{Sicurezza}

Per quel che riguarda la sicurezza dei dati che il sistema tratta, essa viene garantita attraverso un sistema a password. Ciascun utente del sistema accede alle funzionalità solo dopo aver inserito le proprie credenziali (login e password).

\subsection{Legali}
Il sistema deve essere realizzato nel rispetto della privacy degli utenti.

\subsection{Usabilità}
 
Il sistema è stato progettato con un’interfaccia semplice ed intuitiva per rendere l’utilizzo fruibile a qualsiasi utente.

\subsection{Performance}

Il sistema è stato progettato per influire il meno possibile sull’hardware rimandando l’esecuzione di alcune funzionalità al server.

\subsection{Interazione con sistemi esterni}
 
Il sistema si interfaccia con il sistema esterno Esse3 per quanto riguarda la gestione delle anagrafiche e della carriera dello studente.

\subsection{Affidabilità}
 
Gran parte delle operazioni eseguibili sul sistema devono essere disponibili anche senza connessione Internet. Tutto il sistema si basa su file locali che vengono aggiornati periodicamente e ad ogni login. Il sistema richiede tuttavia il collegamento internet per alcune funzionalità che richiedono l’invio o la ricezione di dati per essere completate con successo.
\begin{comment}
\section{Pseudorequisiti e vincoli}
\paragraph{Sostituire con gli pseudorequisiti e vincoli da rispettare \\}
Lorem ipsum dolor sit amet...
\end{comment}
