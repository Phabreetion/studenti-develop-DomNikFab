\section{Pseudorequisiti e vincoli}
%%%%%%%%%%%%% 7.3.1 Implementazione %%%%%%%%%%%%%
\subsection{Implementazione}
\paragraph{}
L’app dovrà essere realizzata tramite l’utilizzo del framework \textit{Ionic}, versione 4, che comprende tecnologie come \emph{TypeScript}, \emph{Angular} e \emph{CSS}.
Per il \emph{back-end} sarà utilizzato \emph{PHP}.\\
Per il pannello di amministrazione, l'applicativo software sarà implementato con:\\
\emph{HTML}, \emph{CSS}, \emph{JavaScript} per il \emph{front-end} e \emph{PHP} per il \emph{back-end}.


%%%%%%%%%%%%% 7.3.2 Interafccia %%%%%%%%%%%%%
\subsection{Interfaccia}
\paragraph{}
L’app si appoggerà a servizi esterni offerti dall’Università utilizzando un suo sincronizzatore costruito \textit{ad hoc} per le chiamate a diversi portali.

%%%%%%%%%%%%% 7.3.3 Packaging %%%%%%%%%%%%%
\subsection{Packaging}
\paragraph{}
L’app potrà essere installata ed eseguita su tutti i dispositivi \textit{Andorid 4.4+} e \textit{iOS 9+}.

%%%%%%%%%%%%% 7.3.4 Legali %%%%%%%%%%%%%
\subsection{Legali}
\paragraph{}
L’utilizzo dell’app non comporterà il pagamento di alcuna royalty da parte dello studente.

\clearpage