\subsection{Gestione materiale didattico}
\paragraph{} 
L’app mostrerà l’elenco dei file relativi ad un corso selezionato, dopo aver richiesto i dati al sincronizzatore, averli ricevuti e salvati nello \textit{storage} dell’applicazione. Per visualizzare un file selezionato dalla lista, questo dovrà essere prima scaricato all’interno dello \textit{storage} dell’app: in tal caso, il file potrà essere aperto e visualizzato. Nel caso in cui il file selezionato fosse stato precedentemente scaricato sarà possibile eliminarlo, altrimenti quest’ultima notificherà allo studente l’assenza del file nello \textit{storage} e gli chiederà se è intenzionato a scaricarlo. Se lo studente sceglierà di scaricare un file, il sistema richiederà i dati aggiornati al sincronizzatore, il quale si occuperà di salvarli nello \textit{storage} dell’app.
Il relativo diagramma dei casi d'uso è visibile al punto \ref{req:gestioneMaterialeDidattico}, dove sono esplicitati anche i riferimenti per gli altri diagrammi.
