\subsection{Visualizza dettagli corso}
\paragraph{} 
L’app dovrà mostrare informazioni relative a ciascun corso cliccando sull'apposita icona nella sezione piano di studio. Selezionando la voce \textit{dettagli corso}, l’app mostrerà allo studente, per ogni corso, il/i docente/i responsabile/i dell’insegnamento, il numero di CFU, l’anno accademico in cui viene frequentato il corso, i suoi contenuti, i testi consigliati, gli obiettivi formativi e le eventuali propedeuticità. Da tale sezione, inoltre, si potranno raggiungere le interfacce del materiale didattico e dell’elenco appelli. Selezionando un corso il cui esame è già stato sostenuto sarà, invece, possibile ottenere anche informazioni relative alla data in cui è stato svolto, alla data in cui è stato verbalizzato ed il voto ottenuto. Il sistema richiederà i dati aggiornati al sincronizzatore, il quale si occuperà di salvarli nello \textit{storage} dell’app.
Il relativo diagramma dei casi d'uso è visibile al punto \ref{req:visualizzaDettagliCorso}, dove sono esplicitati anche i riferimenti per gli altri diagrammi.
