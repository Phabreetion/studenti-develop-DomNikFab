\subsection{Gestione appelli}
\paragraph{} 
L’app dovrà mostrare l’elenco degli appelli disponibili richiedendoli al sincronizzatore e salvando nello \textit{storage} i dati ricevuti: lo studente potrà prenotarsi a uno specifico appello tra quelli visualizzati, che saranno solo quelli prenotabili. Lo studente potrà effettuare le operazioni di ricerca, filtro con parola chiave e ordinamento dall'elenco di appelli disponibili e scegliere se memorizzare nello \textit{storage} le sue preferenze oppure resettarle. Il sistema richiederà i dati aggiornati al sincronizzatore, il quale si occuperà di salvarli nello \textit{storage} dell’app. Lo studente, selezionando una data di appello, potrà effettuare una prenotazione, la quale verrà inserita dal sistema nell’elenco degli appelli prenotati. La prenotazione potrà essere annullata fino a cinque giorni prima della data di esame.
Il relativo diagramma dei casi d'uso è visibile al punto \ref{req:gestioneAppelli}, dove sono esplicitati anche i riferimenti per gli altri diagrammi.