\subsection{Assistenza}

\paragraph{Requisito 1: richiedi assistenza\\}
Il sistema, per risolvere tutte le problematiche degli utenti e per andare in contro ad ogni loro esigenza, prevederà la possibilità di comunicare con un gruppo di esperti che si occuperà della risoluzione delle problematiche.

\paragraph{Requisito 2: comunicazione problematiche\\}
Sarà disponibile per gli utenti una \emph{chat} di assistenza, disabilitata fino all’accettazione della prima richiesta di supporto, in cui sarà possibile descrivere il problema e comunicare con un gruppo di assistenti;

\paragraph{Requisito 3: gruppo \emph{supporter}\\}
L’assistenza sarà composta da un gruppo di utenti esperti con una conoscenza dell’app \emph{Studenti Unimol} tale da poter comprendere la natura delle problematiche e fornire istruzioni adeguate agli utenti richiedenti assistenza. 

\paragraph{Requisito 4: \emph{chat} \emph{supporter}\\}
Il gruppo di assistenti visualizzerà la \emph{chat} di assistenza diversamente dagli altri utenti, sarà presente un canale che conterrà le problematiche risolte e uno con quelle ancora irrisolte.

\paragraph{Requisito 5: risoluzione problematiche\\}
Per ogni problematica qualunque esperto sarà capace di comunicare direttamente con il richiedente assistenza e potrà inoltre aggiungere la problematica tra quelle risolte. 