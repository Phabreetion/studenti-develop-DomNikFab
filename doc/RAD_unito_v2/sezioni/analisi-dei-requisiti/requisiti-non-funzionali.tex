\section{Requisiti non funzionali}

Facendo riferimento al modello FURPS+ sono stati individuati i seguenti requisiti
non funzionali.

\paragraph{Funzionalità\\} 
Il sistema sarà in grado di fruire tutte le funzionalità richieste con la massima completezza dell’applicazione.

\paragraph{Usabilità\\} 
Il sistema presenterà un’interfaccia semplice ed essenziale che garantirà la completezza e la comprensibilità per un utilizzo facile ed intuitivo per il monitoraggio della propria carriera universitaria. Al primo avvio l’utente, tramite un tutorial introduttivo, sarà guidato nell’utilizzo dell’app \emph{Studenti Unimol}.\\ 
Le informazioni presentate sullo schermo sono in grado di indirizzare l’utente verso le funzionalità a cui desidera accedere, cercando di volta in volta di isolare soltanto le informazioni necessarie.\\
L'interfaccia dell' \textit{app Studenti} mostrerà messaggi di errore con una grafica piacevole.

\paragraph{Robustezza\\} 
Ogni volta che un’operazione dell’utente sul sistema presenta un insuccesso, si invia all’utente una notifica con un messaggio di errore dando la possibilità di riprovare ad effettuare l’operazione. In caso contrario il sistema notifica il successo dell’operazione. 
Il sistema sarà anche in grado di riconoscere le parole offensive contenute all’interno di un messaggio e di non far visualizzare quest’ultimo all’interno della \emph{chat} a cui era destinato, in modo da rendere quanto più decorosa la conversazione. 

\paragraph{Performance\\} 
Il sistema dovrà ridurre al minimo i tempi di risposta in modo da rendere più piacevole e fluido l’utilizzo da parte dell’utente. Inoltre deve essere in grado di gestire più richieste da parte di diversi utenti contemporaneamente.

\paragraph{Supportabilità\\} 
Il sistema dovrà garantire semplicità nelle attività di manutenzione, risoluzione degli errori, evoluzione e aggiunta di nuove funzionalità o modifica di quelle esistenti.
Queste attività saranno garantite da aggiornamenti del \emph{software} che riguarderanno sia l’aggiunta di nuove funzionalità che l'utilizzo di nuove tecnologie per far fronte a difetti del sistema.

\paragraph{Affidabilità\\} 
Gran parte delle operazioni eseguibili sul sistema devono essere disponibili anche senza connessione \emph{Internet}. Tutto il sistema si basa sugli ultimi dati salvati nello \textit{storage}, aggiornati periodicamente.Tuttavia, il sistema richiede la connessione \emph{internet} per alcune funzionalità.

\paragraph{Sicurezza e privacy\\} 
Il sistema permetterà agli utenti del sistema l’accesso tramite credenziali. Le aree riservate e i dati sensibili saranno memorizzati sul dispositivo rispettando la normativa vigente in materia di protezione dei dati personali(\textit{GDPR}). Nella fattispecie, il sistema sarà in grado di proteggere i dati sensibili contenuti al suo interno e proteggersi dalle vulnerabilità presenti nel codice.

\paragraph{Tracciabilità\\} 
Il sistema garantirà un \emph{report} basato sull’orario di invio e conserverà informazioni delle conversazioni relative alle \emph{chat} dei corsi, al fine di poter effettuare una eventuale verifica futura.\\