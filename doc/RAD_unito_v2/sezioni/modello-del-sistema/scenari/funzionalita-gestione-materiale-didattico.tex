\subsection{Funzionalità Gestione materiale didattico}
\paragraph{Visualizza elenco file}
\begin{itemize}
	\item \textit{Visualizzazione elenco file avvenuta con successo:}
	Lo studente Giuseppe, iscritto al terzo anno della facoltà di Informatica, dopo aver visualizzato l'elenco dei corsi del piano di studio, seleziona il corso di \textit{Informatica giuridica} e chiede di visualizzarne il materiale didattico. Il sistema mostra: 
	\begin{tabbing}
		%La prima riga non viene stampata, serve solo per la spaziatura
		\hspace{1cm}-----------------File---------------------------\kill
		%Scrivere da qui
		\hspace{1cm} • Interventi in aula.pdf  \\
		\hspace{1cm} • Link utili.pdf  \\
		\hspace{1cm} • Programma di dettaglio.pdf  \\	
		\hspace{1cm} • Slides.pdf  \\
		\\
	Lo studente Giuseppe visualizza l’elenco dei file anche accedendo alla sezione \\\textit{Materiale didattico} dalla quale gestisce anche i file scaricati.
	\end{tabbing} 
	
	\item \textit{Nessun file:}
	Lo studente Giuseppe, iscritto al terzo anno della facoltà di Informatica, dopo aver visualizzato l'elenco dei corsi del piano di studio, seleziona il corso di Evoluzione del calcolo automatico  e chiede di visualizzarne il materiale didattico. Il sistema cerca i file di \textit{Evoluzione del calcolo automatico}, ma non trova nessun file, quindi mostra il messaggio “Nessun file disponibile.”.
\end{itemize}

\paragraph{Ricerca file}
\begin{itemize}
	\item \textit{Ricerca avvenuta con successo:}
	Lo studente Giuseppe, iscritto al terzo anno della  facoltà di Informatica, dopo aver visualizzato l’elenco del materiale didattico del corso di Informatica giuridica, inserisce la parola chiave \textbf{“Slides”}. Il sistema trova il file e mostra i risultati:   
	\begin{tabbing}
		%La prima riga non viene stampata, serve solo per la spaziatura
		\hspace{1cm}-----------------File---------------------------\kill
		%Scrivere da qui
		\hspace{1cm} • Slides.pdf  \\
	\end{tabbing} 
	
	\item \textit{La ricerca non restituisce risultati:}
	Lo studente Giuseppe, iscritto al terzo anno della facoltà di Informatica, dopo aver visualizzato l’elenco del materiale didattico del corso di Informatica giuridica, inserisce la parola chiave “Sliders”. Il sistema non restituisce alcun riscontro e mostra allo studente il messaggio di errore “Nessun file trovato.”.
\end{itemize}

\paragraph{Visualizza dettagli file}
\begin{itemize}
	\item \textit{Visualizzazione dettagli file avvenuta con successo:}
	Lo studente Giuseppe, iscritto al terzo anno della facoltà di Informatica, dopo aver visualizzato l’elenco del materiale didattico del corso di \textit{Informatica giuridica}, seleziona il file \textit{Slides.pdf}. Il sistema elabora la richiesta e mostra allo studente i seguenti dettagli relativi al file selezionato:
	\begin{tabbing}
		%La prima riga non viene stampata, serve solo per la spaziatura
		\hspace{1cm}-----------------Info---------------------------\= infoRegistrate\kill
		%Scrivere da qui
		\hspace{1cm} • \textbf{Nome file} \> Slides.pdf  \\
		\hspace{1cm} • \textbf{Docente/i} \> Troncarelli Barbara  \\
		\hspace{1cm} • \textbf{Data di caricamento} \> 10/02/2019  \\
		\hspace{1cm} • \textbf{Note} \> In allegato il materiale didattico...  \\
	\end{tabbing} 
\end{itemize}

\paragraph{Apri file}
\begin{itemize}
	\item \textit{Apertura del file avvenuta con successo:}
	Lo studente Giuseppe, iscritto al terzo anno della facoltà di Informatica, dopo aver visualizzato i dettagli relativi al file \textit{Slides.pdf}, sceglie di aprirlo dalla schermata che ne mostra i dettagli. Il sistema cerca il file all'interno dello \textit{storage}, mostrando il suo contenuto allo studente.
	
	\item \textit{File non presente nello \textbf{storage}:}
	Lo studente Giuseppe, iscritto al terzo anno della facoltà di Informatica, dopo aver visualizzato i dettagli relativi al file \textit{Slides.pdf}, sceglie di aprirlo dalla schermata che ne mostra i dettagli. Il sistema cerca il file all'interno dello storage e, non trovandolo, mostra il messaggio “Il file non è presente sul dispositivo. Vuoi scaricarlo ora?”. Lo studente conferma l'operazione, il sistema scarica il file e successivamente lo apre.
\end{itemize}

\paragraph{Rimuovi file}
\begin{itemize}
	\item \textit{Rimozione del file avvenuta con successo:}
	Lo studente Giuseppe, iscritto al terzo anno della facoltà di Informatica, dopo aver visualizzato i dettagli relativi al file \textit{Slides.pdf}, chiede al sistema di rimuoverlo. Il sistema cerca il file all'interno dello \textit{storage} e lo rimuove, successivamente mostra il messaggio “File rimosso con successo.”.
	
	\item \textit{File non presente nello \textbf{storage}:}
	Lo studente Giuseppe, iscritto al terzo anno della facoltà di Informatica, dopo aver visualizzato i dettagli relativi al file \textit{Slides.pdf}, chiede al sistema di rimuoverlo. Il sistema cerca il file all'interno delloe e, non trovandolo, mostra il messaggio di errore “Impossibile rimuovere il file.”.
\end{itemize}
