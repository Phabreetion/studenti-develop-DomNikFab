\subsection{Funzionalità Visualizza dettagli corso}
\begin{itemize}
	\item \textit{Visualizzazione dettagli di un corso superato:}
	Lo studente Giuseppe, iscritto al terzo anno della facoltà di Informatica, sceglie di visualizzare i dettagli del corso superato di Informatica giuridica. Il sistema mostra allo studente i seguenti dettagli:
	\begin{tabbing}
		%La prima riga non viene stampata, serve solo per la spaziatura
		\hspace{1cm}-----------------info1--------------------------- \= --inforegistrata1--- \= --info2--\=--inofregistarta2 \kill
		%Scrivere da qui
		\hspace{1cm} • \textbf{Descrizione} Informatica giuridica \> \textbf{Docente} Troncarelli Barbara\\
		\hspace{1cm} • \textbf{Anno} 1 \> \textbf{CFU} 6   \\
		\hspace{1cm} • \textbf{Data esame} 19/06/2018 \> \textbf{Voto} 25 \\
	\end{tabbing}

	\item \textit{Visualizzazione dettagli di un corso non superato:}
	Lo studente Giuseppe, iscritto al terzo anno della facoltà di Informatica, sceglie di visualizzare i dettagli del corso non superato di Ingegneria del software. Il sistema mostra allo studente i seguenti dettagli:
	\begin{tabbing}
		%La prima riga non viene stampata, serve solo per la spaziatura
		\hspace{1cm}-----------------info1--------------------------- \= --inforegistrata1--- \= --info2--\=--inofregistarta2 \kill
		%Scrivere da qui
		\hspace{1cm} • \textbf{Descrizione} Matematica \> \textbf{Docente} Giovanni  Capobianco\\
		\hspace{1cm} • \textbf{Anno} 1 \> \textbf{CFU} 12  \\
	\end{tabbing}
\end{itemize}