\subsection{Funzionalità chat}
\subsubsection{Chat Studenti}
\paragraph{CUS1 - Visualizza canale}

\begin{itemize}
	
	\item \textit{Scenario 1:\\}
	\textit{Tonino}, Studente di Informatica, vuole visualizzare il canale di discussione relativo ad uno specifico corso. Apre l’app Studenti Unimol e dal \textit{menù} seleziona la voce \textit{“chat”}. A questo punto visualizza la lista delle \textit{chat}, ne apre una e gli viene mostrato il canale di discussione di \textit{default.\\}
	
	\item \textit{Scenario 2 - più canali di discussione presenti nella chat:\\}
	\textit{Tonino}, Studente di Informatica, vuole visualizzare il canale di discussione relativo ad uno specifico corso. Apre l’app \textit{Studenti Unimol} e dal \textit{menù} seleziona la voce \textit{“chat”}. A questo punto visualizza la lista delle \textit{chat}, ne apre una e gli viene mostrato il canale di discussione di default.Tonino seleziona un altro canale di discussione tramite l’apposito pulsante.\\
	
	\item \textit{Scenario 3 - Nessuna risposta dal server:\\}
	\textit{Tonino}, \textit{Studente} di Informatica, vuole visualizzare il canale di discussione relativo ad uno specifico corso. Apre l’app Studenti Unimol e a seguito di una delle seguenti azioni:\\
	1. Seleziona la voce \textit{“chat”} dell’app Studenti Unimol;\\
	3. Seleziona la chat desiderata;\\
	5. Seleziona un canale di discussione alternativo;\\
	il sistema riscontra problemi nel gestire la richiesta, pertanto mostra un messaggio di errore.\\
	
	\item \textit{Scenario 4 - Connessione assente:\\}
	\textit{Tonino}, \textit{Studente} di Informatica, vuole visualizzare il canale di discussione relativo ad uno specifico corso. Apre l’app Studenti Unimol e a seguito di una delle seguenti azioni:\\
	1. Seleziona la  voce \textit{“chat”} dell’app Studenti Unimol;\\
	3. Seleziona la chat desiderata;\\
	5. Seleziona un canale di discussione alternativo;\\
	Tonino riscontra problemi di connessione pertanto il sistema mostra l’ultima copia presente in locale.\\
	
	\item \textit{Scenario 5 - Copia non presente:\\}
	\textit{Tonino}, \textit{Studente} di Informatica, vuole visualizzare il canale di discussione relativo ad uno specifico corso. Apre l’app Studenti Unimol e a seguito di una delle seguenti azioni:
	1. Seleziona la voce \textit{“chat”} dell’app Studenti Unimol;\\
	3. Seleziona la chat desiderata;\\
	5. Seleziona un canale di discussione alternativo;\\
	Tonino riscontra problemi di connessione, il sistema non avendo una copia presente in locale mostra un messaggio di errore.\\
\end{itemize}


\paragraph{CUS2 - Invio messaggio\\}
\begin{itemize}
	
	\item \textit{Scenario 1:\\}
	\textit{Tonino}, {Studente} di Informatica, vuole chiedere ad altri studenti alcune informazioni relative alle lezioni della settimana seguente. Nella sezione \textit{chat} sceglie il canale di discussione relativo al suo anno di corso e digita nella casella di testo il messaggio che invia.\\
	
	\item \textit{Scenario 2 - Connessione assente:\\}
	\textit{Tonino}, \textit{Studente} di Informatica, vuole chiedere ad altri studenti alcune informazioni relative alle lezioni della settimana seguente. Nella sezione \textit{chat} sceglie il canale di discussione relativo al suo anno di corso e digita nella casella di testo il messaggio che invia. Non avendo connessione, il messaggio viene messo in coda ed inviato successivamente quando sarà disponibile la connessione.\\
	
	\item \textit{Scenario 3 - Nessuna risposta dal server:\\}
	\textit{Tonino}, \textit{Studente} di Informatica, vuole chiedere ad altri studenti alcune informazioni relative alle lezioni della settimana seguente. Nella sezione \textit{chat} sceglie il canale di discussione relativo al suo anno di corso e digita nella casella di testo il messaggio che invia. Il sistema riscontra problemi nel gestire la richiesta, pertanto mostra un messaggio di errore.\\
\end{itemize}


\paragraph{CUS3 - Invio allegato\\}
\begin{itemize}
	
	\item \textit{Scenario 1:\\}
	\textit{Tonino},\textit{Studente} di Informatica, vuole condividere il RAD di esempio fornito dal Prof. Fausto Fasano con gli altri studenti del corso, pertanto seleziona il pulsante per scegliere l’allegato e seleziona il RAD dall’elenco dei file presenti sul dispositivo. Il sistema reputa idoneo il \textit{file} selezionato e lo invia.\\
	
	\item \textit{Scenario 2 - Nessuna risposta dal server:\\}
	\textit{Tonino},\textit{Studente} di Informatica, vuole condividere il RAD di esempio fornito dal Prof. Fausto Fasano con gli altri studenti del corso, pertanto seleziona il pulsante per scegliere l’allegato e seleziona il RAD dall’elenco dei file presenti sul dispositivo. Il sistema però e a seguito di una delle seguenti azioni;\\
	2. Mostra l’elenco dei \textit{file} presenti sul dispositivo dell’utente;\\
	4. Controlla se l’allegato è idoneo all’invio nel canale di comunicazione e mostra un messaggio di conferma;
	riscontra problemi nel gestire la richiesta pertanto visualizza un messaggio di errore.\\
	
	\item \textit{Scenario 3 - Il file non é idoneo:\\}
	\textit{Tonino},\textit{Studente} di Informatica, vuole condividere il RAD di esempio fornito dal Prof. Fausto Fasano con gli altri studenti del corso, pertanto seleziona il pulsante per scegliere l’allegato e seleziona il RAD dall’elenco dei \textit{file} presenti sul dispositivo. Il sistema non reputa idoneo il \textit{file} selezionato pertanto annulla l’invio e visualizza un messaggio di errore.\\
	
	\item \textit{Scenario 4 - Connessione assente:\\}
	\textit{Tonino}, \textit{Studente} di Informatica, vuole condividere il RAD di esempio fornito dal Prof. Fausto Fasano con gli altri studenti del corso, pertanto seleziona il pulsante per scegliere l’allegato e seleziona il RAD dall’elenco dei file presenti sul dispositivo. \textit{Tonino} però riscontra problemi con la connessione pertanto l’allegato viene messo in coda ed inviato successivamente quando sarà disponibile la connessione.
\end{itemize}

\paragraph{CUS4 - Rispondi a singolo messaggio\\}
\begin{itemize}
	\item \textit{Scenario 1:\\}
	\textit{Tonino}, Studente di Informatica, desidera rispondere al messaggio di un membro del canale che sta visualizzando per chiedere ulteriori informazioni riguardo un argomento, seleziona quindi il messaggio e sceglie l’opzione di risposta al messaggio pertanto il sistema visualizza il messaggio evidenziato.
\end{itemize}

\paragraph{CUS5 - Scarica  allegato\\}
\begin{itemize}
	\item \textit{Scenario 1:\\}
	\textit{Tonino}, \textit{Studente} frequentante il primo anno di Informatica desidera scaricare un’immagine dal canale di comunicazione dedicato a matematica. Accede alla funzionalità di download ed il sistema scarica il \textit{file} richiesto sul dispositivo di \textit{Tonino}.\\
	
	\item \textit{Scenario 2 - Connessione assente:\\}
	\textit{Tonino}, \textit{Studente} frequentante il primo anno di Informatica desidera scaricare un’immagine dal canale di comunicazione dedicato a matematica. \textit{Tonino} seleziona l’opzione per scaricare l’immagine ma riscontra problemi con la connessione pertanto il sistema impedisce il download dell’immagine e visualizza un messaggio di errore.\\
	
	\item \textit{Scenario 3 - Lo studente nega il download dell'allegato:\\}
	\textit{Tonino}, \textit{Studente} frequentante il primo anno di Informatica desidera scaricare un’immagine dal canale di comunicazione dedicato a matematica. \textit{Tonino} seleziona l’opzione per scaricare l’immagine ma avendo selezionato un’immagine sbagliata annulla il download pertanto il \textit{file} non viene salvato sul dispositivo.\\
\end{itemize}

\paragraph{CUS6 - Segnalazione messaggio:\\}
\begin{itemize}
	\item \textit{Scenario 1:\\}
	\textit{Tonino}, dopo aver visualizzato la \textit{chat}, seleziona con un tap il messaggio che intende segnalare, poichè ritiene che il contenuto sia moralmente inadatto.\textit{Tonino} accede alla sezione \textit{menú} del canale di discussione e seleziona l'opzione \textit{"segnala"}.\textit{Tonino} conferma l'invio della segnalazione e il sistema visualizza un messaggio di conferma.\\
	
	\item \textit{Scenario 2 - Connessione assente:\\}
	\textit{Tonino}, dopo aver visualizzato la \textit{chat}, seleziona con un tap il messaggio che intende segnalare, poichè ritiene che il contenuto sia moralmente inadatto.\textit{Tonino} accede alla sezione \textit{menú} del canale di discussione e seleziona l'opzione \textit{"segnala"}. il sistema mostra un messaggio che comunica a \textit{Tonino} che il suo dispositivo non risulta essere connesso ad una rete internet e non consente la segnalazione.\\
	
	\item \textit{Scenario 3 - Nessuna risposta dal server:\\}
	\textit{Tonino}, dopo aver visualizzato la \textit{chat}, seleziona con un tap il messaggio che intende segnalare, poichè ritiene che il contenuto sia moralmente inadatto.\textit{Tonino} accede alla sezione \textit{menú} del canale di discussione e seleziona l'opzione \textit{"segnala"}. il sistema mostra un messaggio di errore a \textit{Tonino}.
\end{itemize}

\paragraph{CUS7 - Ricerca testo nella chat\\}
\begin{itemize}
	\item \textit{Scenario 1:\\}
	\textit{Tonino}, \textit{Studente} di Informatica, ha intenzione di cercare un vecchio messaggio. Accede alla sezione \textit{menù} del canale di discussione e seleziona il pulsante di ricerca e digita il testo da cercare nella casella di testo. 
	Può, quindi, visualizzare tutti i messaggi che contengono il testo cercato e scorrerli finchè non trova quello desiderato.\\
	
	\item \textit{Scenario 2 - Nessuna risposta dal server:\\}
	\textit{Tonino}, \textit{Studente} di Informatica, ha intenzione di cercare un vecchio messaggio e a seguito di una delle seguenti azioni:\\
	2. Accede alla sezione \textit{“menù”} del canale di discussione;\\
	4. Seleziona il pulsante di ricerca;\\
	6. Digita il testo da cercare;\\
	il sistema riscontra problemi nel gestire la richiesta, pertanto mostra un messaggio di errore.\\
	
	\item \textit{Scenario 3 - Il testo cercato non é presente:\\}
	\textit{Tonino}, \textit{Studente} di Informatica, ha intenzione di cercare un vecchio messaggio. Accede alla sezione \textit{menù} del canale di discussione e seleziona il pulsante di ricerca e digita il testo da cercare nella casella di testo.Non essendo presente il testo cercato nei vecchi messaggi, l’utente visualizza un avviso che lo informa che la ricerca non ha portato risultati.\\
	
\end{itemize}


\paragraph{CUS8 - Tag membro in messaggio\\}
\begin{itemize}
	\item \textit{Scenario 1:\\}
	\textit{Tonino},ha intenzione di inviare un messaggio richiamando l’attenzione di una determinata persona, per fare questo, utilizza la chiocciola (@) e poi scrive il nome della persona, a \textit{Tonino} appare una lista di nomi da cui può selezionare quello desiderato che viene visualizzato nella casella di testo.\\
	
	\item \textit{Scenario 2 - Lo studente digita il nome errato:\\}
	\textit{Tonino},ha intenzione di inviare un messaggio richiamando l’attenzione di una determinata persona, per fare questo, utilizza la chiocciola (@) e poi scrive il nome della persona, a \textit{Tonino} non appare la lista dei nomi poiché ha digitato un nome sbagliato o che non è presente in quel determinato canale di discussione.\\
	
\end{itemize}


\paragraph{CUS9 - Gestisci notifiche chat\\}
\begin{itemize}
	\item \textit{Scenario 1:\\}
	\textit{Tonino}, \textit{Studente} di Informatica, si trova in un canale di discussione e vuole disattivare lo stato delle notifiche del canale per non ricevere più avvisi, quindi accede alla sezione \textit{“menú”}, e seleziona la voce per disattivare le notifiche del canale.\\
	
	\item \textit{Scenario 2 - Connessione assente:\\}
	\textit{Tonino}, \textit{Studente} di Informatica, si trova in un canale di discussione e vuole disattivare lo stato delle notifiche del canale per non ricevere più avvisi, quindi accede alla sezione menú, e seleziona la voce per disattivare le notifiche del canale. Purtroppo però \textit{Tonino} ha problemi con la connessione pertanto l’operazione viene annullata e viene mostrato un messaggio di errore.\\
	
	\item \textit{Scenario 3 - Nessuna risposta del server:\\}
	\textit{Tonino}, \textit{Studente} di Informatica, si trova in un canale di discussione e vuole disattivare lo stato delle notifiche del canale per non ricevere più avvisi, quindi accede alla sezione menú, e seleziona la voce per disattivare le notifiche del canale. Purtroppo però il sistema riscontra dei problemi nell’eseguire la richiesta pertanto annulla l’operazione e viene mostrato un messaggio di errore.\\
	
\end{itemize}


\paragraph{CUS10 - Selezione emoji\\}

\textit{Scenario 1:\\}
\textit{Tonino}, che ha aperto la chat di un canale di discussione, clicca sull’icona \textit{emoji} ed il sistema mostra una finestra con varie emoji che può scegliere, seleziona quelle che desidera e vengono mostrate nella casella di testo.

\paragraph{CUS11 - Visualizza elenco membri chat\\}
\begin{itemize}
	\item \textit{Scenario 1:\\}
	\textit{Tonino}, \textit{Studente} di Informatica, si trova in un canale di discussione e vuole visualizzarne i membri. Seleziona il nome del canale e visualizza l’elenco dei partecipanti e il loro numero.\\
	
	\item \textit{Scenario 2 - Connessione assente:\\}
	\textit{Tonino}, \textit{Studente} di Informatica, si trova in un canale di discussione e vuole visualizzarne i membri. Seleziona il nome del canale ma riscontra problemi con la connessione pertanto il sistema mostra l’ultima copia presente in locale.\\
	
	\item \textit{Scenario 3 - Nessuna risposta dal server:\\}
	
	\textit{Tonino}, \textit{Studente} di Informatica, si trova in un canale di discussione e vuole visualizzarne i membri. Seleziona il nome del canale ma il sistema riscontra problemi nel gestire la richiesta, pertanto mostra un messaggio di errore.\\
	
	\item \textit{Scenario 4 - Copia non presente:\\}
	\textit{Tonino}, \textit{Studente} di Informatica, si trova in un canale di discussione e vuole visualizzarne i membri. Seleziona il nome del canale ma riscontra problemi di connessione, il sistema non avendo una copia presente in locale mostra un messaggio di errore.\\
\end{itemize}

\subsubsection{Chat Docenti}
\paragraph{CUD1 - Creazione canale\\}
\begin{itemize}
	\item \textit{Scenario 1:\\}
	\textit{Giacomo}, \textit{Docente} dell’\textit{Università degli Studi del Molise}, si trova nella \textit{chat} di Matematica di cui è \textit{Amministratore} e vuole creare un canale secondario per differenziare le comunicazioni tra I modulo e II modulo, pertanto  accede alla sezione \textit{“menú”} e seleziona la voce per la creazione di un nuovo canale. \textit{Giacomo} inserisce come nome del canale  “Matematica II” e, dopo aver confermato il nome inserisce gli identificativi degli  studenti che seguono il suo corso e successivamente conferma i dati inseriti.\\
	
	\item \textit{Scenario 2 - Connessione assente:\\}
	\textit{Giacomo}, \textit{Docente} dell’\textit{Università degli Studi del Molise}, si trova nella \textit{chat} di Matematica di cui è \textit{Amministratore} e vuole creare un canale secondario per differenziare le comunicazioni tra I modulo e II modulo,  pertanto  accede alla sezione \textit{“menú”} e seleziona la voce per la creazione di un nuovo canale. Purtroppo \textit{Giacomo} ha problemi con la sua connessione ad \textit{internet}, pertanto visualizza il messaggio di errore che gli comunica l’impossibilità di andare avanti con la creazione del canale.\\
	
	\item \textit{Scenario 3 - Nessuna risposta dal server:\\}
	\textit{Giacomo}, \textit{Docente} dell’\textit{Università degli Studi del Molise}, si trova nella \textit{chat} di “Matematica” di cui è \textit{Amministratore} e vuole creare un canale secondario per differenziare le comunicazioni tra I modulo e II modulo pertanto accede alla sezione \textit{“menú”} e a seguito di una delle seguenti azioni :\\
	3. Seleziona la voce di creazione del nuovo canale;\\
	5. Inserisce il nome del nuovo canale;\\
	7. Conferma del nome inserito;\\
	9. Inserisce l’identificativo dei membri da aggiungere al canale;\\
	11. Conferma gli identificativi inseriti;\\
	il sistema riscontra problemi nel gestire la richiesta, pertanto mostra un messaggio di errore.\\
\end{itemize}

\paragraph{CUD2 - Cancellazione canale\\}
\begin{itemize}
	\item \textit{Scenario 1:\\}
	\textit{Giacomo}, \textit{Docente} dell’\textit{Università degli Studi del Molise}, si trova nella \textit{chat} di Matematica di cui è \textit{Amministratore} e vuole cancellare il canale secondario  “Matematica II” che sta visualizzando, quindi accede alla sezione \textit{“menú”} e seleziona la voce per la cancellazione del canale. Confermata la scelta, il canale viene cancellato dalla \textit{chat}.\\
	
	\item \textit{Scenario 2 - Connessione assente:\\}
	\textit{Giacomo}, \textit{Docente} dell’\textit{Università degli Studi del Molise}, si trova nella \textit{chat} di Matematica di cui è \textit{Amministratore} e vuole cancellare il canale secondario  “Matematica II” che sta visualizzando, quindi accede alla sezione \textit{“menú”}  e seleziona la voce per la cancellazione del canale. \textit{Giacomo} riscontra problemi con la sua connessione pertanto l’operazione viene annullata.\\
	
	\item \textit{Scenario 3 - È presente un solo canale di discussione:\\}
	\textit{Giacomo}, \textit{Docente} dell’\textit{Università degli Studi del Molise}, si trova nella \textit{chat} di “Matematica” di cui è \textit{Amministratore} e vuole cancellare il canale “Matematica” che sta visualizzando, quindi accede alla sezione \textit{“menú”} e seleziona la voce per la cancellazione del canale. \textit{Giacomo} successivamente conferma la sua scelta ma il sistema annulla l’operazione perchè il canale da cancellare è l’unico presente nella chat e mostra un messaggio di errore.\\
	
	\item \textit{Scenario 4 - Nessuna risposta dal server:\\}
	\textit{Giacomo}, \textit{Docente} dell’\textit{Università degli Studi del Molise}, si trova nella chat di “Matematica” di cui è amministratore e vuole cancellare il canale secondario  “Matematica II” che sta visualizzando, quindi accede alla sezione \textit{“menú”} e seleziona la voce per la cancellazione del canale. Il sistema ha problemi nel gestire la richiesta, quindi l’operazione viene annullata e viene mostrato un messaggio di errore.\\
\end{itemize}

\paragraph{CUD3 - Aggiungi membro ad un canale\\}
\begin{itemize}
	\item \textit{Scenario 1:\\}
	\textit{Giacomo}, \textit{Docente} dell’\textit{Università degli Studi del Molise}, vuole aggiungere un nuovo \textit{Studente} appena iscritto al corso, al canale “Matematica II” che sta visualizzando, quindi accede alla sezione \textit{“menù”} e seleziona la voce per aggiungere  nuovo membro al canale. \textit{Giacomo} inserisce l’identificativo del membro da aggiungere al canale e dopo aver confermato il dato il sistema aggiunge il membro al canale.\\
	
	\item \textit{Scenario 2 - Connessione assente:\\}
	\textit{Giacomo}, \textit{Docente} dell’\textit{Università degli Studi del Molise}, vuole aggiungere un nuovo \textit{Studente} appena iscritto al corso, al canale “Matematica II” che sta visualizzando, quindi accede alla sezione \textit{“menú”} e seleziona la voce per aggiungere  nuovo membro al canale. \textit{Giacomo} riscontra problemi con la connessione ad internet, quindi il sistema mostra un messaggio di errore e annulla l’operazione.\\
	
	\item \textit{Scenario 3 - Nessuna risposta dal server:\\}
	\textit{Giacomo}, \textit{Docente} dell’\textit{Università degli Studi del Molise}, vuole aggiungere un nuovo \textit{Studente} appena iscritto al corso, al canale “Matematica II” che sta visualizzando, quindi accede alla sezione \textit{“menú”} e a seguito di una delle seguenti azioni :\\
	3. Seleziona la voce di aggiunta nuovo membro al canale;\\
	5. Inserisce l’identificativo del membro da aggiungere;\\
	7. Conferma i dati inseriti;\\
	il sistema riscontra problemi nel gestire la richiesta, pertanto mostra un messaggio di errore e annulla l’operazione.
\end{itemize}

\paragraph{CUD4 - Rimuovi membro da un canale:\\}
\begin{itemize}
	\item \textit{Scenario 1:\\}
	\textit{Giacomo}, \textit{Docente} dell’\textit{Università degli Studi del Molise}, vuole rimuovere uno \textit{Studente} iscritto al corso, al canale “Matematica II” che sta visualizzando, quindi accede alla sezione \textit{“menú”} e seleziona la voce per rimuovere un membro dal canale. \textit{Giacomo} inserisce l’identificativo del membro da rimuovere e dopo aver confermato il dato il sistema rimuove il membro dal canale.\\
	
	\item \textit{Scenario 2 - Connessione assente:\\}
	\textit{Giacomo}, \textit{Docente} dell’\textit{Università degli Studi del Molise},vuole rimuovere uno \textit{Studente} iscritto al corso, al canale “Matematica II” che sta visualizzando, quindi accede alla sezione \textit{“menú”} e seleziona la voce per rimuovere un membro dal canale.
	\textit{Giacomo} riscontra problemi con la connessione ad internet quindi il sistema mostra un messaggio di errore e annulla l’operazione.\\
	
	\item \textit{Scenario 3 - Nessuna risposta dal server:\\}
	\textit{Giacomo}, \textit{Docente} dell’\textit{Università degli Studi del Molise}, vuole rimuovere uno \textit{Studente} iscritto al corso, al canale “Matematica II” che sta visualizzando, quindi accede alla sezione \textit{“menú”} e a seguito di una delle seguenti azioni :\\
	3. Seleziona la voce di rimozione membro dal canale;\\
	5. Inserisce l’identificativo del membro da rimuovere;\\
	7. Conferma i dati inseriti;\\
	il sistema riscontra problemi nel gestire la richiesta, pertanto mostra un messaggio di errore e annulla l’operazione.\\
\end{itemize}

\paragraph{CUD5 - Blocca studente:\\}
\begin{itemize}
	\item \textit{Scenario 1:\\}
	\textit{Giacomo}, professore di matematica ha ritenuto inappropriati i contenuti dei messaggi inviati da \textit{Tonino} nel canale di comunicazione della sua materia. Così \textit{Giacomo} seleziona un messaggio inviato da \textit{Tonino}, il sistema mostra l’opzione di silenziare \textit{Tonino}, \textit{Giacomo} conferma l’opzione di blocco e il sistema impedisce a \textit{Tonino} di inviare i messaggi in quel canale di comunicazione.\\
	
	\item \textit{Scenario 2 - Il \textit{Docente} non conferma il blocco dello \textit{Studente}:\\}
	\textit{Giacomo}, professore di matematica ha ritenuto inappropriati i contenuti dei messaggi inviati da \textit{Tonino} nel canale di comunicazione della sua materia. Quindi seleziona dalla lista contatti della \textit{chat} \textit{Tonino}, e seleziona l’opzione per bloccare lo studente, \textit{Giacomo} capisce però di aver frainteso le parole di \textit{Tonino} e quindi non conferma l’opzione di blocco e l’operazione di blocco viene annullata, pertanto \textit{Tonino} ha la possibilità di continuare a inviare messaggi.\\
	
	\item \textit{Scenario 3 - La connessione è assente:\\}
	\textit{Giacomo}, professore di matematica ha ritenuto inappropriati i contenuti dei messaggi inviati da \textit{Tonino} nel canale di comunicazione della sua materia. Così \textit{Giacomo} seleziona un messaggio inviato da \textit{Tonino}, il sistema mostra l’opzione di silenziare  lo studente, \textit{Giacomo} conferma l’opzione di blocco, ma riscontra problemi con la connessione pertanto viene visualizzato un messaggio di errore.\\
	
	\item \textit{Scenario 4 - Nessuna risposta dal server:\\}
	\textit{Giacomo}, professore di matematica ha ritenuto inappropriati i contenuti dei messaggi inviati da \textit{Tonino} nel canale di comunicazione della sua materia. Così \textit{Giacomo} seleziona un messaggio inviato da \textit{Tonino}, il sistema mostra l’opzione per silenziare \textit{Tonino}, e conferma l’opzione di blocco, ma il sistema riscontra problemi nel gestire la richiesta pertanto viene visualizzato un messaggio di errore.\\
\end{itemize}

\paragraph{CUD6 - Sblocca studente\\}
\begin{itemize}
	\item \textit{Scenario 1:\\}
	\textit{Giacomo}, professore di matematica desidera sbloccare lo studente \textit{Tonino} nel canale di comunicazione della sua materia. Così \textit{Giacomo} seleziona un messaggio inviato da \textit{Tonino}, il sistema mostra l’opzione di sbloccare lo studente e conferma l’opzione di sblocco e permette a \textit{Tonino} di inviare i messaggi in quel canale di comunicazione.\\
	
	\item \textit{Scenario 2 - Il Docente non conferma lo sblocco dello Studente:\\}
	\textit{Giacomo}, professore di matematica desidera sbloccare \textit{Tonino} nel canale di comunicazione della sua materia. Quindi seleziona dalla lista contatti della \textit{chat} \textit{Tonino}, e seleziona l’opzione per sbloccare lo studente, \textit{Giacomo} decide però di annullare l’operazione, quindi non conferma l’opzione di sblocco e l’operazione viene annullata, pertanto \textit{Tonino} non ha la possibilità di inviare messaggi nel canale.\\
	
	\item \textit{Scenario 3 - La connessione è assente:\\}
	\textit{Giacomo}, professore di matematica desidera sbloccare \textit{Tonino} nel canale di comunicazione della sua materia. Così \textit{Giacomo} seleziona un messaggio inviato da \textit{Tonino}, il sistema mostra l’opzione per sbloccare \textit{Tonino} e conferma l’opzione di sblocco ma riscontra problemi con la connessione pertanto viene visualizzato un messaggio di errore.\\
	
	\item \textit{Scenario 4 - Nessuna risposta dal server:\\}
	\textit{Giacomo}, professore di matematica desidera sbloccare \textit{Tonino} nel canale di comunicazione della sua materia. Così \textit{Giacomo} seleziona un messaggio inviato da \textit{Tonino}, il sistema mostra l’opzione per silenziare \textit{Tonino} e conferma l’opzione di blocco, ma il sistema riscontra problemi nel gestire la richiesta, pertanto viene visualizzato un messaggio di errore.\\
\end{itemize}

\subsubsection{Pannello di amministrazione}
\paragraph{CUP1 - Login \\}
\begin{itemize}
	
	\item \textit{Scenario 1:\\}
	\textit{Pasquale}, in qualità di amministratore, visualizza la schermata di \textit{Login}. Seleziona la voce \textit{Login} e visualizza la schermata per l’inserimento dei dati. \textit{Pasquale}, a tal fine inserisce la propria \textit{username}: L.pasquale, e la relativa \textit{password}: Pasquale123. In seguito, sottomessa la richiesta ed effettuato il \textit{login} visualizza la schermata principale.\\
	
	\item \textit{Scenario 2 - Connessione assente:\\}
	\textit{Pasquale}, in qualità di amministratore del sistema, visualizza la schermata di \textit{Login}. Seleziona la voce \textit{Login} e visualizza la schermata per l’inserimento dei dati. \textit{Pasquale}, a tal fine inserisce la propria \textit{username}: L.pasquale, e la relativa \textit{password}: Pasquale123. In seguito alla conferma, visualizza un messaggio di errore, il quale avvisa \textit{Pasquale} che non è possibile effettuare l’accesso in quanto non vi è connessione.\\
	
	\item \textit{Scenario 3 - Errore di sistema:\\}
	\textit{Pasquale}, in qualità di amministratore del sistema, visualizza la schermata di \textit{Login}. Seleziona la voce \textit{Login} e visualizza la schermata per l’inserimento dei dati. \textit{Pasquale}, a tal fine inserisce la propria \textit{username}: L.pasquale, e la relativa \textit{password}: Pasquale123. In seguito alla conferma, visualizza un messaggio di errore, il quale avvisa \textit{Pasquale} che non è possibile effettuare l’accesso in quanto il sistema non risponde.\\
	
	\item \textit{Scenario 4 - Uno o entrambi i campi sono vuoti:\\}
	\textit{Pasquale}, in qualità di amministratore del sistema, visualizza la schermata di \textit{Login}. Seleziona la voce \textit{Login} e visualizza la schermata per l’inserimento dei dati. \textit{Pasquale}, a tal fine procede con l’inserimento delle credenziali:\\
	1. \textit{username}: L.pasquale, dimenticando di inserire la relativa \textit{password};\\
	2. \textit{password}: pasquale123, dimenticando di inserire il proprio \textit{username};\\
	3. procede senza inserire le proprie credenziali.\\
	In seguito, sottomette la richiesta e visualizza un messaggio di errore, il quale lo avvisa che sono presenti dei campi vuoti. Ciò in quanto nel corso dell’inserimento delle credenziali dimentica di compilare uno o entrambi i campi. Di conseguenza, non effettua il \textit{login} e visualizza nuovamente la schermata di \textit{login}.\\
	
	\item \textit{Scenario 5 - Le credenziali inserite non sono valide (una o entrambe):\\}
	\textit{Pasquale}, in qualità di amministratore del sistema, visualizza la schermata di \textit{Login}. Seleziona la voce \textit{Login} e visualizza la schermata per l’inserimento dei dati. \textit{Pasquale}, a tal fine inserisce le proprie credenziali:\\
	1. \textit{username}: Pasquale, \textit{password}: Pasquale123;\\
	2. \textit{username}: L.pasquale, \textit{password}: pasquale12;\\
	3. \textit{username}: L.Pasquale, \textit{password}: pasquale132;\\
	In seguito, sottomette la richiesta e visualizza un messaggio di errore, in quanto le credenziali inserite non risultano corrette. Di conseguenza, non effettua il \textit{login} e visualizza nuovamente la schermata di \textit{login}.\\
\end{itemize}

\paragraph{CUP2 - Logout\\}
\begin{itemize}
	\item \textit{Scenario 1:\\}
	\textit{Pasquale}, in qualità di amministratore, decide di uscire dal sistema, a tal fine visualizza la schermata principale e seleziona la voce \textit{Logout}. In seguito, \textit{Paquale} visualizza la schermata relativa al \textit{Login}.\\
	
	\item \textit{Scenario 2 - Connessione assente:\\}
	\textit{Pasquale}, in qualità di amministratore, decide di uscire dal sistema, a tal fine visualizza la schermata principale e seleziona la voce \textit{Logout}. In seguito, \textit{Paquale} visualizza un messaggio di errore, il quale lo avvisa che non è possibile effettuare il \textit{Logout} in quanto non vi è connessione. \textit{Pasquale} visualizza la schermata principale.\\
	
	\item \textit{Scenario 3 - Chiusura sessione non corretta:\\}
	\textit{Pasquale}, in qualità di amministratore, decide di uscire dal sistema, a tal fine visualizza la schermata principale e seleziona la voce \textit{Logout}. In seguito, \textit{Paquale}, visualizza un messaggio di errore della chiusura del sistema, non è possibile pertanto effettuare il \textit{Logout}.\\
	\textit{Pasquale} visualizza la schermata principale.\\
\end{itemize}

\paragraph{CUP3 - Ricerca chat\\}
\begin{itemize}
	\item \textit{Scenario 1:\\}
	\textit{Pasquale}, in qualità di amministratore del sistema, seleziona come tipologia di \textit{chat} da visualizzare la \textit{chat} dei corsi e visualizza la schermata relativa alla ricerca.
	\textit{Pasquale}, in seguito, ricerca la singola \textit{chat} relativa al corso di Basi di dati inserendo il nome della \textit{chat} all’interno di un’area di testo. Dopo aver effettuato tale ricerca, visualizza la \textit{chat} desiderata.\\
	
	\item \textit{Scenario 2 - Applicazione di un filtro:\\}
	\textit{Pasquale}, in qualità di amministratore del sistema, seleziona come tipologia di \textit{chat} da visualizzare la \textit{chat} dei corsi e visualizza la schermata relativa alla ricerca. Nel momento in cui vuole ricercare le \textit{chat} dei corsi del secondo anno di Informatica,
	applica i relativi filtri. Seleziona in primis il filtro “Dipartimento”, e visualizzata la lista dei dipartimenti, seleziona il Dipartimento Bioscienze e Territorio. A quel punto, decide di applicare il secondo filtro relativo ai corsi di laurea e visualizza questi ultimi.
	Selezionato il corso di laurea di Informatica, procede con il selezionare il filtro inerente all’anno di corso. \textit{Pasquale}, visualizza la coorte degli anni che individuano un anno accademico di un corso di laurea, seleziona la coorte e procede con la ricerca.
	Dopo aver effettuato tale ricerca, visualizza l’elenco delle \textit{chat} relative ai filtri applicati.\\
	
	\item \textit{Scenario 3 - Connessione assente:\\}
	\textit{Pasquale}, amministratore del sistema, seleziona la tipologia di \textit{chat} relativa ai corsi e vuole effettuare una ricerca al suo interno. 
	Il dispositivo di \textit{Pasquale} però, non ha in quel momento una connessione ad internet disponibile, di conseguenza, il sistema riscontra problemi nel gestire tale richiesta, pertanto \textit{Pasquale} visualizza un messaggio di errore.\\
	
	\item \textit{Scenario 4 - Errore di sistema:\\}
	\textit{Pasquale}, amministratore del sistema, seleziona come tipologia di \textit{chat} da visualizzare la \textit{chat} dei corsi e visualizza la schermata relativa alla ricerca. Nel momento in cui vuole ricercare le \textit{chat} dei corsi del secondo anno di Informatica, applica i relativi filtri. 
	Tuttavia, il sistema negherà la visualizzazione della schermata che consente di effettuare la scelta del filtro selezionato mostrando un messaggio di errore.\\
	
	\item \textit{Scenario 5 - Chat non trovata:\\}
	\textit{Pasquale}, amministratore del sistema, seleziona la tipologia di \textit{chat} relativa ai corsi e visualizza la schermata relativa alla ricerca. \textit{Pasquale}, decide di ricercare una singola \textit{chat}, a tal fine seleziona la voce ricerca e inserisce il relativo nome all’interno di un’area di testo. In seguito all’inserimento del testo e la relativa ricerca, si verifica un errore, in quanto la \textit{chat} ricercata non è presente e di conseguenza visualizza un messaggio il quale lo informa che la ricerca effettuata non ha portato risultati.\\
\end{itemize}

\paragraph{CUP4 - Visualizza lista chat\\}
\begin{itemize}
	\item \textit{Scenario 1:\\}
	\textit{Pasquale}, in qualità di amministratore del sistema, intende visualizzare la tipologia di \textit{chat} relativa agli studenti. Pertanto, effettua il \textit{login}, accede alla schermata principale, e dal \textit{menù} laterale seleziona la voce "\textit{chat} studenti". A quel punto visualizza una schermata per la ricerca, all’interno della quale, compila i campi ed effettua la ricerca della \textit{chat} desiderata. \textit{Pasquale} visualizza una schermata con l’elenco delle \textit{chat} degli studenti, inerenti alla ricerca effettuata.\\
	
	\item \textit{Scenario 2:\\}
	\textit{Pasquale}, in qualità di amministratore del sistema, intende visualizzare la tipologia di \textit{chat} relativa ai corsi. Pertanto, effettua il \textit{login}, accede alla schermata principale, e dal \textit{menù} laterale seleziona la voce "\textit{chat} corsi". A quel punto visualizza una schermata per la ricerca, all’interno della quale, compila i campi ed effettua la ricerca della chat desiderata. Pasquale visualizza una schermata con l’elenco delle \textit{chat} attive,inerenti alla ricerca effettuata.\\
	
	\item \textit{Scenario 3:\\}
	\textit{Pasquale}, in qualità di amministratore del sistema, intende visualizzare la tipologia di \textit{chat} relativa ai corsi. Pertanto, effettua il \textit{login}, accede alla schermata principale, e dal \textit{menù} laterale seleziona la voce "\textit{chat} corsi".  A quel punto visualizza una schermata per la ricerca, all’interno della quale, compila i campi ed effettua la ricerca della \textit{chat} desiderata. \textit{Pasquale} visualizza una schermata con l’elenco delle \textit{chat} non attive,inerenti alla ricerca effettuata.\\
	
	\item \textit{Scenario 4 - Connessione assente:\\}
	\textit{Pasquale}, in qualità di amministratore del sistema, intende visualizzare la tipologia di \textit{chat} relativa agli studenti. Pertanto, effettua il \textit{login}, accede alla schermata principale, e dal \textit{menù} laterale seleziona la voce "\textit{chat} studenti". Il dispositivo di \textit{Pasquale} però, non ha in quel momento una connessione ad internet disponibile, di conseguenza, il sistema non consentirà la selezione della categoria di \textit{chat} e comunicherà a \textit{Pasquale} l’assenza di connessione.\\
	
	\item \textit{Scenario 5 - Errore di sistema:\\}
	\textit{Pasquale}, in qualità di amministratore del sistema, intende visualizzare la tipologia di \textit{chat} relativa agli studenti. Pertanto, effettua il \textit{login}, accede alla schermata principale, e dal \textit{menù} laterale seleziona la voce "\textit{chat} corsi". Tuttavia, il sistema negherà la visualizzazione della schermata che consente di effettuare la ricerca mostrando un messaggio di errore.\\
\end{itemize}

\paragraph{CUP5 - Abilita chat\\}
\begin{itemize}
	\item \textit{Scenario 1:\\}
	\textit{Pasquale}, in qualità di amministratore del sistema, a seguito di una richiesta di abilitazione della \textit{chat} “Basi di dati”,seleziona la categoria \textit{chat} relativa ai corsi all’interno della quale seleziona la voce \textit{chat} non attive e visualizza le \textit{chat} disabilitate . Scelta la \textit{chat} desiderata, seleziona la voce abilitata \textit{chat}. Visualizza, in seguito, un pannello in cui il sistema chiede la conferma per abilitare la \textit{chat}, a tal fine \textit{Pasquale} decide di confermare l’operazione effettuata. \textit{Pasquale}, visualizza la \textit{chat} “Basi di dati” nell’elenco delle \textit{chat} attive.\\
	
	\item \textit{Scenario 2 - Connessione assente:\\}
	\textit{Pasquale}, amministratore del sistema, a seguito di una richiesta di abilitazione della \textit{chat} “Basi di dati”, seleziona la categoria \textit{chat} relativa ai corsi all’interno della quale visualizza le \textit{chat} non abilitate. Tuttavia, dopo aver selezionato la voce abilita \textit{chat}, il sistema riscontra problemi nel gestire tale richiesta a causa dell’assenza di connessione, di conseguenza \textit{Pasquale} visualizza un messaggio di errore.\\
	
	\item \textit{Scenario 3 -  Errore di sistema:\\}
	\textit{Pasquale}, amministratore del sistema, a seguito di una richiesta di abilitazione della \textit{chat} “Basi di dati”, seleziona la categoria \textit{chat} relativa ai corsi all’interno della quale visualizza le \textit{chat} non abilitate.  In seguito alla conferma dell’abilitazione della \textit{chat}, il sistema riscontra problemi nel gestire la richiesta, pertanto mostra un messaggio di errore e l’operazione non viene eseguita.
\end{itemize}

\paragraph{CUP6 - Disabilita chat\\}
\begin{itemize}
	\item \textit{Scenario 1:\\}
	\textit{Pasquale}, in qualità di amministratore del sistema, a seguito di una richiesta di disabilitazione della \textit{chat} “Basi di dati”, seleziona la categoria \textit{chat} relativa ai corsi all’interno della quale visualizza le \textit{chat} abilitate. Scelta la \textit{chat} desiderata, seleziona la voce disabilitata \textit{chat}. Visualizza, in seguito, un pannello in cui il sistema chiede la conferma per disabilitare la \textit{chat}, a tal fine \textit{Pasquale} decide di confermare l’operazione effettuata. Pasquale, visualizza la \textit{chat} “Basi di dati” nell’elenco delle \textit{chat} non attive.\\
	
	\item \textit{Scenario 2 - Connessione assente:\\}
	\textit{Pasquale}, amministratore del sistema, a seguito di una richiesta di disabilitazione della \textit{chat} “Basi di dati”, seleziona la categoria \textit{chat} relativa ai corsi all’interno della quale visualizza le \textit{chat} abilitate. Tuttavia, dopo aver selezionato la voce disabilita \textit{chat}, il sistema riscontra problemi nel gestire tale richiesta a causa dell’assenza di connessione, di conseguenza Pasquale visualizza un messaggio di errore.\\
	
	\item \textit{Scenario 3 - Errore di sistema:\\}
	\textit{Pasquale}, amministratore del sistema,a seguito di una richiesta di disabilitazione della \textit{chat} “Basi di dati”, seleziona la categoria \textit{chat} relativa ai corsi all’interno della quale visualizza le \textit{chat} abilitate. In seguito alla conferma della disabilitazione della \textit{chat}, il sistema riscontra problemi nel gestire la richiesta, pertanto mostra un messaggio di errore e l’operazione non viene eseguita.\\
\end{itemize}

\paragraph{CUP7 - Visualizza canale\\}
\begin{itemize}
	\item \textit{Scenario 1:\\}
	\textit{Pasquale}, in qualità di amministratore del sistema, vuole visualizzare un canale 
	con i relativi messaggi in esso contenuti. Pertanto, seleziona dal \textit{menù} laterale una specifica categoria di \textit{chat} dopodichè, utilizzando l'area di testo, va alla ricerca della \textit{chat} desiderata. Pasquale entra nella \textit{chat} cercata e seleziona uno specifico canale , all’interno del quale visualizza la conversazione di interesse.\\
	
	\item \textit{Scenario 2 - Connessione assente:\\}
	\textit{Pasquale}, in qualità di amministratore del sistema, vuole visualizzare un canale 
	con i relativi messaggi in esso contenuti. Pertanto, seleziona dal menù laterale una specifica categoria di \textit{chat} dopodichè, utilizzando l'area di testo, va alla ricerca della \textit{chat} desiderata. Il dispositivo di \textit{Pasquale} però, non ha in quel momento una connessione ad internet disponibile, di conseguenza, il sistema non consente la selezione della \textit{chat} da visualizzare e comunica a \textit{Pasquale} l’assenza di connessione.\\
	
	\item \textit{Scenario 3 - Errore di sistema:\\}
	\textit{Pasquale}, in qualità di amministratore del sistema, vuole visualizzare un canale 
	con i relativi messaggi in esso contenuti. Pertanto, seleziona dal menù laterale una specifica categoria di \textit{chat} dopodichè, utilizzando l'area di testo, va alla ricerca della \textit{chat} desiderata. Pasquale seleziona la \textit{chat} cercata relativa agli studenti, ma il sistema nega la visualizzazione di quest’ultima e dei canali in essa contenuti mostrando un messaggio di errore.\\
	
\end{itemize}

\paragraph{CUP8 - Aggiungi Canale\\}
\begin{itemize}
	\item \textit{Scenario 1:\\}
	\textit{Pasquale}, in qualità di amministratore, vuole creare un nuovo canale all’interno della \textit{chat} di “Basi di dati e sistemi informativi”. \textit{Pasquale} raggiunge la \textit{chat} di “Basi di dati e sistemi informativi”,  clicca sulla voce per la gestione dei canali e il sistema mostra tutte le operazioni relative ai canali che \textit{Pasquale} può decidere di fare all’interno della \textit{chat}. 
	\textit{Pasquale} seleziona la voce “crea nuovo canale” ed il sistema mostra le opzioni che il nuovo canale deve avere, \textit{Pasquale} inserisce il nome, procede e il sistema mostra una schermata per aggiungere gli utenti che ne faranno parte e procede all’aggiunta e cliccando sulla voce “Crea canale”. \textit{Pasquale} è avvisato tramite un messaggio di conferma che il sistema ha effettivamente creato il nuovo canale all’interno della \textit{chat}. \textit{Pasquale} visualizza il nuovo canale.\\
	
	\item \textit{Scenario 2 - Connessione assente:\\}
	\textit{Pasquale} in qualità di amministratore vuole creare un nuovo canale all’interno della \textit{chat} di “Zoologia”. \textit{Pasquale} raggiunge la \textit{chat} di “Zoologia” e seleziona la voce per la gestione dei canali.Il sistema mostra un messaggio che comunica a \textit{Pasquale} che il suo dispositivo non risulta essere connesso ad una rete internet e non consentirà la gestione dei canali della \textit{chat}.\\
	
	\item \textit{Scenario 3 - Errore di sistema:\\}
	\textit{Pasquale}, in qualità di amministratore, vuole creare un canale all’interno della \textit{chat} di “Programmazione web e Mobile”. Pasquale raggiunge la \textit{chat} di “Programmazione web e Mobile” e seleziona la voce per la gestione dei canali non visualizza nessuna evoluzione della schermata, il sistema infatti nega la visualizzazione di quest’area mostrando un messaggio di errore.\\
	
	\item \textit{Scenario 4 - Il canale non viene creato:\\}
	\textit{Pasquale}, in qualità di amministratore, vuole creare un nuovo canale all’interno della \textit{chat} di “Istologia”. \textit{Pasquale} raggiunge la \textit{chat} di “Istologia” e seleziona la voce per la gestione dei canali. Il sistema mostra tutte le operazioni relative ai canali che Pasquale può decidere di fare all’interno della \textit{chat}. 
	\textit{Pasquale} seleziona la voce “crea nuovo canale” ed il sistema mostra le opzioni che il nuovo canale deve avere, \textit{Pasquale} inserisce il nome, procede e il sistema mostra una schermata per aggiungere gli utenti che ne faranno parte e procede all’aggiunta e seleziona la voce “Crea canale”. \textit{Pasquale} non visualizza la conferma dell’effettiva creazione del canale, infatti il sistema non crea il canale e di conseguenza \textit{Pasquale} non visualizza il nuovo canale all’interno della \textit{chat}.\\
\end{itemize}

\paragraph{CUP9 - Cancella Canale\\}
\begin{itemize}
	\item \textit{Scenario 1:\\}
	\textit{Pasquale}, in qualità di amministratore, vuole eliminare un canale all’interno della \textit{chat} di “Reti di calcolatori e Sicurezza”. \textit{Pasquale} raggiunge la \textit{chat} di “Reti di calcolatori e Sicurezza” e seleziona la voce per la gestione dei canali. Il sistema mostra tutte le operazioni relative ai canali che \textit{Pasquale} può decidere di fare all’interno della \textit{chat}. 
	\textit{Pasquale} seleziona la voce “Elimina canale” ed il sistema mostra la lista di tutti i canali che sono presenti nella \textit{chat} di “Reti di calcolatori e Sicurezza”. \textit{Pasquale} decide di eliminare il canale “Prima Esercitazione” selezionando l’apposita icona. 
	Il sistema elimina il canale e mostra un messaggio di conferma eliminazione a \textit{Pasquale} che non visualizza più il canale all’interno della \textit{chat}.\\
	
	\item \textit{Scenario 2 - Connessione assente:\\}
	\textit{Pasquale,} in qualità di amministratore, vuole eliminare un canale all’interno della \textit{chat} del corso di “Zoologia”. \textit{Pasquale} raggiunge la \textit{chat} di “Zoologia” e seleziona la voce per la gestione dei canali. Il sistema mostra un messaggio che comunica a \textit{Pasquale} che il suo dispositivo non risulta essere connesso ad una rete \textit{internet} e non consente la gestione dei canali della \textit{chat}.\\ 
	
	\item \textit{Scenario 3 - Errore di sistema:\\}
	\textit{Pasquale}, in qualità di amministratore, vuole eliminare un canale all’interno della \textit{chat} di “Programmazione web e Mobile”. \textit{Pasquale} raggiunge la \textit{chat} di “Programmazione web e Mobile” e selezionando la voce per la gestione dei canali non visualizza nessuna evoluzione della schermata, il sistema infatti nega la visualizzazione di quest’area mostrando un messaggio di errore.\\
	
	\item \textit{Scenario 4 - Non ci sono canali nella chat:\\}
	\textit{Pasquale}, in qualità di amministratore, vuole eliminare un canale all’interno della \textit{chat} di “Sistemi operativi”. \textit{Pasquale} raggiunge la \textit{chat} di “Sistemi operativi” e clicca sulla voce per la gestione dei canali. Il sistema mostra tutte le operazioni relative ai canali che \textit{Pasquale} può decidere di fare all’interno della \textit{chat}. 
	\textit{Pasquale} seleziona la voce “Elimina canale” ed il sistema avvisa Pasquale tramite un messaggio che nella \textit{chat} non risultano essere attivi dei canali.\\
	
	\item \textit{Scenario 5 - Il canale non viene eliminato:\\}
	\textit{Pasquale}, in qualità di amministratore, vuole eliminare un canale all’interno della \textit{chat} di “Reti di calcolatori e Sicurezza”. \textit{Pasquale} raggiunge la \textit{chat} di “Reti di calcolatori e Sicurezza” e seleziona la voce per la gestione dei canali. Il sistema mostra tutte le operazioni relative ai canali che \textit{Pasquale} può decidere di fare all’interno della \textit{chat}. 
	Pasquale seleziona la voce “Elimina canale” ed il sistema mostra la lista di tutti i canali che sono presenti nella \textit{chat} di “Reti di calcolatori e Sicurezza”. \textit{Pasquale} decide di eliminare il canale “Prima Esercitazione” selezionando l’apposita icona. 
	Il sistema non elimina il canale e non mostra alcun messaggio di conferma, \textit{Pasquale} infatti visualizza ancora il canale all’interno della \textit{chat}. \\
\end{itemize}

\paragraph{CUP10 - Visualizza lista utenti\\}
\begin{itemize}
	\item \textit{Scenario 1:\\}
	\textit{Pasquale}, si trova in un canale di una \textit{chat} e vuole visualizzare i membri che popolano il canale. Seleziona la voce “Visualizza lista utenti” e visualizza il numero e la lista degli utenti del canale.\\
	
	\item \textit{Scenario 2 - Connessione assente:\\}
	\textit{Pasquale}, si trova in un canale di una \textit{chat} e vuole visualizzare i membri che popolano il canale. Seleziona la voce “Visualizza lista utenti”, ma visualizza un messaggio che lo informa della mancanza di connessione del suo dispositivo. Il sistema non é in grado di caricare la lista degli utenti.\\
	
	\item \textit{Scenario 3 - Errore di sistema:\\}
	\textit{Pasquale}, si trova in un canale di una \textit{chat} e vuole visualizzare i membri che popolano il canale. Seleziona la voce “Visualizza lista utenti”, ma visualizza un messaggio che lo informa di un errore di sistema che non permette il caricamento della lista degli utenti.\\
\end{itemize}

\paragraph{CUP11 - Aggiungi un utente ad un canale\\}
\begin{itemize}
	\item \textit{Scenario 1:\\}
	\textit{Pasquale}, in qualità di amministratore, su richiesta di \textit{Giovanni}, vuole aggiungerlo in una specifica \textit{chat}. \textit{Pasquale}, entra nella \textit{chat} del corso, seleziona la voce “Aggiungi membro” e inserisce la matricola di \textit{Giovanni} da aggiungere alla \textit{chat}.\\
	
	\item \textit{Scenario 2 - Connessione assente:\\}
	\textit{Pasquale}, in qualità di amministratore del sistema, in seguito ad una richiesta di \textit{Giovanni} di essere aggiunto ad un canale, procede con la selezione della voce “Aggiungi membro”. Tuttavia, in seguito a tale operazione visualizza una pagina di assenza di connessione.\\
	
	\item \textit{Scenario 3 - Identificativo non trovato:\\}
	\textit{Pasquale}, in qualità di amministratore, su richiesta di \textit{Giovanni} che vuole essere membro di una \textit{chat} di corso degli anni precedenti al suo. Entra nella \textit{chat} del corso, seleziona la voce “Aggiungi membro” e inserisce l’identificativo di \textit{Giovanni}, tuttavia il sistema lo avvisa che l’identificativo inserito non è stato trovato.\\
\end{itemize}

\paragraph{CUP12 - Rimuovere un utente da un canale \\}
\begin{itemize}
	\item \textit{Scenario 1:\\}
	\textit{Pasquale}, in qualità di amministratore, su richiesta di \textit{Giovanni}, il quale è stato inserito erroneamente oppure, per varie ragioni, non vuole più far parte di quel canale di discussione, procede con la rimozione di quest’ultimo dal canale.
	Pasquale entra nella \textit{chat} del corso, seleziona il canale all’interno del quale si vuole rimuovere \textit{Giovanni}, e visualizzata la lista degli utenti, ricerca l’identificativo di \textit{Giovanni}, lo seleziona e sceglie la voce “rimuovi membro”.\\
	
	\item \textit{Scenario 2 - Connessione assente:\\}
	\textit{Pasquale}, in qualità di amministratore del sistema, in seguito alla richiesta di \textit{Giovanni} di essere rimosso da un canale,  entra nella \textit{chat} del corso e seleziona lo specifico canale. Quindi visualizza la lista degli utenti, seleziona la voce rimuovi utente e
	visualizza una pagina di errore a causa dell’assenza di connessione.\\
	
	\item \textit{Scenario 3 - Utente non trovato:\\}
	\textit{Pasquale}, in qualità di amministratore, su richiesta di \textit{Giovanni} il quale è stato inserito erroneamente o per varie ragioni non fa più parte di quel canale di discussione, entra nella \textit{chat} del corso e seleziona il canale dove si vuole rimuovere \textit{Giovanni}. Quindi visualizza la lista degli utenti e ricerca l’identificativo di \textit{Giovanni}, tuttavia non riesce a trovarlo in quanto non risulta presente in quel canale.\\
\end{itemize}

\paragraph{CUP13 - Silenziare un utente in un canale\\}
\begin{itemize}
	\item \textit{Scenario 1 - Silenzia utente da un messaggio segnalato:\\}
	\textit{Pasquale}, in qualità di amministratore del sistema, dopo essersi accertato di un messaggio offensivo di \textit{Giacomo},  decide di silenziare l’utente autore del messaggio.
	\textit{Pasquale} visualizza un messaggio segnalato e seleziona il pulsante che permette di silenziare l’utente autore del messaggio. \textit{Pasquale} visualizza un messaggio di conferma del silenziamento dell’utente nel canale in cui e visualizzato il messaggio.\\
	
	\item \textit{Scenario 2 - Silenzia utente dalla lista degli utenti del canale:\\}
	\textit{Pasquale}, in qualità di amministratore del sistema, decide di segnalare uno specifico utente di un canale. Dopo aver raggiunto e visualizzato la lista degli utenti di un canale di un \textit{chat}, seleziona l’utente e, dopo che il sistema gli mostra l’apposito pulsante, silenzia l’utente selezionando il pulsante preposto. \textit{Pasquale} visualizza la riuscita dell’operazione tramite un messaggio di conferma. \\
	
	\item \textit{Scenario 3 - Connessione assente:\\}
	\textit{Pasquale} visualizza il messaggio segnalato e seleziona il pulsante che permette di silenziare l’utente autore del messaggio. \textit{Pasquale} visualizza un messaggio che lo informa della difficoltà del sistema nel procedere con l’azione richiesta a causa di una mancata connessione ad internet.\\
	
	\item \textit{Scenario 3.1 - Connessione assente:\\}
	\textit{Pasquale} visualizza la lista degli utenti di un canale e seleziona uno specifico utente. \textit{Pasquale} visualizza un messaggio che lo informa della difficoltà del sistema nel procedere con l’azione richiesta a causa di una mancata connessione ad internet.\\
	
	\item \textit{Scenario 4 -  L’utente non viene silenziato:\\}
	\textit{Pasquale}, in qualità di amministratore del sistema, legge un messaggio inopportuno da parte di \textit{Giacomo}, seleziona il comando “Silenzia”, ma compare un messaggio che avvisa \textit{Pasquale} che \textit{Giacomo} non è stato silenziato.\\
	
	\item \textit{Scenario 4.1 -  L’utente non viene silenziato:\\}
	\textit{Pasquale}, in qualità di amministratore del sistema, decide di segnalare uno specifico utente di un canale. Quindi visualizza la lista degli utenti di un canale di una \textit{chat} e seleziona l’apposito pulsante per segnalare \textit{Giacomo}. Pasquale però non visualizza nessuna modifica allo stato dell’utente, ma visualizza un messaggio di errore. L’utente non è stato silenziato.\\ 
\end{itemize}

\paragraph{CUP14 - Reintegra un utente in un canale\\}
\begin{itemize}
	\item \textit{Scenario 1:\\}
	\textit{Pasquale}, in qualità di amministratore, entrato nel canale e visualizzata la lista degli utenti del canale della \textit{chat}, seleziona l’utente precedentemente silenziato e seleziona il pulsante per reintegrare l’utente ed il sistema reintegra l’utente nel canale. \textit{Pasquale} visualizza un messaggio di conferma della modifica dell’utente.\\
	
	\item \textit{Scenario 2 - Connessione assente:\\}
	\textit{Pasquale}, in qualità di amministratore, entrato nel canale e visualizzata la lista degli utenti del canale della \textit{chat}, seleziona l’utente precedentemente silenziato, ma visualizza un messaggio di impossibilità da parte del sistema di procedere in quanto il dispositivo in uso da \textit{Pasquale} non è connesso ad internet. \\
	
	\item \textit{Scenario 3 - L'utente non è silenziato:\\}
	\textit{Pasquale}, in qualità di amministratore, entrato nel canale visualizza la lista degli utenti del canale della \textit{chat}, ma la lista del canale non contiene nessun utente silenziato.\\
	
	\item \textit{Scenario 3 - L'utente non è silenziato:\\}
	\textit{Pasquale}, in qualità di amministratore, entrato nel canale visualizza la lista degli utenti del canale della \textit{chat}, seleziona l’utente precedentemente silenziato e seleziona il pulsante per reintegrare l’utente, ma il sistema lo avvisa che non è stato silenziato a causa di un errore di sistema. 
\end{itemize}

\paragraph{CUP15 - Modificare i permessi di un utente in un canale \\}
\begin{itemize}
	\item \textit{Scenario 1:\\}
	\textit{Pasquale}, in qualità di amministratore del sistema, vuole modificare i permessi di un utente per un determinato canale. 
	\textit{Pasquale}, dopo aver visualizzato la lista degli utenti di uno specifico canale della \textit{chat}, seleziona  l’utente interessato e modifica il suo ruolo rendendolo amministratore o revocandogli i permessi di amministratore dal canale.\\
	
	\item \textit{Scenario 2 - Connessione assente:\\}
	\textit{Pasquale}, in qualità di amministratore del sistema, viene richiesto da \textit{Giacomo} di modificare i permessi di un utente per un determinato canale di una \textit{chat}.
	\textit{Pasquale} visualizza la lista degli utenti di uno specifico canale della \textit{chat} e procede col selezionare l’utente interessato, ma visualizza un messaggio che lo informa della mancanza di connessione e, quindi, non puó procedere con l’operazione di modifica.\\
	
	\item \textit{Scenario 3 - L'utente ricopre già il ruolo assegnato:\\}
	\textit{Pasquale}, in qualità di amministratore del sistema, a seguito di una richiesta di \textit{Giacomo}, modifica i permessi di un utente per una determinata \textit{chat}.
	\textit{Pasquale} dalla lista degli utenti del canale seleziona l’utente interessato ed effettua la modifica dei permessi, ma visualizza un messaggio che lo avvisa che l’utente ricopre già il ruolo che si vuole assegnare.\\
	
	
\end{itemize}
\paragraph{CUP16 - Nascondi messaggio\\}
\begin{itemize}
	\item \textit{Scenario 1 - Nascondi messaggio non segnalato:\\}
	\textit{Pasquale}, in qualità di amministratore, vuole visualizzare uno specifico messaggio non segnalato e tramite l’icona preposta vuole nascondere il contenuto del messaggio all’interno del canale. Seleziona il messaggio e visualizza l’icona che permette al sistema di nascondere il contenuto del singolo messaggio all’interno del canale. \textit{Pasquale} clicca sull’icona ed il sistema nasconde il contenuto del messaggio. \textit{Pasquale}, quindi, non visualizza più il testo del messaggio ma al suo posto legge la scritta “Questo messaggio è stato nascosto”.\\
	
	\item \textit{Scenario 2 - Nascondi messaggio segnalato:\\}
	\textit{Pasquale}, in qualità di amministratore, vuole visualizzare uno specifico messaggio che risulta essere stato segnalato e tramite l’icona preposta vuole nascondere il contenuto del messaggio considerato inopportuno dal canale. Seleziona il messaggio e visualizza l’icona che permette al sistema di nascondere il contenuto del singolo messaggio all’interno del canale. \textit{Pasquale} clicca sull’icona ed il sistema nasconde il contenuto del messaggio. \textit{Pasquale}, quindi, non visualizza più il testo del messaggio, ma al suo posto legge la scritta “Questo messaggio è stato nascosto” ed il messaggio non sarà più evidenziato.\\
	
	\item \textit{Scenario 3 - Connessione assente:\\}
	\textit{Pasquale}, in qualità di amministratore, vuole visualizzare uno specifico messaggio e tramite l’icona preposta vuole nascondere il contenuto del messaggio all’interno del canale. Seleziona il messaggio e visualizza un messaggio che lo avvisa che il suo dispositivo non ha una connessione ad internet.\\
	
	\item \textit{Scenario 4 - Errore di sistema:\\}
	\textit{Pasquale}, in qualità di amministratore, vuole visualizzare uno specifico messaggio e tramite l’icona preposta vuole nascondere il contenuto del messaggio all’interno del canale. Seleziona il messaggio e visualizza un messaggio che informa Pasquale dell’impossibilità da parte del sistema di procedere.\\
	
	\item \textit{Scenario 5 - Lo stato del messaggio non cambia:\\}
	\textit{Pasquale}, in qualità di amministratore, visualizza uno specifico messaggio e tramite l’icona che permette di nascondere  un messaggio vuole modificare lo stato attuale del messaggio, overo vuole nascondere il suo testo all’interno del canale. Seleziona il messaggio e visualizza tutte le icone che permettono una modifica allo stato attuale del messaggio . \textit{Pasquale} seleziona l’icona per nascondere il testo del messaggio, ma non visualizza nessun cambiamento né un messaggio di conferma dell’operazione. Di fatto il sistema non ha cambiato lo stato del messaggio.\\
\end{itemize}

\paragraph{CUP17 - Reintegra messaggio\\}
\begin{itemize}
	\item \textit{Scenario 1:\\}
	\textit{Pasquale}, in qualità di amministratore, vuole visualizzare uno specifico messaggio segnalato e tramite l’icona preposta vuole reintegrare il messaggio all’interno della \textit{chat}, ovvero vuole renderlo non più evidenziato come inopportuno, ma allo stesso tempo conservare l’originalità del testo in quanto ritenuto non offensivo. Seleziona il messaggio e visualizza l’icona che permette al sistema di togliere la segnalazione al singolo messaggio all’interno del canale. \textit{Pasquale} seleziona l’icona ed il sistema rimuove la segnalazione al messaggio. \textit{Pasquale}, quindi, non visualizza più il testo del messaggio evidenziato, ma lo visualizza come un semplice messaggio della conversazione.\\
	
	\item \textit{Scenario 2 - Connessione assente:\\}
	\textit{Pasquale}, in qualità di amministratore, vuole visualizzare uno specifico messaggio segnalato e tramite l’icona preposta vuole reintegrare il  messaggio all’interno del canale. Seleziona il messaggio e visualizza un messaggio che lo avvisa che il suo dispositivo non ha una connessione ad internet.\\
	
	\item \textit{Scenario 3 - Errore di sistema:\\}
	\textit{Pasquale}, in qualità di amministratore, vuole visualizzare uno specifico messaggio segnalato e tramite l’icona preposta vuole reintegrare il  messaggio all’interno del canale. Seleziona il messaggio e visualizza un messaggio che informa \textit{Pasquale} dell’impossibilità da parte del sistema di procedere alle operazioni per la reintegrazione del messaggio.\\
	
	\item \textit{Scenario 4 - Lo stato del messaggio non cambia:\\}
	\textit{Pasquale}, in qualità di amministratore, vuole visualizzare uno specifico messaggio segnalato e tramite l’icona preposta vuole reintegrare il  messaggio all’interno del canale. 
	Seleziona il messaggio e visualizza tutte le icone che permettono una modifica allo stato attuale del messaggio. \textit{Pasquale} seleziona l’icona per reintegrare un messaggio all’interno del canale ma non visualizza nessun cambiamento né un messaggio di conferma dell’operazione. Di fatto il sistema non ha cambiato lo stato del messaggio che resta dunque segnalato.\\
\end{itemize}

\paragraph{CUP18 - Invio Notifiche \\}
\begin{itemize}
	\item \textit{Scenario 1:\\}
	\textit{Pasquale}, in qualità di amministratore, vuole scrivere ed inviare una notifica a tutte le \textit{chat} che fanno parte del corso di laurea di Informatica. \textit{Pasquale} dal menù laterale seleziona la voce “Invio Notifiche”. Il sistema mostra una nuova schermata. \textit{Pasquale} scrive il messaggio della notifica nell’apposita area di testo e seleziona come destinatarie della notifica tutte le \textit{chat} che fanno parte del corso di laurea di “Informatica” tramite il filtro “Informatica”. In seguito seleziona la voce “Invia Notifica” e \textit{Pasquale} visualizza un messaggio di conferma dell’invio effettivo della notifica alle \textit{chat} selezionate.\\
	
	\item \textit{Scenario 2 - Connessione assente:\\}
	\textit{Pasquale}, in qualità di amministratore, vuole scrivere ed inviare una notifica a tutte le \textit{chat} che fanno parte del dipartimento di “Bioscienze e Territorio”. \textit{Pasquale} dal menù laterale seleziona la voce “Invio Notifiche”. Il sistema mostra un messaggio che comunica a \textit{Pasquale} che il suo dispositivo non risulta essere connesso ad una rete internet e non consente la scrittura e l’invio della notifica.\\
	
	\item \textit{Scenario 3 - Errore di sistema:\\}
	\textit{Pasquale}, in qualità di amministratore, vuole scrivere ed inviare una notifica a tutte le \textit{chat} che fanno parte del dipartimento di Informatica. \textit{Pasquale} dal menù laterale seleziona la voce “Invio Notifiche” non visualizza, però, alcuna evoluzione della schermata del pannello. Il sistema nega, infatti, la visualizzazione di quest’area, mostrando un messaggio di errore.\\
	
	\item \textit{Scenario 4 - Il sistema non permette la digitazione del messaggio:\\}
	\textit{Pasquale}, in qualità di amministratore, vuole scrivere ed inviare una notifica a tutte le \textit{chat} che fanno parte del corso di laurea di Biologia. \textit{Pasquale} dal \textit{menù} laterale seleziona la voce “Invio Notifiche”. Il sistema mostra una nuova schermata. \textit{Pasquale} tenta di scrivere il messaggio della notifica nell’apposita area di testo, ma il sistema non permette la digitazione del messaggio. \textit{Pasquale} ritenta l’operazione da capo.\\
	
	\item \textit{Scenario 4 - Il campo di testo è vuoto:\\}
	\textit{Pasquale}, in qualità di amministratore, vuole scrivere ed inviare una notifica alla \textit{chat} “Algoritmi e strutture dati”. \textit{Pasquale} dal menù laterale seleziona la voce “Invio Notifiche”. Il sistema mostra una  nuova schermata. \textit{Pasquale} seleziona come destinatarie della notifica la \textit{chat} “Algoritmi e strutture dati” per farlo effettua una ricerca tramite l’apposita area.
	In seguito clicca su “Invia Notifica” e \textit{Pasquale} visualizza un messaggio di errore in cui il sistema lo avvisa della mancanza della scrittura di un messaggio nella notifica e il sistema non invia la notifica. \textit{Pasquale} completa l’operazione scrivendo il messaggio. 
\end{itemize}

\paragraph{CUP19 - Gestione messaggi inopportuni\\}
\begin{itemize}
	\item \textit{Scenario 1:\\}
	\textit{Pasquale}, in qualità di amministratore, vuole visualizzare la sezione dedicata ai messaggi segnalati come inopportuni. Seleziona dal \textit{menù} laterale la voce di gestione dei messaggi inopportuni, il sistema mostra la lista delle \textit{chat} dove è presente almeno un messaggio segnalato ancora da verificare, \textit{Pasquale} decide di aprire la \textit{chat} “Calcolo numerico” perché risulta essere la \textit{chat} con maggior segnalazioni riportate nel \textit{flag}. Il sistema apre la \textit{chat} selezionata e \textit{Pasquale} visualizza i messaggi che risultano essere segnalati.\\
	
	\item \textit{Scenario 2 - Connessione assente:\\}
	\textit{Pasquale}, in qualità di amministratore del sistema, vuole visualizzare la sezione dedicata ai messaggi segnalati come inopportuni. Pertanto dalla schermata di \textit{home} seleziona la voce di gestione dei messaggi inopportuni. Il dispositivo di \textit{Pasquale} però, non ha in quel momento una connessione ad internet disponibile, di conseguenza, il sistema non consente la selezione dell’area richiesta e comunica a \textit{Pasquale} l’assenza di connessione tramite un messaggio di errore.\\
	
	\item \textit{Scenario 3 - Errore di sistema:\\}
	\textit{Pasquale}, in qualità di amministratore del sistema, vuole visualizzare la sezione dedicata ai messaggi segnalati come inopportuni. Pertanto dalla schermata di \textit{home} seleziona la voce di gestione dei messaggi inopportuni. Il sistema però non consente la selezione dell’area richiesta e mostra a \textit{Pasquale} un messaggio di errore di sistema.\\
	
	\item \textit{Scenario 4 -  La chat non contiene messaggi inopportuni:\\}
	\textit{Pasquale}, in qualità di amministratore vuole visualizzare la sezione dedicata ai messaggi segnalati come inopportuni. Seleziona dal \textit{menù} laterale la voce di gestione dei messaggi inopportuni, il sistema mostra la lista delle \textit{chat} dove è presente almeno un messaggio segnalato ancora da verificare, \textit{Pasquale} decide di aprire la \textit{chat} di “Fisica”. Il sistema, a questo punto, apre la \textit{chat} con i relativi messaggi, non evidenziando, però, alcuno di essi. 
	\textit{Pasquale} ripete da capo l’operazione per verificare che vi sia stato effettivamente un errore e non ci siano dunque messaggi inopportuni nel canale.\\
\end{itemize}