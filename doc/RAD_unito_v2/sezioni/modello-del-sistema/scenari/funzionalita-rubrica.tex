\subsection{Funzionalità rubrica}

\paragraph{Ricerca non filtrata \\}
\begin{itemize}
	\item \textit{Scenario 1:\\}
	\textit{Antonio}, \textit{Studente} frequentante il primo anno di Informatica, seleziona la sezione "rubrica" dalla voce del menu e, immediatamente, riesce a visualizzare la rubrica completa.\\
	
	\item \textit{Scenario 2 - Connessione assente:\\}
	\textit{Antonio}, \textit{Studente} frequentante il primo anno di Informatica, seleziona la sezione "rubrica" dalla voce del menu, ma riscontra problemi con la connessione pertanto il sistema restituisce l'ultima copia salvata della rubrica.\\
	
	\item \textit{Scenario 3 - Copia non presente:\\}
	\textit{Antonio}, \textit{Studente} frequentante il primo anno di Informatica, seleziona la sezione "rubrica" dalla voce del menu, ma riscontra problemi con la connessione pertanto il sistema, non avendo una copia salvata da visualizzare, mostra un messaggio di errore.\\	
\end{itemize}

\paragraph{Ricerca filtrata \\}
\begin{itemize}
	\item \textit{Scenario 1:\\}
	\textit{Antonio}, \textit{Studente} frequentante il primo anno di Informatica, dopo aver selezionato la sezione “rubrica” dalla voce del menu, attraverso la barra di ricerca, inserisce dei parametri di ricerca per trovare uno specifico contatto.\\
	
	\item \textit{Scenario 2 - Connessione assente:\\}
	\textit{Antonio}, \textit{Studente} frequentante il primo anno di Informatica, seleziona la sezione "rubrica" dalla voce del menu, ma riscontra problemi con la connessione pertanto il sistema restituisce l'ultima copia della rubrica salvata ed effettua la ricerca.\\
	
	\item \textit{Scenario 3 - Copia non presente:\\}
	\textit{Antonio}, \textit{Studente} frequentante il primo anno di Informatica, seleziona la sezione "rubrica" dalla voce del menu, ma riscontra problemi con la connessione pertanto il sistema, non avendo una copia salvata da visualizzare, mostra un messaggio di errore.\\
	
	\item \textit{Scenario 4 - Contatto non presente:\\}
	\textit{Antonio}, \textit{Studente} frequentante il primo anno di Informatica, dopo aver selezionato la sezione “rubrica” dalla voce del menu, attraverso la barra di ricerca, inserisce dei parametri di ricerca per trovare uno specifico contatto, ma il risultato non viene trovato pertanto il sistema visualizza un messaggio di "contatto non trovato".\\
\end{itemize}

\paragraph{Visualizza contatto \\}
\begin{itemize}
	\item \textit{Scenario 1:\\}
	\textit{Antonio}, \textit{Studente} frequentante il primo anno di Informatica, dopo aver effettuato la visualizzazione della rubrica o di uno specifico contatto, seleziona il conatto e riesce a visualizzare le informazioni relative ad esso.\\
\end{itemize}