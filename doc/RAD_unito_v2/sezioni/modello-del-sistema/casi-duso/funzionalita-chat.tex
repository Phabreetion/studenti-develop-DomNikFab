\subsection{Funzionalità chat}

\paragraph{CUS1 - Visualizza canale \\}

Lo \emph{Studente}, una volta selezionata la voce \emph{“chat”} dal \emph{menù} dell’app \emph{Studenti Unimol},  seleziona una \emph{chat} specifica della lista e visualizza i messaggi presenti nel canale di discussione della stessa. Nel caso in cui vi sia l’assenza di connessione viene visualizzata l’ultima copia salvata in locale del canale qualora essa sia presente.
\begin{table}
	\small % Dimensione testo piccola
	\label{CUS1 - Visualizza canale}
	\begin{tabular}{| p{\useCaseLeft} | p{\useCaseNum} | p{\useCaseTwoCol} | p{\useCaseTwoCol} |}
		\hline
		\textbf{Nome caso d'uso} & \multicolumn{3}{p{\useCaseMulticol} |}{\textbf{CUS 1 - Visualizza canale}} \\
		\hline
		\textbf{Attori partecipanti} & \multicolumn{3}{p{\useCaseMulticol} |}{Inizializzato da \textbf{\emph{Studente}.}} \\
		\hline
		\textbf{Condizioni d'ingresso} & \multicolumn{3}{p{\useCaseMulticol} |}{Lo \emph{Studente} seleziona la voce \emph{“chat”} dal \emph{menù} dell’\emph{app Studenti Unimol}.} \\
		\hline
		\textbf{Flusso degli eventi} & \textbf{\#} & \textbf{\emph{Studente}} & \textbf{Sistema} \\
		\hline
		\textbf{} & \textbf{1} & \textbf{} & Mostra l’\emph{home-chat} con le relative \emph{chat}; \\
		\hline
		\textbf{} & \textbf{2} & Seleziona la \emph{chat} desiderata; & \textbf{} \\
		\hline
		\textbf{} & \textbf{3} & \textbf{} & Mostra il canale di discussione di \emph{default} della \emph{chat}; \\
		\hline
		\textbf{} & \textbf{4} & Seleziona un canale di discussione alternativo; & \\
		\hline
		\textbf{} & \textbf{5} & \textbf{} & Mostra il canale di discussione selezionato; \\
		\hline
		\textbf{Eccezioni} & \multicolumn{3}{p{\useCaseMulticol} |}{1.1 - 3.1 - 5.1 Nessuna risposta dal \emph{server}: viene visualizzato un messaggio di errore; \newline1.2 - 3.2 - 5.2  Connessione assente: il sistema restituisce l’ultima copia salvata in locale; \newline 1.3 - 3.3 - 5.3 Copia non presente: viene visualizzato un messaggio di errore;} \\
		\hline
		\textbf{Condizioni d'uscita} & \multicolumn{3}{p{\useCaseMulticol} |}{Lo \emph{Studente} visualizza il canale correttamente.} \\
		\hline
	\end{tabular}
	\caption{CUS1 - Visualizza canale}
\end{table}

\newpage
	\paragraph{CUS2 - Invio messaggio \\}

Lo \emph{Studente}, all’interno di un canale di discussione, invia un messaggio selezionando l’apposito spazio dedito alla scrittura. Una volta digitato il testo del messaggio e selezionato il pulsante di invio, il sistema lo mostra nel canale di discussione. \\
\begin{table}

	\small % Dimensione testo piccola
	\label{CUS2 - Invio messaggio}
	\begin{tabular}{| p{\useCaseLeft} | p{\useCaseNum} | p{\useCaseTwoCol} | p{\useCaseTwoCol} |}
		\hline
		\textbf{Nome caso d'uso} & \multicolumn{3}{p{\useCaseMulticol} |}{\textbf{CUS2 - Invio messaggio}} \\
		\hline
		\textbf{Attori partecipanti} & \multicolumn{3}{p{\useCaseMulticol} |}{Inizializzato da \textbf{\emph{Studente}.}} \\
		\hline
		\textbf{Condizioni d'ingresso} & \multicolumn{3}{p{\useCaseMulticol} |}{Lo \emph{Studente} seleziona il canale di discussione in cui vuole inviare il messaggio.} \\
		\hline
		\textbf{Flusso degli eventi} & \textbf{\#} & \textbf{\emph{Studente}} & \textbf{Sistema} \\
		\hline
		\textbf{} & \textbf{1} & Seleziona la casella di testo, digita il messaggio ed invia; & \textbf{} \\
		\hline
		\textbf{} & \textbf{2} & \textbf{} & Riceve il messaggio e lo mostra nella \emph{chat}; \\
		\hline
		\textbf{Eccezioni} & \multicolumn{3}{p{\useCaseMulticol} |}{1.1 Connessione assente: il messaggio viene messo in coda e inviato quando sarà disponibile la connessione; \newline2.1 Nessuna risposta dal \emph{server}: viene visualizzato un messaggio di errore;} \\
		\hline
		\textbf{Condizioni d'uscita} & \multicolumn{3}{p{\useCaseMulticol} |}{Lo \emph{Studente} visualizza il messaggio nel canale della \emph{chat}.} \\
		\hline
	\end{tabular}
	\caption{CUS2 - Invio messaggio}
\end{table}


\newpage
\paragraph{CUS3 - Invio allegato \\}

Lo \emph{Studente} invia un file che viene mostrato all’interno del canale di discussione, similmente ad un messaggio. \\
	
\begin{table}
	
	\small % Dimensione testo piccola
	\label{CUS3 - Invio allegato}	
	\begin{tabular}{| p{\useCaseLeft} | p{\useCaseNum} | p{\useCaseTwoCol} | p{\useCaseTwoCol} |}
		\hline
		\textbf{Nome caso d'uso} & \multicolumn{3}{p{\useCaseMulticol} |}{\textbf{CUS3 - Invio Allegato}} \\
		\hline
		\textbf{Attori partecipanti} & \multicolumn{3}{p{\useCaseMulticol} |}{Inizializzato da \textbf{\emph{Studente}.}} \\
		\hline
		\textbf{Condizioni d'ingresso} & \multicolumn{3}{p{\useCaseMulticol} |}{Lo \emph{Studente} seleziona un canale di discussione.} \\
		\hline
		\textbf{Flusso degli eventi} & \textbf{\#} & \textbf{\emph{Studente}} & \textbf{Sistema} \\
		\hline
		\textbf{} & \textbf{1} & Seleziona il pulsante di scelta allegato; & \textbf{} \\
		\hline
		\textbf{} & \textbf{2} & \textbf{} & Mostra l’elenco dei \emph{file} presenti sul dispositivo dell’utente; \\
		\hline
		\textbf{} & \textbf{3} & Seleziona il \emph{file} da allegare; & \textbf{} \\
		\hline
		\textbf{} & \textbf{4} & \textbf{} & Controlla se l’allegato è idoneo all’invio nel canale di comunicazione e mostra un messaggio di conferma; \\
		\hline
		\textbf{} & \textbf{5} & Invia l’allegato; & \textbf{} \\
		\hline
		\textbf{} & \textbf{6} & \textbf{} & Conferma l’invio dell’allegato; \\
		\hline
		\textbf{Eccezioni} & \multicolumn{3}{p{\useCaseMulticol} |}{2.1 - 4.1 Nessuna risposta dal \emph{server}: viene visualizzato un messaggio di errore; \newline 3.1 Il \emph{file} non è idoneo: viene visualizzato un messaggio di errore; \newline 4.2 Connessione assente: l’allegato viene messo in coda e inviato quando sarà disponibile la connessione;} \\
		\hline
		\textbf{Condizioni d'uscita} & \multicolumn{3}{p{\useCaseMulticol} |}{Il sistema mostra l’allegato nel canale di comunicazione.} \\
		\hline
	\end{tabular}
	\caption{CUS3 - Invio allegato}
\end{table}


\newpage
\paragraph{CUS4 - Rispondi a singolo messaggio \\}
Lo \emph{Studente} visualizza il canale di discussione  desiderato e seleziona il singolo messaggio a cui desidera rispondere. \\
\begin{table}
	\small % Dimensione testo piccola
	\label{CUS4 - Rispondi a singolo messaggio}
	\begin{tabular}{| p{\useCaseLeft} | p{\useCaseNum} | p{\useCaseTwoCol} | p{\useCaseTwoCol} |}
		\hline
		\textbf{Nome caso d'uso} & \multicolumn{3}{p{\useCaseMulticol} |}{\textbf{CUS4 - Rispondi a singolo messaggio}} \\
		\hline
		\textbf{Attori partecipanti} & \multicolumn{3}{p{\useCaseMulticol} |}{Inizializzato da \textbf{\emph{Studente}.}} \\
		\hline
		\textbf{Condizioni d'ingresso} & \multicolumn{3}{p{\useCaseMulticol} |}{Lo \emph{Studente} seleziona il canale di discussione che contiene il messaggio a cui vuole rispondere.} \\
		\hline
		\textbf{Flusso degli eventi} & \textbf{\#} & \textbf{\emph{Studente}} & \textbf{Sistema} \\
		\hline
		\textbf{} & \textbf{1} & Seleziona un messaggio; & \textbf{} \\
		\hline
		\textbf{} & \textbf{2} & \textbf{} & Mostra il \emph{menú}; \\
		\hline
		\textbf{} & \textbf{3} & Seleziona l’opzione di risposta a messaggio; & \textbf{} \\
		\hline
		\textbf{} & \textbf{4} & \textbf{} & Evidenzia il messaggio a cui si vuole rispondere; \\
		\hline
		\textbf{Eccezioni} & \multicolumn{3}{p{\useCaseMulticol} |}{ \textbf{} } \\
		\hline
		\textbf{Condizioni d'uscita} & \multicolumn{3}{p{\useCaseMulticol} |}{Lo \emph{Studente} visualizza il messaggio evidenziato.} \\
		\hline
	\end{tabular}
	\caption{CUS4 - Rispondi a singolo messaggio}
\end{table}


\newpage
\paragraph{CUS5 - Scarica allegato \\}
Lo \emph{Studente} è intenzionato a scaricare un \emph{file} presente nel canale di discussione. \\
\begin{table}
	\small % Dimensione testo piccola
	\label{CUS5 - Scarica allegato}
	
	\begin{tabular}{| p{\useCaseLeft} | p{\useCaseNum} | p{\useCaseTwoCol} | p{\useCaseTwoCol} |}
		\hline
		\textbf{Nome caso d'uso} & \multicolumn{3}{p{\useCaseMulticol} |}{\textbf{CUS5 - Scarica allegato}} \\
		\hline
		\textbf{Attori partecipanti} & \multicolumn{3}{p{\useCaseMulticol} |}{Inizializzato da \textbf{\emph{Studente}.}} \\
		\hline
		\textbf{Condizioni d'ingresso} & \multicolumn{3}{p{\useCaseMulticol} |}{Lo \emph{Studente} seleziona un canale di discussione.} \\
		\hline
		\textbf{Flusso degli eventi} & \textbf{\#} & \textbf{\emph{Studente}} & \textbf{Sistema} \\
		\hline
		\textbf{} & \textbf{1} & Selezione il messaggio dove è presente l’allegato; & \textbf{} \\
		\hline
		\textbf{} & \textbf{2} & \textbf{} & Mostra l’opzione di scaricare l’allegato associato al messaggio selezionato; \\
		\hline
		\textbf{} & \textbf{3} & Seleziona l’opzione per scaricare l’allegato; & \textbf{} \\
		\hline
		\textbf{Eccezioni} & \multicolumn{3}{p{\useCaseMulticol} |}{1.1  Connessione assente: viene visualizzato un messaggio di errore; \newline3.1 Lo \emph{Studente} nega il \emph{download} dell’allegato: l’allegato non viene salvato sul dispositivo;} \\
		\hline
		\textbf{Condizioni d'uscita} & \multicolumn{3}{p{\useCaseMulticol} |}{Il sistema salva il file sul dispositivo.} \\
		\hline
	\end{tabular}
	\caption{CUS5 - Scarica allegato}
\end{table}


\newpage
\paragraph{CUS6 - Segnalazione messaggio \\}
Lo \emph{Studente} dopo aver visualizzato la chat interessata seleziona il messaggio che desidera segnalare. \\
\begin{table}
	\small % Dimensione testo piccola
	\label{CUS6 - Segnalazione messaggio}
	
	\begin{tabular}{| p{\useCaseLeft} | p{\useCaseNum} | p{\useCaseTwoCol} | p{\useCaseTwoCol} |}
		\hline
		\textbf{Nome caso d'uso} & \multicolumn{3}{p{\useCaseMulticol} |}{\textbf{CUS6 - Segnalazione messaggio}} \\
		\hline
		\textbf{Attori partecipanti} & \multicolumn{3}{p{\useCaseMulticol} |}{Inizializzato da \textbf{\emph{Studente}.}} \\
		\hline
		\textbf{Condizioni d'ingresso} & \multicolumn{3}{p{\useCaseMulticol} |}{Lo \emph{Studente} seleziona il canale di discussione che contiene il messaggio che vuole segnalare.} \\
		\hline
		\textbf{Flusso degli eventi} & \textbf{\#} & \textbf{\emph{Studente}} & \textbf{Sistema} \\
		\hline
		\textbf{} & \textbf{1} & Seleziona un messaggio; & \textbf{} \\
		\hline
		\textbf{} & \textbf{2} & \textbf{} & Mostra il \emph{menú}; \\
		\hline
		\textbf{} & \textbf{3} & Seleziona l’opzione “Segnala”; & \textbf{} \\
		\hline
		\textbf{} & \textbf{4} & \textbf{} & Chiede conferma di invio segnalazione; \\
		\hline
		\textbf{} & \textbf{5} & Conferma invio segnalazione; & \textbf{} \\
		\hline
		\textbf{} & \textbf{6} & \textbf{} & Conferma l’avvenuta segnalazione;  \\
		\hline
		\textbf{Eccezioni} & \multicolumn{3}{p{\useCaseMulticol} |}{3.1 Connessione assente: viene visualizzato un messaggio di errore; \newline4.1 - 6.1 Nessuna risposta dal \emph{server}: verrà visualizzato un messaggio di errore;} \\
		\hline
		\textbf{Condizioni d'uscita} & \multicolumn{3}{p{\useCaseMulticol} |}{Il sistema conferma il successo dell’operazione.} \\
		\hline
	\end{tabular}
	\caption{CUS6 - Segnalazione messaggio}
\end{table}


\newpage
\paragraph{CUS7 - Ricerca testo nella chat \\}
Lo \emph{Studente} vuole visualizzare vecchi messaggi seleziona la voce “cerca” nel \emph{menù} interno del canale di discussione, digitando il testo da cercare in un’apposita casella di testo. Il sistema mostra tutti i messaggi che contengono il testo partendo dal più recente. \\
\begin{table}
	\small % Dimensione testo piccola
	\label{CUS7 - Ricerca testo nella chat}
	
	\begin{tabular}{| p{\useCaseLeft} | p{\useCaseNum} | p{\useCaseTwoCol} | p{\useCaseTwoCol} |}
		\hline
		\textbf{Nome caso d'uso} & \multicolumn{3}{p{\useCaseMulticol} |}{\textbf{CUS7 - Ricerca testo nella chat}} \\
		\hline
		\textbf{Attori partecipanti} & \multicolumn{3}{p{\useCaseMulticol} |}{Inizializzato da \textbf{\emph{Studente}.}} \\
		\hline
		\textbf{Condizioni d'ingresso} & \multicolumn{3}{p{\useCaseMulticol} |}{Lo \emph{Studente} seleziona un canale di discussione.} \\
		\hline
		\textbf{Flusso degli eventi} & \textbf{\#} & \textbf{\emph{Studente}} & \textbf{Sistema} \\
		\hline
		\textbf{} & \textbf{1} & Accede alla sezione \emph{“menù”} del canale di discussione; & \textbf{} \\
		\hline
		\textbf{} & \textbf{2} & \textbf{} & Mostra il \emph{menú}; \\
		\hline
		\textbf{} & \textbf{3} & Seleziona il pulsante di ricerca; & \textbf{} \\
		\hline
		\textbf{} & \textbf{4} & \textbf{} &  Mostra la casella di testo di ricerca; \\
		\hline
		\textbf{} & \textbf{5} & Digita il testo da cercare; & \textbf{} \\
		\hline
		\textbf{} & \textbf{6} & \textbf{} & Evidenzia il testo cercato nei messaggi, mostrando prima il più recente;\\
		\hline
		\textbf{} & \textbf{7} & Scorre i messaggi fino a trovare quello ricercato; & \textbf{} \\
		\hline
		\textbf{Eccezioni} & \multicolumn{3}{p{\useCaseMulticol} |}{2.1 - 4.1 - 6.1  Nessuna risposta dal \emph{server}: viene visualizzato un messaggio di errore; \newline6.2 Il testo cercato non è presente: viene visualizzato un messaggio di notifica;} \\
		\hline
		\textbf{Condizioni d'uscita} & \multicolumn{3}{p{\useCaseMulticol} |}{Il sistema evidenzia i messaggi che contengono il testo cercato.} \\
		\hline
	\end{tabular}
	\caption{CUS7 - Ricerca testo nella chat}
\end{table}


\newpage
\paragraph{CUS8 - Tag membro in messaggio \\}
Lo \emph{Studente} che si trova all’interno di un canale di discussione ha la possibilitá di richiamare l’attenzione di uno specifico membro durante la digitazione di un messaggio. Lo studente digita il nome del membro preceduto da un carattere speciale (@). Il membro selezionato viene avvisato tramite una notifica diretta. \\
\begin{table}
	\small % Dimensione testo piccola
	\label{CUS8 - Tag membro in messaggio}
	
	\begin{tabular}{| p{\useCaseLeft} | p{\useCaseNum} | p{\useCaseTwoCol} | p{\useCaseTwoCol} |}
		\hline
		\textbf{Nome caso d'uso} & \multicolumn{3}{p{\useCaseMulticol} |}{\textbf{CUS8 - Tag membro in messaggio}} \\
		\hline
		\textbf{Attori partecipanti} & \multicolumn{3}{p{\useCaseMulticol} |}{Inizializzato da \textbf{\emph{Studente}.}} \\
		\hline
		\textbf{Condizioni d'ingresso} & \multicolumn{3}{p{\useCaseMulticol} |}{Lo \emph{Studente} seleziona un canale di discussione.} \\
		\hline
		\textbf{Flusso degli eventi} & \textbf{\#} & \textbf{\emph{Studente}} & \textbf{Sistema} \\
		\hline
		\textbf{} & \textbf{1} & Seleziona la  casella di testo quindi digita il nome di un membro preceduto dal carattere speciale; & \textbf{} \\
		\hline
		\textbf{} & \textbf{2} & \textbf{} & Mostra la lista di utenti che corrispondono al nome inserito; \\
		\hline
		\textbf{} & \textbf{3} & Seleziona il membro da taggare; & \textbf{} \\
		\hline
		\textbf{Eccezioni} & \multicolumn{3}{p{\useCaseMulticol} |}{1.1 Lo  \emph{Studente} digita il nome errato: il testo digitato viene inserito come messaggio di testo;} \\
		\hline
		\textbf{Condizioni d'uscita} & \multicolumn{3}{p{\useCaseMulticol} |}{Il sistema mostra il messaggio contenente il  \emph{tag} all’interno della casella di testo.} \\
		\hline
	\end{tabular}
	\caption{CUS8 - Tag membro in messaggio}
\end{table}


\newpage
\paragraph{CUS9 - Gestisci notifiche chat \\}
Lo \emph{Studente} che si trova all’interno di una \emph{chat} ha la possibilitá di accedere al \emph{menù} e gestire le notifiche ovvero di attivarle o disattivarle in base alle sue preferenze. \\
\begin{table}
	\small % Dimensione testo piccola
	\label{CUS9 - Gestisci notifiche chat}
	
	\begin{tabular}{| p{\useCaseLeft} | p{\useCaseNum} | p{\useCaseTwoCol} | p{\useCaseTwoCol} |}
		\hline
		\textbf{Nome caso d'uso} & \multicolumn{3}{p{\useCaseMulticol} |}{\textbf{CUS9 - Gestisci notifiche chat}} \\
		\hline
		\textbf{Attori partecipanti} & \multicolumn{3}{p{\useCaseMulticol} |}{Inizializzato da \textbf{\emph{Studente}.}} \\
		\hline
		\textbf{Condizioni d'ingresso} & \multicolumn{3}{p{\useCaseMulticol} |}{Lo \emph{Studente} seleziona un canale di discussione.} \\
		\hline
		\textbf{Flusso degli eventi} & \textbf{\#} & \textbf{\emph{Studente}} & \textbf{Sistema} \\
		\hline
		\textbf{} & \textbf{1} & Accede alla sezione \emph{menú} del canale di discussione; & \textbf{} \\
		\hline
		\textbf{} & \textbf{2} & \textbf{} & Mostra il \emph{menú}; \\
		\hline
		\textbf{} & \textbf{3} & Seleziona il pulsante per attivare/disattivare le notifiche della \emph{chat}; & \textbf{} \\
		\hline
		\textbf{Eccezioni} & \multicolumn{3}{p{\useCaseMulticol} |}{3.1  Connessione assente: viene visualizzato un messaggio di errore; \newline 3.2  Nessuna risposta dal \emph{server}: viene visualizzato un messaggio di errore;} \\
		\hline
		\textbf{Condizioni d'uscita} & \multicolumn{3}{p{\useCaseMulticol} |}{Il sistema cambia lo stato delle notifiche della \emph{chat}.} \\
		\hline
	\end{tabular}
	\caption{CUS9 - Gestisci notifiche chat}
\end{table}


\newpage
\paragraph{CUS10 - Selezione \emph{emoji} \\}
Lo \emph{Studente} che si trova all’interno di un canale di discussione ha la possibilitá di scegliere una o piú \emph{emoji} tra quelle disponibili. \\
\begin{table}
	\small % Dimensione testo piccola
	\label{CUS10 - Selezione emoji}
	
	\begin{tabular}{| p{\useCaseLeft} | p{\useCaseNum} | p{\useCaseTwoCol} | p{\useCaseTwoCol} |}
		\hline
		\textbf{Nome caso d'uso} & \multicolumn{3}{p{\useCaseMulticol} |}{\textbf{CUS10 - Selezione \emph{emoji}}} \\
		\hline
		\textbf{Attori partecipanti} & \multicolumn{3}{p{\useCaseMulticol} |}{Inizializzato da \textbf{\emph{Studente}.}} \\
		\hline
		\textbf{Condizioni d'ingresso} & \multicolumn{3}{p{\useCaseMulticol} |}{Lo \emph{Studente} seleziona un canale di discussione.} \\
		\hline
		\textbf{Flusso degli eventi} & \textbf{\#} & \textbf{\emph{Studente}} & \textbf{Sistema} \\
		\hline
		\textbf{} & \textbf{1} & Seleziona l’icona \emph{emoji}; & \textbf{} \\
		\hline
		\textbf{} & \textbf{2} & \textbf{} & Mostra una finestra con le \emph{emoji} disponibili; \\
		\hline
		\textbf{} & \textbf{3} & Seleziona una o piú \emph{emoji}; & \textbf{} \\
		\hline
		\textbf{} & \textbf{4} & \textbf{} & Il sistema inserisce l’\emph{emoji} nel messaggio; \\
		\hline
		\textbf{Eccezioni} & \multicolumn{3}{p{\useCaseMulticol} |}{ \textbf{} } \\
		\hline
		\textbf{Condizioni d'uscita} & \multicolumn{3}{p{\useCaseMulticol} |}{Lo \emph{Studente} visualizza il messaggio contenente le \emph{emoji} selezionate.} \\
		\hline
	\end{tabular}
	\caption{CUS10 - Selezione \emph{emoji}}
\end{table}


\newpage
\paragraph{CUS11 - Visualizza elenco membri chat \\}
Lo \emph{Studente} visualizza il numero dei partecipanti al canale di discussione e l’elenco contenente l’\emph{username} degli stessi. Nel caso in cui vi sia l’assenza di connessione viene visualizzata l’ultima copia salvata in locale dell’elenco dei partecipanti al canale qualora essa sia presente. \\
\begin{table}
	\small % Dimensione testo piccola
	\label{CUS11 - Visualizza elenco membri chat}
	
	\begin{tabular}{| p{\useCaseLeft} | p{\useCaseNum} | p{\useCaseTwoCol} | p{\useCaseTwoCol} |}
		\hline
		\textbf{Nome caso d'uso} & \multicolumn{3}{p{\useCaseMulticol} |}{\textbf{CUS11 - Visualizza elenco membri chat}} \\
		\hline
		\textbf{Attori partecipanti} & \multicolumn{3}{p{\useCaseMulticol} |}{Inizializzato da \textbf{\emph{Studente}.}} \\
		\hline
		\textbf{Condizioni d'ingresso} & \multicolumn{3}{p{\useCaseMulticol} |}{Lo \emph{Studente} seleziona un canale di discussione.} \\
		\hline
		\textbf{Flusso degli eventi} & \textbf{\#} & \textbf{\emph{Studente}} & \textbf{Sistema} \\
		\hline
		\textbf{} & \textbf{1} & Seleziona il nome del canale di discussione; & \textbf{} \\
		\hline
		\textbf{} & \textbf{2} & \textbf{} & Mostra il numero dei membri del canale di discussione e l’elenco dei loro nomi; \\
		\hline
		\textbf{Eccezioni} & \multicolumn{3}{p{\useCaseMulticol} |}{2.1 Nessuna risposta dal \emph{server}: viene visualizzato un messaggio di errore; \newline2.2 Connessione assente: il sistema restituisce l’ultima copia salvata in locale; \newline2.3 Copia non presente: viene visualizzato un messaggio di errore;} \\
		\hline
		\textbf{Condizioni d'uscita} & \multicolumn{3}{p{\useCaseMulticol} |}{Lo \emph{Studente} visualizza l’elenco dei membri del canale di discussione.} \\
		\hline
	\end{tabular}
	\caption{CUS11 - Visualizza elenco membri chat}
\end{table}

\paragraph{CUE1 - Connessione assente \\}
Lo \emph{Studente} effettua un’operazione che richiede connessione alla rete ma quest’ultima non è disponibile pertanto visualizza un messaggio d’errore. \\
\begin{table}
	\small % Dimensione testo piccola
	\label{CUE1 - Connessione assente}
	\begin{tabular}{| p{\useCaseLeft} | p{\useCaseNum} | p{\useCaseTwoCol} | p{\useCaseTwoCol} |}
		\hline
		\textbf{Nome caso d'uso} & \multicolumn{3}{p{\useCaseMulticol} |}{\textbf{CUE1 - Connessione assente}} \\
		\hline
		\textbf{Attori partecipanti} & \multicolumn{3}{p{\useCaseMulticol} |}{Inizializzato da \textbf{Sistema}, partecipa \textbf{\emph{Studente}.}} \\
		\hline
		\textbf{Condizioni d'ingresso} & \multicolumn{3}{p{\useCaseMulticol} |}{Lo \emph{Studente} effettua un’operazione che richiede connessione alla rete.} \\
		\hline
		\textbf{Flusso degli eventi} & \textbf{\#} & \textbf{\emph{Studente}} & \textbf{Sistema} \\
		\hline
		\textbf{} & \textbf{1} & \textbf{} & Mostra un messaggio d’errore; \\
		\hline
		\textbf{Eccezioni} & \multicolumn{3}{p{\useCaseMulticol} |}{ \textbf{} } \\
		\hline
		\textbf{Condizioni d'uscita} & \multicolumn{3}{p{\useCaseMulticol} |}{Lo \emph{Studente} visualizza il messaggio d’errore.} \\
		\hline
	\end{tabular}
	\caption{CUE1 - Connessione assente}
\end{table}


\newpage
\paragraph{CUE2 - Nessuna risposta dal sistema \\}
Lo \emph{Studente} effettua un’operazione che richiede una risposta dal Sistema, ma quest’ultimo non è in grado di soddisfare la richiesta pertanto visualizza un messaggio d’errore. \\
\begin{table}
	\small % Dimensione testo piccola
	\label{CUE2 - Nessuna risposta dal sistema}
	
	
	Lo \emph{Studente} effettua un’operazione che richiede una risposta dal Sistema, ma quest’ultimo non è in grado di soddisfare la richiesta pertanto visualizza un messaggio d’errore. \\
	
	\begin{tabular}{| p{\useCaseLeft} | p{\useCaseNum} | p{\useCaseTwoCol} | p{\useCaseTwoCol} |}
		\hline
		\textbf{Nome caso d'uso} & \multicolumn{3}{p{\useCaseMulticol} |}{\textbf{CUE2 - Nessuna risposta dal sistema}} \\
		\hline
		\textbf{Attori partecipanti} & \multicolumn{3}{p{\useCaseMulticol} |}{Inizializzato da \textbf{Sistema}, partecipa \textbf{\emph{Studente}.}} \\
		\hline
		\textbf{Condizioni d'ingresso} & \multicolumn{3}{p{\useCaseMulticol} |}{Il Sistema riceve un messaggio di errore.} \\
		\hline
		\textbf{Flusso degli eventi} & \textbf{\#} & \textbf{\emph{Studente}} & \textbf{Sistema} \\
		\hline
		\textbf{} & \textbf{1} & \textbf{} & Riscontra un errore e lo inoltra allo \emph{Studente}; \\
		\hline
		\textbf{Eccezioni} & \multicolumn{3}{p{\useCaseMulticol} |}{ \textbf{} } \\
		\hline
		\textbf{Condizioni d'uscita} & \multicolumn{3}{p{\useCaseMulticol} |}{Lo \emph{Studente} visualizza il messaggio d’errore.} \\
		\hline
	\end{tabular}
	\caption{CUE2 - Nessuna risposta dal sistema}
\end{table}

\paragraph{CUD1 - Creazione canale \\}
Il \emph{Docente}, che si trova all'interno di un cnale di discussione accede alla sezione "\emph{menù}" e crea un nuovo canale all'interno di quello attuale in uso, assengando ad esso un nome ed aggiungendo uno o più membri. \\
\begin{table}
	\small % Dimensione testo piccola
	\label{CUD1 - Creazione canale}
	\begin{tabular}{| p{\useCaseLeft} | p{\useCaseNum} | p{\useCaseTwoCol} | p{\useCaseTwoCol} |}
		\hline
		\textbf{Nome caso d'uso} & \multicolumn{3}{p{\useCaseMulticol} |}{\textbf{CUD1 - Creazione canale}} \\
		\hline
		\textbf{Attori partecipanti} & \multicolumn{3}{p{\useCaseMulticol} |}{Inizializzato da \textbf{\emph{Docente}.}} \\ 
		\hline
		\textbf{Condizioni d'ingresso} & \multicolumn{3}{p{\useCaseMulticol} |}{Il \emph{Docente} seleziona un canale di discussione.} \\
		\hline
		\textbf{Flusso degli eventi} & \textbf{\#} & \textbf{\emph{Docente}} & \textbf{Sistema} \\
		\hline
		\textbf{} & \textbf{1} & Accede alla sezione \emph{menù} del canale di discussione; & \textbf{} \\
		\hline
		\textbf{} & \textbf{2} & \textbf{} & Mostra il \emph{menù}; \\
		\hline
		\textbf{} & \textbf{3} & Seleziona la voce di creazione del nuovo canale; & \textbf{} \\
		\hline
		\textbf{} & \textbf{4} & \textbf{} & Chiede l'immissione del nome del nuovo canale; \\
		\hline
		\textbf{} & \textbf{5} & Inserisce il nome del nuovo canale; & \textbf{} \\
		\hline
		\textbf{} & \textbf{6} & \textbf{} & Chiede conferma del nome inserito; \\
		\hline
		\textbf{} & \textbf{7} & Conferma il nome inserito; & \textbf{} \\
		\hline
		\textbf{} & \textbf{8} & \textbf{} & Chiede l'identificativo dei membri da inserire all'interno del canale; \\
		\hline
		\textbf{} & \textbf{9} & Inserisce l'identificativo dei membri da aggiungere al canale; & \textbf{} \\
		\hline
		\textbf{} & \textbf{10} & \textbf{} & Chiede conferma degli identificativi inseriti; \\
		\hline
		\textbf{} & \textbf{11} & Conferma gli identificativi inseriti; & \textbf{} \\
		\hline
		\textbf{} & \textbf{10} & \textbf{} & Conferma la creazione del canale; \\
		\hline
		\textbf{Eccezioni} & \multicolumn{3}{p{\useCaseMulticol} |}{3.1 Connessione assente: il sistema non procede con la creazione canale; \newline 4.1 - 6.1 - 8.1 - 10.1 - 12.1 Nessuna risposta dal \emph{server}: viene visualizzato un messaggio di errore;} \\
		\hline
		\textbf{Condizioni d'uscita} & \multicolumn{3}{p{\useCaseMulticol} |}{Il \emph{Docente} visualizza il canale appena creato.} \\
		\hline
	\end{tabular}
	\caption{CUD1 - Creazione canale}
\end{table}


\newpage
\paragraph{CUD2 - Cancellazione canale \\}
Il \emph{Docente} che si trova all'interno di un canale di discussione accede alla sezione "\emph{menù}" e effettua la cancellazione di uno dei canali presenti all'interno della \emph{chat}. \\
\begin{table}
	\small % Dimensione testo piccola
	\label{CUD2 - Cancellazione canale}
	\begin{tabular}{| p{\useCaseLeft} | p{\useCaseNum} | p{\useCaseTwoCol} | p{\useCaseTwoCol} |}
		\hline
		\textbf{Nome caso d'uso} & \multicolumn{3}{p{\useCaseMulticol} |}{\textbf{CUD2 - Cancellazione canale}} \\
		\hline
		\textbf{Attori partecipanti} & \multicolumn{3}{p{\useCaseMulticol} |}{Inizializzato da \textbf{\emph{Docente}.}} \\ 
		\hline
		\textbf{Condizioni d'ingresso} & \multicolumn{3}{p{\useCaseMulticol} |}{Il \emph{Docente} seleziona un canale di discussione.} \\
		\hline
		\textbf{Flusso degli eventi} & \textbf{\#} & \textbf{\emph{Docente}} & \textbf{Sistema} \\
		\hline
		\textbf{} & \textbf{1} & Accede alla sezione \emph{menù} del canale di discussione; & \textbf{} \\
		\hline
		\textbf{} & \textbf{2} & \textbf{} & Mostra il \emph{menù}; \\
		\hline
		\textbf{} & \textbf{3} & Seleziona la voce di cancellazione del canale; & \textbf{} \\
		\hline
		\textbf{} & \textbf{4} & \textbf{} & Chiede conferma della cancellaizone; \\
		\hline
		\textbf{} & \textbf{5} & Conferma la scelta; & \textbf{} \\
		\hline
		\textbf{} & \textbf{6} & \textbf{} & procede alla cancellazione del canale; \\
		\hline
		\textbf{Eccezioni} & \multicolumn{3}{p{\useCaseMulticol} |}{3.1 Connessione assente: il canale non viene cancellato; \newline E' presente un solo canale di discussione: il canale non viene cancellato; \newline 4.1 - 6.1 Nessuna risposta dal \emph{server}: viene visualizzato un messaggio di errore;} \\
		\hline
		\textbf{Condizioni d'uscita} & \multicolumn{3}{p{\useCaseMulticol} |}{Il \emph{Docente} visualizza i canali attivi.} \\
		\hline
	\end{tabular}
	\caption{CUD2 - Cancellazione canale}
\end{table}


\newpage
\paragraph{CUD3 - Aggiungi membro ad un canale \\}
Il \emph{Docente} che si trova all'interno di un canale di discussione ha la possibilità di aggiungere un nuovo membro ad un canale attraverso il \emph{menù} di modifica che include l'opzione di aggiunta nuovo membro. \\
\begin{table}
	\small % Dimensione testo piccola
	\label{CUD3 - Aggiungi membro ad un canale}
		\begin{tabular}{| p{\useCaseLeft} | p{\useCaseNum} | p{\useCaseTwoCol} | p{\useCaseTwoCol} |}
		\hline
		\textbf{Nome caso d'uso} & \multicolumn{3}{p{\useCaseMulticol} |}{\textbf{CUD3 - Aggiungi membro ad un canale}} \\
		\hline
		\textbf{Attori partecipanti} & \multicolumn{3}{p{\useCaseMulticol} |}{Inizializzato da \textbf{\emph{Docente}.}} \\ 
		\hline
		\textbf{Condizioni d'ingresso} & \multicolumn{3}{p{\useCaseMulticol} |}{Il \emph{Docente} seleziona un canale di discussione.} \\
		\hline
		\textbf{Flusso degli eventi} & \textbf{\#} & \textbf{\emph{Docente}} & \textbf{Sistema} \\
		\hline
		\textbf{} & \textbf{1} & Accede alla sezione \emph{menù} del canale di discussione; & \textbf{} \\
		\hline
		\textbf{} & \textbf{2} & \textbf{} & Mostra il \emph{menù}; \\
		\hline
		\textbf{} & \textbf{3} & Seleziona la voce aggiungi nuovo membro al canale; & \textbf{} \\
		\hline
		\textbf{} & \textbf{4} & \textbf{} & Chiede l'identificativo del membro da aggiungere al canale; \\
		\hline
		\textbf{} & \textbf{5} & Inserisce l'identificativo del membro da aggiungere; & \textbf{} \\
		\hline
		\textbf{} & \textbf{6} & \textbf{} & Chiede conferma dell'identificativo inserito; \\
		\hline
		\textbf{} & \textbf{7} & Conferma i dati inseriti; & \textbf{} \\
		\hline
		\textbf{} & \textbf{8} & \textbf{} & Conferma l'aggiunta del nuovo \emph{Docente}; \\
		\hline
		\textbf{Eccezioni} & \multicolumn{3}{p{\useCaseMulticol} |}{3.1 Connessione assente: il membro non viene aggiunto; \newline 4.1 - 6.1 - 8.1 Nessuna risposta dal \emph{server}: viene visualizzato un messaggio di errore;} \\
		\hline
		\textbf{Condizioni d'uscita} & \multicolumn{3}{p{\useCaseMulticol} |}{Il \emph{Docente} visualizza l'elenco aggiornato dei membri assegnati al canale.} \\
		\hline
	\end{tabular}
	\caption{CUD3 - Aggiungi membro ad un canale}
\end{table}


\newpage
\paragraph{CUD4 - Rimuovi membro ad un canale \\}
Il \emph{Docente} che si trova all'interno di un canale di discussione ha la possibilità di rimuovere un \emph{Docente} o uno \emph{Studente} da un canale attraverso il \emph{menù} di modifica che include l'opzione di rimozione \emph{Docente} o \emph{Studente}. \\
\begin{table}
	\small % Dimensione testo piccola
	\label{CUD4 - Rimuovi membro ad un canale}
	\begin{tabular}{| p{\useCaseLeft} | p{\useCaseNum} | p{\useCaseTwoCol} | p{\useCaseTwoCol} |}
		\hline
		\textbf{Nome caso d'uso} & \multicolumn{3}{p{\useCaseMulticol} |}{\textbf{CUD4 - Rimuovi membro ad un canale}} \\
		\hline
		\textbf{Attori partecipanti} & \multicolumn{3}{p{\useCaseMulticol} |}{Inizializzato da \textbf{\emph{Docente}.}} \\ 
		\hline
		\textbf{Condizioni d'ingresso} & \multicolumn{3}{p{\useCaseMulticol} |}{Il \emph{Docente} seleziona un canale di discussione.} \\
		\hline
		\textbf{Flusso degli eventi} & \textbf{\#} & \textbf{\emph{Docente}} & \textbf{Sistema} \\
		\hline
		\textbf{} & \textbf{1} & Accede alla sezione \emph{menù} del canale di discussione; & \textbf{} \\
		\hline
		\textbf{} & \textbf{2} & \textbf{} & Mostra il \emph{menù}; \\
		\hline
		\textbf{} & \textbf{3} & Seleziona la voce rimuovi membro dal canale; & \textbf{} \\
		\hline
		\textbf{} & \textbf{4} & \textbf{} & Chiede l'immissione dell'identificativo del membro da rimuovere; \\
		\hline
		\textbf{} & \textbf{5} & Inserisce l'identificativo del membro; & \textbf{} \\
		\hline
		\textbf{} & \textbf{6} & \textbf{} & Chiede conferma dell'identificativo inserito; \\
		\hline
		\textbf{} & \textbf{7} & Conferma l'identificativo inserito; & \textbf{} \\
		\hline
		\textbf{} & \textbf{8} & \textbf{} & Conferma la rimozione del membro dal canale; \\
		\hline
		\textbf{Eccezioni} & \multicolumn{3}{p{\useCaseMulticol} |}{3.1 Connessione assente: non viene eliminato il membro dal canale; \newline 4.1 - 6.1 - 8.1 Nessuna risposta dal \emph{server}: viene visualizzato un messaggio di errore;} \\
		\hline
		\textbf{Condizioni d'uscita} & \multicolumn{3}{p{\useCaseMulticol} |}{Il \emph{Docente} visualizza l'elenco aggiornato dei membri assegnati al canale.} \\
		\hline
	\end{tabular}
	\caption{CUD4 - Rimuovi membro ad un canale}
\end{table}


\newpage
\paragraph{CUD5 - Blocca Studente \\}
Il \emph{Docente} ha la possibilità di silenziare uno \emph{Studente}, nel caso in cui ritenga inappropriati alcuni interventi nel canale di comunicazione. Per silenziare si intende impedire l'invio di nuovi messaggi ad un determinato \emph{Studente}. \\
\begin{table}
	\small % Dimensione testo piccola
	\label{CUD5 - Blocca Studente}
	\begin{tabular}{| p{\useCaseLeft} | p{\useCaseNum} | p{\useCaseTwoCol} | p{\useCaseTwoCol} |}
		\hline
		\textbf{Nome caso d'uso} & \multicolumn{3}{p{\useCaseMulticol} |}{\textbf{CUD5 - Blocca Studente}} \\
		\hline
		\textbf{Attori partecipanti} & \multicolumn{3}{p{\useCaseMulticol} |}{Inizializzato da \textbf{\emph{Docente}.}} \\ 
		\hline
		\textbf{Condizioni d'ingresso} & \multicolumn{3}{p{\useCaseMulticol} |}{Il \emph{Docente} seleziona un canale di discussione.} \\
		\hline
		\textbf{Flusso degli eventi} & \textbf{\#} & \textbf{\emph{Docente}} & \textbf{Sistema} \\
		\hline
		\textbf{} & \textbf{1} & Seleziona un messaggio; & \textbf{} \\
		\hline
		\textbf{} & \textbf{2} & \textbf{} & Mostra l'opzione di blocco dello \emph{Studente} associato al messaggio; \\
		\hline
		\textbf{} & \textbf{3} & Conferma il blocco dello \emph{Studente}; & \textbf{} \\
		\hline
		\textbf{} & \textbf{4} & \textbf{} & Blocca lo \emph{Studente} e mostra un messaggio di conferma; \\
		\hline
		\textbf{} & \textbf{1.1} & Seleziona il nome del canale di discussione; & \textbf{} \\
		\hline
		\textbf{} & \textbf{1.2} & \textbf{} & Mostra il numero dei membri della \emph{chat} e l'elenco dei loro nomi; \\
		\hline
		\textbf{} & \textbf{1.3} & Seleziona uno \emph{Studente}; & \textbf{} \\
		\hline
		\textbf{} & \textbf{2.1} & \textbf{} & Mostra l'opzione di blocco dello \emph{Studente}; \\
		\hline
		\textbf{Eccezioni} & \multicolumn{3}{p{\useCaseMulticol} |}{3.1 Il \emph{Docente} non conferma il blocco dello \emph{Studente}: lo \emph{Studente} non viene bloccato \newline Connessione assente: lo \emph{Studente} non viene bloccato; \newline 4.2 Nessuna risposta dal \emph{server}: viene visualizzato un messaggio di errore;} \\
		\hline
		\textbf{Condizioni d'uscita} & \multicolumn{3}{p{\useCaseMulticol} |}{Il sistema blocca lo \emph{Studente scelto}.} \\
		\hline
	\end{tabular}
	\caption{CUD5 - Blocca Studente}
\end{table}


\newpage
\paragraph{CUD6 - Sblocca Studente \\}
Il \emph{Docente} Il Docente ha la possibilità di sbloccare uno Studente precedentemente bloccato  nel caso in cui ritenga opportuno farlo, così da poter permettere nuovamente allo studente di inviare messaggi nel canale di discussione.    \\
\begin{table}
	\small % Dimensione testo piccola
	\label{CUD6 - Sblocca Studentee}
	\begin{tabular}{| p{\useCaseLeft} | p{\useCaseNum} | p{\useCaseTwoCol} | p{\useCaseTwoCol} |}
		\hline
		\textbf{Nome caso d'uso} & \multicolumn{3}{p{\useCaseMulticol} |}{\textbf{CUD6 - Sblocca Studente}} \\
		\hline
		\textbf{Attori partecipanti} & \multicolumn{3}{p{\useCaseMulticol} |}{Inizializzato da \textbf{\emph{Docente}.}} \\ 
		\hline
		\textbf{Condizioni d'ingresso} & \multicolumn{3}{p{\useCaseMulticol} |}{Il \emph{Docente} seleziona un canale di discussione.} \\
		\hline
		\textbf{Flusso degli eventi} & \textbf{\#} & \textbf{\emph{Docente}} & \textbf{Sistema} \\
		\hline
		\textbf{} & \textbf{1} & Seleziona un messaggio; & \textbf{} \\
		\hline
		\textbf{} & \textbf{2} & \textbf{} & Mostra l'opzione di sblocco dello \emph{Studente} associato al messaggio; \\
		\hline
		\textbf{} & \textbf{3} & Conferma lo sblocco dello \emph{Studente}; & \textbf{} \\
		\hline
		\textbf{} & \textbf{4} & \textbf{} & Sblocca lo \emph{Studente} e mostra un messaggio di conferma; \\
		\hline
		\textbf{} & \textbf{1.1} & Seleziona il nome del canale di discussione; & \textbf{} \\
		\hline
		\textbf{} & \textbf{1.2} & \textbf{} & Mostra il numero dei membri della \emph{chat} e l'elenco dei loro nomi; \\
		\hline
		\textbf{} & \textbf{1.3} & Seleziona uno \emph{Studente}; & \textbf{} \\
		\hline
		\textbf{} & \textbf{2.1} & \textbf{} & Mostra l'opzione di sblocco dello \emph{Studente}; \\
		\hline
		\textbf{Eccezioni} & \multicolumn{3}{p{\useCaseMulticol} |}{3.1 Il \emph{Docente} non conferma lo sblocco dello \emph{Studente}: lo \emph{Studente} non viene sbloccato \newline Connessione assente: lo \emph{Studente} non viene sbloccato; \newline 4.2 Nessuna risposta dal \emph{server}: viene visualizzato un messaggio di errore;} \\
		\hline
		\textbf{Condizioni d'uscita} & \multicolumn{3}{p{\useCaseMulticol} |}{Il sistema sblocca lo \emph{Studente scelto}.} \\
		\hline
	\end{tabular}
	\caption{CUD6 - Sblocca Studente}
\end{table}


\newpage
\paragraph{CUP1 - Login \\}
L'\emph{Amministratore} vuole accedere al pannello di amministrazione. L’accesso avviene mediante l’inserimento di una \emph{username} e \emph{password}. Il \emph{login} dovrà verificare in modo sicuro l’identità di chi sta effettuando l’accesso.\\
\begin{table} 
	\small % Dimensione testo piccola
	\label{CUP1-Login}
	
	\begin{tabular}{| p{\useCaseLeft} | p{\useCaseNum} | p{\useCaseTwoCol} | p{\useCaseTwoCol} |}
		\hline
		\textbf{Nome caso d'uso} & \multicolumn{3}{p{\useCaseMulticol} |}{\textbf{CUP1 - \emph{Login}}} \\
		\hline
		\textbf{Attori partecipanti} & \multicolumn{3}{p{\useCaseMulticol} |}{Inizializzato da \textbf{\emph{Amministratore}}.} \\
		\hline
		\textbf{Condizioni d'ingresso} & \multicolumn{3}{p{\useCaseMulticol} |}{L'\emph{Amministratore} ha raggiunto la schermata di \emph{login}.} \\
		\hline
		\textbf{Flusso degli eventi} & \textbf{\#} & \textbf{\emph{Amministratore}} & \textbf{Sistema} \\
		\hline
		\textbf{} & \textbf{1} & Seleziona la voce \emph{login}; & \textbf{}  \\
		\hline
		\textbf{} & \textbf{2} &  \textbf{} & Propone una schermata per l'inserimento dei dati necessari per il \emph{login}, \emph{username} e \emph{password} dell'\emph{Amministratore};\\
		\hline
		\textbf{} & \textbf{3} & Inserisce i dati e sottomette la richiesta; & \textbf{}  \\
		\hline
		\textbf{} & \textbf{4} &  \textbf{} & Controlla che siano stati inseriti i campi e avvia le operazioni di visualizzazione;\\
		\hline
		
		\textbf{Eccezioni} & \multicolumn{3}{p{\useCaseMulticol} |}{1.1 Connessione assente: non viene visualizzata la schermata di \emph{login};
			\newline 2.1 - 4.1 Errore di sistema: non viene effettuato il \emph{login};
			\newline 3.1 Uno o entrambi i campi sono vuoti: non viene effettutato il \emph{login} e viene evidenziato il campo vuoto con un messaggio di errore;\newline 3.2 Le credenziali inserite non sono valide (una o entrambe): non viene effettuato il \emph{login} e viene visualizzato un messaggio che notifica l'errore di inserimento delle credenziali;} \\
		\hline
		\textbf{Condizioni d'uscita} & \multicolumn{3}{p{\useCaseMulticol} |}{L'\emph{Amministratore} visualizza la schermata \emph{home}.} \\
		\hline
	\end{tabular}
	\caption{\textbf{CUP1 - \emph{Login}}}
\end{table}

\newpage
\paragraph{CUP2 - Logout \\}
L'\emph{Amministratore} vuole uscire dal sistema, chiudendo la sessione in modo sicuro.\\
\begin{table}
	%\normalsize % Dimensione testo normale
	\small % Dimensione testo piccola
	\label{CUP2-Logout}
	\begin{tabular}{| p{\useCaseLeft} | p{\useCaseNum} | p{\useCaseTwoCol} | p{\useCaseTwoCol} |}
		\hline
		\textbf{Nome caso d'uso} & \multicolumn{3}{p{\useCaseMulticol} |}{\textbf{CUP2 - \emph{Logout}}} \\
		\hline
		\textbf{Attori partecipanti} & \multicolumn{3}{p{\useCaseMulticol} |}{Inizializzato da \textbf{\emph{Amministratore}}.} \\
		\hline
		\textbf{Condizioni d'ingresso} & \multicolumn{3}{p{\useCaseMulticol} |}{L'\emph{Amministratore} ha effettuato il \emph{login}.} \\
		\hline
		\textbf{Flusso degli eventi} & \textbf{\#} & \textbf{\emph{Amministratore}} & \textbf{Sistema} \\
		\hline
		\textbf{} & \textbf{1} & Seleziona la voce \emph{logout}; & \textbf{}  \\
		\hline
		\textbf{} & \textbf{2} &  \textbf{} & Chiude la sessione disconnettendo l'\emph{Amministratore};\\
		\hline
		\textbf{Eccezioni} & \multicolumn{3}{p{\useCaseMulticol} |}{1.1 Connessione assente: non viene effettuato il \emph{logout};
			\newline 2.1 Chiusura della sessione non corretta: non viene effettuato il \emph{logout};} \\
		\hline
		\textbf{Condizioni d'uscita} & \multicolumn{3}{p{\useCaseMulticol} |}{L'\emph{Amministratore} visualizza la schermata di \emph{login}.} \\
		\hline
	\end{tabular}
	\caption{\textbf{CUP2 - \emph{Logout}}}
\end{table}

\newpage
\paragraph{CUP3 - Ricerca chat \\}
L'\emph{Amministratore}, seleziona una specifica categoria di \emph{chat} e visualizza una schermata relativa alla ricerca, all’interno della quale è possibile cercare una specifica \emph{chat} o un gruppo di \emph{chat}. Tale ricerca, può essere effettuata mediante l’inserimento del nome della \emph{chat} da ricercare, o attraverso l’utilizzo di filtri.\\
\begin{table}
	%\normalsize % Dimensione testo normale
	\small % Dimensione testo piccola
	\label{CUP3- Ricerca chat}
	\begin{tabular}{| p{\useCaseLeft} | p{\useCaseNum} | p{\useCaseTwoCol} | p{\useCaseTwoCol} |}
		\hline
		\textbf{Nome caso d'uso} & \multicolumn{3}{p{\useCaseMulticol} |}{\textbf{CUP3 - Ricerca \emph{chat}}} \\
		\hline
		\textbf{Attori partecipanti} & \multicolumn{3}{p{\useCaseMulticol} |}{Inizializzato da \textbf{\emph{Amministratore}}.} \\
		\hline
		\textbf{Condizioni d'ingresso} & \multicolumn{3}{p{\useCaseMulticol} |}{L'\emph{Amministratore} visualizza la schermata per effettuare una ricerca;} \\
		\hline
		\textbf{Flusso degli eventi} & \textbf{\#} & \textbf{\emph{Amministratore}} & \textbf{Sistema} \\
		\hline
		\textbf{} & \textbf{1} & Seleziona la voce "Ricerca"; & \textbf{}  \\
		\hline
		\textbf{} & \textbf{2} &  \textbf{} & Chiede il testo da cercare;\\
		\hline
		\textbf{} & \textbf{3} & Inserisce il testo da cercare; & \textbf{}  \\
		\hline
		\textbf{} & \textbf{4} &  \textbf{} & Effettua la ricerca e mostra la \emph{chat} desiderata;\\
		\hline
		\textbf{} & \textbf{1.1} & Seleziona il filtro "Dipartimento"; & \textbf{}  \\
		\hline
		\textbf{} & \textbf{2.1} &  \textbf{} & Mostra i dipartimenti;\\
		\hline
		\textbf{} & \textbf{3.1} & Seleziona un dipartimento e procede col selezionare il secondo filtro “Corsi di laurea”;  & \textbf{}  \\
		\hline
		\textbf{} & \textbf{4.1} &  \textbf{} & Mostra i corsi di laurea del  dipartimento precedentemente selezionato;\\
		\hline
		\textbf{} & \textbf{5.1} & Seleziona un corso di laurea e procede col selezionare il terzo filtro “Anno corso”; & \textbf{}  \\
		\hline
		\textbf{} & \textbf{6.1} &  \textbf{} & Mostra la coorte degli anni che individuano un anno accademico di un corso di laurea; \\
		\hline
		\textbf{} & \textbf{7.1} & Seleziona la coorte di un anno accademico e procede con la ricerca;  & \textbf{}  \\
		\hline
		
		\textbf{Eccezioni} & \multicolumn{3}{p{\useCaseMulticol} |}{1.1 Connessione assente: viene visualizzato un messaggio di errore;
			\newline 2.1 - 2.1.1 - 4.1.1 - 6.1.1 Errore di sistema: viene visualizzato un messaggio di errore;
			\newline 4.1 \emph{Chat} non trovata: viene visualizzato un messaggio di errore;} \\
		\hline
		\textbf{Condizioni d'uscita} & \multicolumn{3}{p{\useCaseMulticol} |}{L'\emph{Amministratore} visualizza correttamente la \emph{chat} ricercata, o l'elenco delle \emph{chat} inerenti ai filtri applicati.} \\
		\hline
	\end{tabular}
	\caption{\textbf{CUP3 - Ricerca \emph{chat}}}
\end{table}

\newpage
\paragraph{CUP4 - Visualizza lista chat \\}
L'\emph{Amministratore} vuole selezionare una tipologia di \emph{chat} da visualizzare.
E’ possibile visualizzare la lista delle \emph{chat} studenti e delle \emph{chat} dei corsi, queste ultime suddivise a loro volta in \emph{chat} attive e non attive. La suddivisione garantisce una corretta caratterizzazione e organizzazione delle \emph{chat}.\\
\begin{table}
	%\normalsize % Dimensione testo normale
	\small % Dimensione testo piccola
	\label{CUP4 - Visualizza lista chat}
	\begin{tabular}{| p{\useCaseLeft} | p{\useCaseNum} | p{\useCaseTwoCol} | p{\useCaseTwoCol} |}
		\hline
		\textbf{Nome caso d'uso} & \multicolumn{3}{p{\useCaseMulticol} |}{\textbf{CUP4 - Visualizza lista \emph{chat}}} \\
		\hline
		\textbf{Attori partecipanti} & \multicolumn{3}{p{\useCaseMulticol} |}{Inizializzato da \textbf{\emph{Amministratore}}.} \\
		\hline
		\textbf{Condizioni d'ingresso} & \multicolumn{3}{p{\useCaseMulticol} |}{L'\emph{Amministratore} effettua il \emph{login} e si trova nella schermata principale.} \\
		\hline
		\textbf{Flusso degli eventi} & \textbf{\#} & \textbf{\emph{Amministratore}} & \textbf{Sistema} \\
		\hline
		\textbf{} & \textbf{1} & Seleziona come categoria di \emph{chat} da visualizzare le \emph{chat} degli studenti; & \textbf{}  \\
		\hline
		\textbf{} & \textbf{2} & Effettua la ricerca come da caso d'uso CUP3 Ricerca \emph{chat}; & \textbf{}  \\
		\hline
		\textbf{} & \textbf{3} &  \textbf{} & Mostra l'elenco delle \emph{chat} degli studenti;\\
		\hline
		\textbf{} & \textbf{1.1} & Seleziona come categoria di \emph{chat} da visualizzare le chat relative ai corsi; & \textbf{}  \\
		\hline
		\textbf{} & \textbf{2.1} & Effettua la ricerca come da caso d'uso CUP3 Ricerca \emph{chat}; & \textbf{}  \\
		\hline
		\textbf{} & \textbf{3.1} &  \textbf{} & Mostra la schermata con l'elenco delle \emph{chat} attive;\\
		\hline
		\textbf{} & \textbf{1.1.1} & Seleziona la schermata con l'elenco delle \emph{chat} non attive; & \textbf{}  \\
		\hline
		\textbf{} & \textbf{3.1.1} &  \textbf{} & Mostra l'elenco delle \emph{chat} non attive; \\
		
		\hline
		
		\textbf{Eccezioni} & \multicolumn{3}{p{\useCaseMulticol} |}{1.1 Connessione assente: non viene visualizzata la lista delle \emph{chat};
			\newline 3.1 - 3.1.1 - 3.1.1.1 Errore di sistema: viene visualizzato un messaggio di errore;
		} \\
		\hline
		\textbf{Condizioni d'uscita} & \multicolumn{3}{p{\useCaseMulticol} |}{L'\emph{Amministratore} visualizza la lista delle \emph{chat} ricercate.} \\
		\hline
	\end{tabular}
	\caption{\textbf{CUP4 - Visualizza lista \emph{chat}}}
	\end{table}


\newpage
\paragraph{CUP5 - Abilita chat \\}
L'\emph{Amministratore} visualizza l’elenco delle \emph{chat} di corso non attive e a seguito di una richiesta abilita una specifica \emph{chat}.\\
\begin{table}
	%\normalsize % Dimensione testo normale
	\small % Dimensione testo piccola
	\label{CUP5-Abilita chat}
	
	\begin{tabular}{| p{\useCaseLeft} | p{\useCaseNum} | p{\useCaseTwoCol} | p{\useCaseTwoCol} |}
		\hline
		\textbf{Nome caso d'uso} & \multicolumn{3}{p{\useCaseMulticol} |}{\textbf{CUP5 - Abilita \emph{chat}}} \\
		\hline
		\textbf{Attori partecipanti} & \multicolumn{3}{p{\useCaseMulticol} |}{Inizializzato da \textbf{\emph{Amministratore}}.} \\
		\hline
		\textbf{Condizioni d'ingresso} & \multicolumn{3}{p{\useCaseMulticol} |}{L'\emph{Amministratore} seleziona la categoria corsi, all’interno della quale visualizza le \emph{chat} abilitate.} \\
		\hline
		\textbf{Flusso degli eventi} & \textbf{\#} & \textbf{\emph{Amministratore}} & \textbf{Sistema} \\
		\hline
		\textbf{} & \textbf{1} & Seleziona una \emph{chat} e accede alla sezione abilita \emph{chat}; & \textbf{}  \\
		\hline
		\textbf{} & \textbf{2} &  \textbf{} & Mostra un pannello in cui chiede conferma per abilitare la \emph{chat} selezionata;\\
		\hline
		\textbf{} & \textbf{3} & Conferma l'abilitazione della \emph{chat}; & \textbf{}  \\
		\hline
		\textbf{} & \textbf{4} &  \textbf{} & Abilita la \emph{chat} e inserisce la \emph{chat} attivata all'interno dell'elenco delle \emph{chat} abilitate;\\
		\hline
		
		\textbf{Eccezioni} & \multicolumn{3}{p{\useCaseMulticol} |}{1.1 Connessione assente: viene visualizzato un messaggio di errore;
			\newline 2.1 - 4.1 Errore di sistema: viene visualizzato un messaggio di errore;
		} \\
		\hline
		\textbf{Condizioni d'uscita} & \multicolumn{3}{p{\useCaseMulticol} |}{L'\emph{Amministratore} visualizza la schermata \emph{chat} attive.} \\
		\hline
	\end{tabular}
	\caption{\textbf{CUP5 - Abilita \emph{chat}}}
\end{table}

\newpage
\paragraph{CUP6 - Disabilita chat \\}
L'\emph{Amministratore} visualizza l’elenco delle \emph{chat} di corso attive e a seguito di una richiesta disabilita una specifica \emph{chat}.\\
\begin{table}
	%\normalsize % Dimensione testo normale
	\small % Dimensione testo piccola
	\label{CUP6-Disabilita chat}
	
	\begin{tabular}{| p{\useCaseLeft} | p{\useCaseNum} | p{\useCaseTwoCol} | p{\useCaseTwoCol} |}
		\hline
		\textbf{Nome caso d'uso} & \multicolumn{3}{p{\useCaseMulticol} |}{\textbf{CUP6 - Disabilita \emph{chat}}} \\
		\hline
		\textbf{Attori partecipanti} & \multicolumn{3}{p{\useCaseMulticol} |}{Inizializzato da \textbf{\emph{Amministratore}}.} \\
		\hline
		\textbf{Condizioni d'ingresso} & \multicolumn{3}{p{\useCaseMulticol} |}{L'\emph{Amministratore} seleziona la categoria corsi, all'interno della quale visualizza le \emph{chat} abilitate.} \\
		\hline
		\textbf{Flusso degli eventi} & \textbf{\#} & \textbf{\emph{Amministratore}} & \textbf{Sistema} \\
		\hline
		\textbf{} & \textbf{1} & Seleziona una \emph{chat} e seleziona la voce disabilita \emph{chat};& \textbf{} \\
		\hline
		\textbf{} & \textbf{2} & \textbf{} & Mostra un pannello in cui chiede conferma per disabilitare la \emph{chat} selezionata; \\
		\hline
		\textbf{} & \textbf{3} & Conferma la disabilitazione della \emph{chat};& \textbf{} \\
		\hline
		\textbf{} & \textbf{4} & \textbf{} & Disabilita la \emph{chat} e inserisce la \emph{chat} disattivata all'interno dell'elenco delle \emph{chat} non abilitate; \\
		\hline
		
		\textbf{Eccezioni} & \multicolumn{3}{p{\useCaseMulticol} |}{1.1 Connessione assente: viene visualizzato un messaggio di errore;\newline 2.1 - 4.1 Errore di sistema: viene visualizzato un messaggio di errore;} \\
		\hline
		\textbf{Condizioni d'uscita} & \multicolumn{3}{p{\useCaseMulticol} |}{L'\emph{Amministratore} visualizza la schermata \emph{chat} non attive.} \\
		\hline
	\end{tabular}
	\caption{\textbf{CUP6 - Disabilita \emph{chat}}}
\end{table}

\newpage
\paragraph{CUP7 - Visualizza canale \\}
L'\emph{Amministratore} seleziona una specifica chat da visualizzare.
In quest’ultima, visualizza i canali in cui è suddivisa e, selezionato un opportuno canale, ne visualizza i messaggi in esso contenuti.\\
\begin{table}
	%\normalsize % Dimensione testo normale
	\small % Dimensione testo piccola
	\label{CUP7- Visualizza canale}
	\begin{tabular}{| p{\useCaseLeft} | p{\useCaseNum} | p{\useCaseTwoCol} | p{\useCaseTwoCol} |}
		\hline
		\textbf{Nome caso d'uso} & \multicolumn{3}{p{\useCaseMulticol} |}{\textbf{CUP7 - Visualizza canale}} \\
		\hline
		\textbf{Attori partecipanti} & \multicolumn{3}{p{\useCaseMulticol} |}{Inizializzato da \textbf{\emph{Amministratore}}.} \\
		\hline
		\textbf{Condizioni d'ingresso} & \multicolumn{3}{p{\useCaseMulticol} |}{L'\emph{Amministratore} seleziona una specifica categoria.} \\
		\hline
		\textbf{Flusso degli eventi} & \textbf{\#} & \textbf{\emph{Amministratore}} & \textbf{Sistema} \\
		\hline
		\textbf{} & \textbf{1} & Seleziona la \emph{chat} da visualizzare; & \textbf{} \\
		\hline
		\textbf{} & \textbf{2} & \textbf{} & Mostra la relativa \emph{chat} e i canali in essa contenuti; \\
		\hline
		\textbf{} & \textbf{3} & Seleziona uno specifico canale;& \textbf{} \\
		\hline
		\textbf{} & \textbf{4} & \textbf{} & Mostra la conversazione inerente al canale selezionato; \\
		\hline
		
		\textbf{Eccezioni} & \multicolumn{3}{p{\useCaseMulticol} |}{1.1 Connessione assente: viene visualizzato un messaggio di errore;\newline 2.1 - 4.1 Errore di sistema: viene visualizzato un messaggio di errore;} \\
		\hline
		\textbf{Condizioni d'uscita} & \multicolumn{3}{p{\useCaseMulticol} |}{L'\emph{Amministratore} visualizza il canale selezionato.} \\
		\hline
	\end{tabular}
	\caption{\textbf{CUP7 - Visualizza canale}}
\end{table}

\newpage
\paragraph{CUP8 - Aggiungi canale \\}
L'\emph{Amministratore} vuole creare un nuovo canale all’interno della \emph{chat} al fine di rendere la conversazione quanto più organizzata possibile e di suddividere eventuali moduli e partizionamenti di un corso. Ha quindi la possibilità di aggiungere utenti al canale della \emph{chat} e di scegliere il nome del nuovo canale.\\
\begin{table}
	
	\small % Dimensione testo piccola
	
	\label{CUP8-Aggiungi canale}
	\begin{tabular}{| p{\useCaseLeft} | p{\useCaseNum} | p{\useCaseTwoCol} | p{\useCaseTwoCol} |}
		\hline
		\textbf{Nome caso d'uso} & \multicolumn{3}{p{\useCaseMulticol} |}{\textbf{CUP8 - Aggiungi canale}} \\
		\hline
		\textbf{Attori partecipanti} & \multicolumn{3}{p{\useCaseMulticol} |}{Inizializzato da \textbf{\emph{Amministratore}}.} \\
		\hline
		\textbf{Condizioni d'ingresso} & \multicolumn{3}{p{\useCaseMulticol} |}{L'\emph{Amministratore} seleziona una \emph{chat} attiva.} \\
		\hline
		\textbf{Flusso degli eventi} & \textbf{\#} & \textbf{\emph{Amministratore}} & \textbf{Sistema} \\
		\hline
		\textbf{} & \textbf{1} & Seleziona la voce "Gestione canali"; & \textbf{}  \\
		\hline
		\textbf{} & \textbf{2} &  \textbf{} & Mostra le opzioni per i canali;\\
		\hline
		\textbf{} & \textbf{3} &Seleziona la voce "Crea nuovo canale"; & \textbf{}  \\
		\hline
		\textbf{} & \textbf{4} &  \textbf{} & Mostra le opzioni per il nuovo canale;\\
		\hline
		
		\textbf{} & \textbf{5} & Inserisce il nome del canale e procede; & \textbf{}  \\
		\hline
		\textbf{} & \textbf{6} &  \textbf{} & Mostra la schermata per procedere all'inserimento degli utenti del canale;\\
		\hline
		\textbf{} & \textbf{7} &Inserisce gli utenti che ne fanno parte e procedere selezionando la voce "Aggiungi utenti"; & \textbf{}  \\
		\hline
		\textbf{} & \textbf{8} &  \textbf{} & Mostra la schermata per confermare la creazione del canale;\\
		\hline
		\textbf{} & \textbf{9} & Procede selezionando "Crea canale"; & \textbf{}  \\
		\hline
		\textbf{} & \textbf{10} &  \textbf{} & Crea e mostra il canale all'interno della \emph{chat};\\
		\hline		
		\textbf{Eccezioni} & \multicolumn{3}{p{\useCaseMulticol} |}{1.1 Connessione assente: viene visualizato un messaggio di errore;
			\newline 2.1 - 4.1 - 6.1 - 8.1 Errore di sistema: viene visualizzato un messaggio di errore;
			\newline 6.1 Il canale non viene creato: viene visualizzato un messaggio di errore;} \\
		\hline
		\textbf{Condizioni d'uscita} & \multicolumn{3}{p{\useCaseMulticol} |}{L'\emph{Amministratore} visualizza un messaggio di conferma della creazione del nuovo canale.} \\
		\hline
	\end{tabular}
	\caption{\textbf{CUP8 - Aggiungi canale}}
\end{table}

\newpage
\paragraph{CUP9 - Cancella canale \\}
L'\emph{Amministratore} potrà eliminare un canale esistente qualora ne venga richiesta l’eliminazione.\\
\begin{table}
	%\normalsize % Dimensione testo normale
	\small % Dimensione testo piccola
	\label{CUP9-Cancella canale}
	\begin{tabular}{| p{\useCaseLeft} | p{\useCaseNum} | p{\useCaseTwoCol} | p{\useCaseTwoCol} |}
		\hline
		\textbf{Nome caso d'uso} & \multicolumn{3}{p{\useCaseMulticol} |}{\textbf{CUP9 - Cancella canale}} \\
		\hline
		\textbf{Attori partecipanti} & \multicolumn{3}{p{\useCaseMulticol} |}{Inizializzato da \textbf{\emph{Amministratore}}.} \\
		\hline
		\textbf{Condizioni d'ingresso} & \multicolumn{3}{p{\useCaseMulticol} |}{L'\emph{Amministratore} sleziona una \emph{chat} attiva.} \\
		\hline
		\textbf{Flusso degli eventi} & \textbf{\#} & \textbf{\emph{Amministratore}} & \textbf{Sistema} \\
		\hline
		\textbf{} & \textbf{1} & Seleziona la voce "Gestione canali"; & \textbf{}  \\
		\hline
		\textbf{} & \textbf{2} &  \textbf{} & Mostra le opzioni per i canali;\\
		\hline
		\textbf{} & \textbf{3} &Seleziona la voce "Elimina canale"; & \textbf{}  \\
		\hline
		\textbf{} & \textbf{4} &  \textbf{} & Mostra lista dei canali della \emph{chat};\\
		\hline
		
		\textbf{} & \textbf{5} & Seleziona il canale da eliminare e procede all'eliminazione; & \textbf{}  \\
		\hline
		\textbf{} & \textbf{6} &  \textbf{} & Il sistema elimina il canale selezionato;\\
		\hline
		
		\textbf{Eccezioni} & \multicolumn{3}{p{\useCaseMulticol} |}{1.1 Connessione assente: viene visualizato un messaggio di errore;
			\newline 2.1 - 4.1 - 6.1  Errore di sistema: viene visualizzato un messaggio di errore;
			\newline 4.1 Non ci sono canali nella \emph{chat}: viene visualizzato un messaggio di errore;} \\
		\hline
		\textbf{Condizioni d'uscita} & \multicolumn{3}{p{\useCaseMulticol} |}{L'\emph{Amministratore} visualizza un messaggio di conferma della rimozione del canale.} \\
		\hline
	\end{tabular}
	\caption{\textbf{CUP9 - Cancella canale}}
\end{table}

\newpage
\paragraph{CUP10 - Visualizza lista utenti \\}
L'\emph{Amministratore} vuole visualizzare la lista degli utenti presenti all'interno di un canale.\\
\begin{table}
	%\normalsize % Dimensione testo normale
	\small % Dimensione testo piccola
	\label{CUP10 - Visualizza lista utenti}
	\begin{tabular}{| p{\useCaseLeft} | p{\useCaseNum} | p{\useCaseTwoCol} | p{\useCaseTwoCol} |}
		\hline
		\textbf{Nome caso d'uso} & \multicolumn{3}{p{\useCaseMulticol} |}{\textbf{CUP10 - Visualizza lista utenti}} \\
		\hline
		\textbf{Attori partecipanti} & \multicolumn{3}{p{\useCaseMulticol} |}{Inizializzato da \textbf{\emph{Amministratore}}.} \\
		\hline
		\textbf{Condizioni d'ingresso} & \multicolumn{3}{p{\useCaseMulticol} |}{L'\emph{Amministratore} seleziona un canale di discussione.} \\
		\hline
		\textbf{Flusso degli eventi} & \textbf{\#} & \textbf{\emph{Amministratore}} & \textbf{Sistema} \\
		\hline
		\textbf{} & \textbf{1} & Seleziona la voce "Visualizza utenti";& \textbf{} \\
		\hline
		\textbf{} & \textbf{2} & \textbf{} & Mostra il numero e la lista degli utenti presenti nel canale distinguendo gli utenti \emph{Amministratori} dai semplici utenti del canale; \\
		\hline
		\textbf{Eccezioni} & \multicolumn{3}{p{\useCaseMulticol} |}{1.1 Connessione assente: viene visualizzato un messaggio di errore;\newline 2.1 Errore di sistema: viene visualizzato un messaggio di errore;} \\
		\hline
		\textbf{Condizioni d'uscita} & \multicolumn{3}{p{\useCaseMulticol} |}{L'\emph{Amministratore} visualizza la lista utenti.} \\
		\hline
	\end{tabular}
	\caption{\textbf{CUP10 - Visualizza lista utenti}}
\end{table} 

\newpage
\paragraph{CUP11 - Aggiungi un utente ad un canale \\}
L'\emph{Amministratore} vuole aggiungere un Utente ad uno specifico canale.\\
\begin{table}
	%\normalsize % Dimensione testo normale
	\small % Dimensione testo piccola
	
	\label{CUP11 - Aggiungi un utente ad un canale}
	\begin{tabular}{| p{\useCaseLeft} | p{\useCaseNum} | p{\useCaseTwoCol} | p{\useCaseTwoCol} |}
		\hline
		\textbf{Nome caso d'uso} & \multicolumn{3}{p{\useCaseMulticol} |}{\textbf{CUP11 - Aggiungi un utente ad un canale}} \\
		\hline
		\textbf{Attori partecipanti} & \multicolumn{3}{p{\useCaseMulticol} |}{Inizializzato da \textbf{\emph{Amministratore}}.} \\
		\hline
		\textbf{Condizioni d'ingresso} & \multicolumn{3}{p{\useCaseMulticol} |}{L'\emph{Amministratore} seleziona un canale di una \emph{chat} e visualizza la lista degli utenti del canale.
		} \\
		\hline
		\textbf{Flusso degli eventi} & \textbf{\#} & \textbf{\emph{Amministratore}} & \textbf{Sistema} \\
		\hline
		\textbf{} & \textbf{1} & Seleziona la voce “Aggiungi membro”; & \textbf{} \\
		\hline
		\textbf{} & \textbf{2} & \textbf{} & Mostra la schermata per aggiungere il membro; \\
		\hline
		\textbf{} & \textbf{3}& Inserisce l’identificativo dell’utente da aggiungere;& \textbf{} \\
		\hline
		\textbf{} & \textbf{4} & \textbf{} & Aggiunge l’utente;\\
		\hline
		
		\textbf{Eccezioni} & \multicolumn{3}{p{\useCaseMulticol} |}{1.1 Connessione assente: viene visualizzato un messaggio di errore;\newline 2.1 - 4.1 Errore di sistema: viene visualizzato un messaggio di errore;\newline 3.1 Identificativo non trovato: viene visualizzato un messaggio di errore;} \\
		\hline
		\textbf{Condizioni d'uscita} & \multicolumn{3}{p{\useCaseMulticol} |}{L'\emph{Amministratore}  visualizza l’utente nella lista utenti.} \\
		\hline
	\end{tabular}
	\caption{\textbf{CUP11 - Aggiungi un utente ad un canale}}
\end{table}

\newpage
\paragraph{CUP12 - Rimuovere un utente da un canale \\}
L'\emph{Amministratore}  vuole rimuovere un Utente ad uno specifico canale.\\
\begin{table}
	%\normalsize % Dimensione testo normale
	\small % Dimensione testo piccola
	\label{CUP12 - Rimuovere un utente da un canale}
	\begin{tabular}{| p{\useCaseLeft} | p{\useCaseNum} | p{\useCaseTwoCol} | p{\useCaseTwoCol} |}
		\hline
		\textbf{Nome caso d'uso} & \multicolumn{3}{p{\useCaseMulticol} |}{\textbf{CUP12 - Rimuovere un utente da un canale}} \\
		\hline
		\textbf{Attori partecipanti} & \multicolumn{3}{p{\useCaseMulticol} |}{Inizializzato da \textbf{\emph{Amministratore}}.} \\
		\hline
		\textbf{Condizioni d'ingresso} & \multicolumn{3}{p{\useCaseMulticol} |}{L'\emph{Amministratore}  seleziona un canale di discussione di una chat e visualizza la lista degli utenti del canale.
		} \\
		\hline
		\textbf{Flusso degli eventi} & \textbf{\#} & \textbf{\emph{Amministratore}} & \textbf{Sistema} \\
		\hline
		\textbf{} & \textbf{1} & Seleziona l’utente nella lista e seleziona la voce “Rimuovi membro”; & \textbf{} \\
		\hline
		\textbf{} & \textbf{2} & \textbf{} & Rimuove l’utente dal canale;  \\
		\hline
		
		\textbf{Eccezioni} & \multicolumn{3}{p{\useCaseMulticol} |}{1.1 Connessione assente: viene visualizzato un messaggio di errore;\newline 2.1 Errore di sistema: viene visualizzato un messaggio di errore;} \\
		\hline
		\textbf{Condizioni d'uscita} & \multicolumn{3}{p{\useCaseMulticol} |}{L'\emph{Amministratore} visualizza un messaggio di conferma rimozione.} \\
		\hline
	\end{tabular}
	\caption{\textbf{CUP12 - Rimuovere un utente da un canale}}
\end{table}

\newpage
\paragraph{CUP13 - Silenziare utente in un canale \\}
L'\emph{Amministratore}vuole silenziare l'utente che ha scritto un messaggio offensivo in una \emph{chat}. Silenziando l'utente, quest'ultimo non avrà la possibilità di inviare messaggi all'interno della \emph{chat}.\\
\begin{table}
	%\normalsize % Dimensione testo normale
	\small % Dimensione testo piccola
	\label{CUP13 - Silenziare utente in un canale}
	\begin{tabular}{| p{\useCaseLeft} | p{\useCaseNum} | p{\useCaseTwoCol} | p{\useCaseTwoCol} |}
		\hline
		\textbf{Nome caso d'uso} & \multicolumn{3}{p{\useCaseMulticol} |}{\textbf{CUP13 - Silenziare utente in un canale}} \\
		\hline
		\textbf{Attori partecipanti} & \multicolumn{3}{p{\useCaseMulticol} |}{Inizializzato da \textbf{\emph{Amministratore}}.} \\
		\hline
		\textbf{Condizioni d'ingresso} & \multicolumn{3}{p{\useCaseMulticol} |}{L'\emph{Amministratore} seleziona il canale di discussione dove vuole silenziare un utente e visualizza il messaggio offensivo o la lista degli utenti del canale. } \\
		\hline
		\textbf{Flusso degli eventi} & \textbf{\#} & \textbf{\emph{Amministratore}} & \textbf{Sistema} \\
		\hline
		\textbf{} & \textbf{1} & Seleziona il messaggio offensivo; & \textbf{}  \\
		\hline
		\textbf{} & \textbf{2} &  \textbf{} & Mostra il pulsante per silenziare l’utente; \\
		\hline
		\textbf{} & \textbf{3} &Seleziona il comando “silenzia”; & \textbf{}  \\
		\hline
		\textbf{} & \textbf{4} &  \textbf{} & Silenzia l’utente; \\
		\hline
		\textbf{} & \textbf{1.1} & Seleziona un utente del canale; & \textbf{}  \\
		\hline
		\textbf{} & \textbf{2.1} &  \textbf{} & Mostra il pulsante per silenziare l’utente;  \\
		\hline
		\textbf{} & \textbf{3.1} &Seleziona il comando “silenzia”; & \textbf{}  \\
		\hline
		\textbf{} & \textbf{4.1} &  \textbf{} & Silenzia l’utente; \\
		\hline
		
		\textbf{Eccezioni} & \multicolumn{3}{p{\useCaseMulticol} |}{1.1 - 1.1.1 Connessione assente: viene visualizzato un messaggio di errore;
			\newline 2.1 - 2.1.1 - 4.1 - 4.1.1 Errore di sistema: viene visualizzato un messaggio di errore;} \\
		\hline
		\textbf{Condizioni d'uscita} & \multicolumn{3}{p{\useCaseMulticol} |}{L'\emph{Amministratore} visualizza un messaggio di conferma del successo dell’operazione.
		} \\
		\hline
	\end{tabular}
	\caption{\textbf{CUP13 - Silenziare utente in un canale}}
\end{table}


\newpage
\paragraph{CUP14 - Reintegra utente in un canale \\}
L'\emph{Amministratore} vuole reintegrare l'utente che è stato precedentemente silenziato in un canale. Reintegrando l'utente, quest'ultimo ha di nuovo la possibilità di inviare messaggi all'interno del canale della \emph{chat}.\\
\begin{table}
	%\normalsize % Dimensione testo normale
	\small % Dimensione testo piccola
	
	\label{CUP14 - Reintegra utente in un canale}
	\begin{tabular}{| p{\useCaseLeft} | p{\useCaseNum} | p{\useCaseTwoCol} | p{\useCaseTwoCol} |}
		\hline
		\textbf{Nome caso d'uso} & \multicolumn{3}{p{\useCaseMulticol} |}{\textbf{CUP14 - Reintegra utente in un canale}} \\
		\hline
		\textbf{Attori partecipanti} & \multicolumn{3}{p{\useCaseMulticol} |}{Inizializzato da \textbf{\emph{Amministratore}}.} \\
		\hline
		\textbf{Condizioni d'ingresso} & \multicolumn{3}{p{\useCaseMulticol} |}{L'\emph{Amministratore}  seleziona il canale di discussione dove vuole reintegrare un utente e visualizza la lista utenti del canale.} \\
		\hline
		\textbf{Flusso degli eventi} & \textbf{\#} & \textbf{\emph{Amministratore}} & \textbf{Sistema} \\
		\hline
		\textbf{} & \textbf{1} & Seleziona un utente silenziato; & \textbf{} \\
		\hline
		\textbf{} & \textbf{2} & \textbf{} & Mostra la voce “Reintegra utente”;  \\
		\hline
		\textbf{} & \textbf{3}& Seleziona la voce ”Reintegra”; & \textbf{} \\
		\hline
		\textbf{} & \textbf{4} & \textbf{} & Reintegra l’utente; \\
		\hline
		
		\textbf{Eccezioni} & \multicolumn{3}{p{\useCaseMulticol} |}{1.1 Connessione assente: viene visualizzato un messaggio di errore;
			\newline 1.2 Il canale non contiene utenti silenziati: viene visualizzato un messaggio di errore;	
			\newline 2.1 - 4.1 Errore di sistema: viene visualizzato un messaggio di errore;} \\
		\hline
		\textbf{Condizioni d'uscita} & \multicolumn{3}{p{\useCaseMulticol} |}{L'\emph{Amministratore}  visualizza un messaggio di conferma del successo dell’operazione.
		} \\
		\hline
	\end{tabular}
	\caption{\textbf{CUP14 - Reintegra utente in un canale}}
\end{table}


\newpage
\paragraph{CUP15 - Modificare i permessi di un utente in un canale \\}
L'\emph{Amministratore}   vuole modificare il ruolo che uno specifico utente ricopre all’interno di un canale di una \emph{chat}, cioè impostare un utente amministratore oppure rimuovere questo titolo e far diventare quindi un amministratore di un canale un semplice utente.\\
\begin{table}
	%\normalsize % Dimensione testo normale
	\small % Dimensione testo piccola
	
	\label{CUP15 - Modificare i permessi di un utente in un canale}
	\begin{tabular}{| p{\useCaseLeft} | p{\useCaseNum} | p{\useCaseTwoCol} | p{\useCaseTwoCol} |}
		\hline
		\textbf{Nome caso d'uso} & \multicolumn{3}{p{\useCaseMulticol} |}{\textbf{CUP15 - Modificare i permessi di un utente in un canale}} \\
		\hline
		\textbf{Attori partecipanti} & \multicolumn{3}{p{\useCaseMulticol} |}{Inizializzato da \textbf{\emph{Amministratore}}.} \\
		\hline
		\textbf{Condizioni d'ingresso} & \multicolumn{3}{p{\useCaseMulticol} |}{L'\emph{Amministratore}  seleziona il canale di discussione dove è presente l’utente da modificare e visualizza la lista degli utenti del canale.
		} \\
		\hline
		\textbf{Flusso degli eventi} & \textbf{\#} & \textbf{\emph{Amministratore}} & \textbf{Sistema} \\
		\hline
		\textbf{} & \textbf{1} &Seleziona un utente e modifica il suo ruolo; & \textbf{} \\
		\hline
		\textbf{} & \textbf{2} & \textbf{} & Modifica ruolo dell’utente e visualizza la modifica; \\
		\hline
		
		\textbf{Eccezioni} & \multicolumn{3}{p{\useCaseMulticol} |}{1.1 Connessione assente: viene visualizzato un messaggio di errore;\newline 2.1 Errore di sistema: viene visualizzato un messaggio di errore;
			\newline 2.2 L’utente ricopre già il ruolo assegnato: viene visualizzato un messaggio di errore;} \\
		\hline
		\textbf{Condizioni d'uscita} & \multicolumn{3}{p{\useCaseMulticol} |}{L'\emph{Amministratore}  visualizza un messaggio di conferma del successo dell’operazione.} \\
		\hline
	\end{tabular}
	\caption{\textbf{CUP15 - Modificare i permessi di un utente in un canale}}
\end{table}


\newpage
\paragraph{CUP16 - Nascondi messaggio \\}
L'\emph{Amministratore} vuole nascondere il contenuto di un messaggio in uno specifico canale di una \emph{chat}. Può risultare opportuno nascondere un messaggio in seguito ad una segnalazione una volta verificato che il messaggio segnalato risulti effettivamente inopportuno. 
Inoltre l'\emph{Amministratore} ha piena libertà di poter nascondere qualsiasi messaggio del canale.\\
\begin{table}
	%\normalsize % Dimensione testo normale
	\small % Dimensione testo piccola
	\label{CUP16 - Nascondi messaggio}
	\begin{tabular}{| p{\useCaseLeft} | p{\useCaseNum} | p{\useCaseTwoCol} | p{\useCaseTwoCol} |}
		\hline
		\textbf{Nome caso d'uso} & \multicolumn{3}{p{\useCaseMulticol} |}{\textbf{CUP16 - Nascondi messaggio}} \\
		\hline
		\textbf{Attori partecipanti} & \multicolumn{3}{p{\useCaseMulticol} |}{Inizializzato da \textbf{\emph{Amministratore}}.} \\
		\hline
		\textbf{Condizioni d'ingresso} & \multicolumn{3}{p{\useCaseMulticol} |}{L'\emph{Amministratore}  seleziona un canale di discussione e visualizza i messaggi.
		} \\
		\hline
		\textbf{Flusso degli eventi} & \textbf{\#} & \textbf{\emph{Amministratore}} & \textbf{Sistema} \\
		\hline
		\textbf{} & \textbf{1} & Seleziona uno specifico messaggio;  & \textbf{} \\
		\hline
		\textbf{} & \textbf{2} & \textbf{} & Mostra l’icona per nascondere lo specifico messaggio nel canale; \\
		\hline
		\textbf{} & \textbf{3}& Seleziona l’icona per nascondere il messaggio; & \textbf{} \\
		\hline
		\textbf{} & \textbf{4} & \textbf{} & Nasconde il contenuto del messaggio all’interno del canale;\\
		\hline
		
		\textbf{Eccezioni} & \multicolumn{3}{p{\useCaseMulticol} |}{1.1 Connessione assente: viene visualizzato un messaggio di errore;\newline 2.1 Errore di sistema: viene visualizzato un messaggio di errore;} \\
		\hline
		\textbf{Condizioni d'uscita} & \multicolumn{3}{p{\useCaseMulticol} |}{L'\emph{Amministratore}  visualizza la scritta “Questo messaggio è stato nascosto” al posto del testo del messaggio originale.} \\
		\hline
	\end{tabular}
	\caption{\textbf{CUP16 - Nascondi messaggio}}
\end{table}


\newpage
\paragraph{CUP17 - Reintegra messaggio \\}
L'\emph{Amministratore}  vuole reintegrare un messaggio in uno specifico canale di una \emph{chat}. 
Può risultare opportuno reintegrare un messaggio in seguito ad una segnalazione una volta verificato che il messaggio segnalato non risulta effettivamente inopportuno. 
Un messaggio per essere reintegrato deve per forza essere stato segnalato quindi risultare evidenziato all’interno del canale.\\
\begin{table}
	%\normalsize % Dimensione testo normale
	\small % Dimensione testo piccola
	\label{CUP17 - Reintegra messaggio}
	\begin{tabular}{| p{\useCaseLeft} | p{\useCaseNum} | p{\useCaseTwoCol} | p{\useCaseTwoCol} |}
		\hline
		\textbf{Nome caso d'uso} & \multicolumn{3}{p{\useCaseMulticol} |}{\textbf{CUP17 - Reintegra messaggio}} \\
		\hline
		\textbf{Attori partecipanti} & \multicolumn{3}{p{\useCaseMulticol} |}{Inizializzato da \textbf{\emph{Amministratore}}.} \\
		\hline
		\textbf{Condizioni d'ingresso} & \multicolumn{3}{p{\useCaseMulticol} |}{L'\emph{Amministratore} seleziona un canale di discussione e visualizza  un messaggio segnalato.
		} \\
		\hline
		\textbf{Flusso degli eventi} & \textbf{\#} & \textbf{\emph{Amministratore}} & \textbf{Sistema} \\
		\hline
		\textbf{} & \textbf{1} & Seleziona uno specifico messaggio evidenziato come segnalato;  & \textbf{} \\
		\hline
		\textbf{} & \textbf{2} & \textbf{} & Mostra l’icona per togliere la segnalazione allo specifico messaggio nel canale;  \\
		\hline
		\textbf{} & \textbf{3}& Seleziona l’icona per nascondere il messaggio; & \textbf{} \\
		\hline
		\textbf{} & \textbf{4} & \textbf{} & Toglie la segnalazione al messaggio selezionato;\\
		\hline
		
		\textbf{Eccezioni} & \multicolumn{3}{p{\useCaseMulticol} |}{1.1 Connessione assente: viene visualizzato un messaggio di errore;\newline 2.1 Errore di sistema: viene visualizzato un messaggio di errore;} \\
		\hline
		\textbf{Condizioni d'uscita} & \multicolumn{3}{p{\useCaseMulticol} |}{L'\emph{Amministratore}  visualizza il messaggio non più segnalato.} \\
		\hline
	\end{tabular}
	\caption{\textbf{CUP17 - Reintegra messaggio}}
\end{table}

\newpage
\paragraph{CUP18 - Invio Notifiche \\}
L'\emph{Amministratore} vuole inviare una particolare notifica scegliendo a quale categoria di \emph{chat} indirizzarla così da avere la sicurezza che solo chi realmente interessato al contenuto della notifica la riceva. \\
\begin{table}
	%\normalsize % Dimensione testo normale
	\small % Dimensione testo piccola
	\label{CUP18 - Invio Notifiche}
	\begin{tabular}{| p{\useCaseLeft} | p{\useCaseNum} | p{\useCaseTwoCol} | p{\useCaseTwoCol} |}
		\hline
		\textbf{Nome caso d'uso} & \multicolumn{3}{p{\useCaseMulticol} |}{\textbf{CUP18 - Invio Notifiche}} \\
		\hline
		\textbf{Attori partecipanti} & \multicolumn{3}{p{\useCaseMulticol} |}{Inizializzato da \textbf{\emph{Amministratore}}.} \\
		\hline
		\textbf{Condizioni d'ingresso} & \multicolumn{3}{p{\useCaseMulticol} |}{L'\emph{Amministratore} effettua il \emph{login}
		} \\
		\hline
		\textbf{Flusso degli eventi} & \textbf{\#} & \textbf{\emph{Amministratore}} & \textbf{Sistema} \\
		\hline
		\textbf{} & \textbf{1} & Seleziona la voce “Invia Notifiche”;   & \textbf{} \\
		\hline
		\textbf{} & \textbf{2} & \textbf{} & Mostra un’area di testo in cui scrivere il messaggio della notifica e la possibilità di effettuare una ricerca;   \\
		\hline
		\textbf{} & \textbf{3}& Scrive il messaggio della notifica;
		Effettua una ricerca; Invia la notifica; & \textbf{} \\
		\hline
		\textbf{} & \textbf{4} & \textbf{} & Riceve la notifica e la invia alle \emph{chat} selezionate; \\
		\hline
		
		\textbf{Eccezioni} & \multicolumn{3}{p{\useCaseMulticol} |}{1.1 Connessione assente: viene visualizzato un messaggio di errore;\newline 2.1 Errore di sistema: viene visualizzato un messaggio di errore;
			\newline 4.1 Il campo di testo è vuoto: viene visualizzato un messaggio di errore;} \\
		\hline
		\textbf{Condizioni d'uscita} & \multicolumn{3}{p{\useCaseMulticol} |}{L'\emph{Amministratore}  visualizza un messaggio di conferma dell’invio della notifica.} \\
		\hline
	\end{tabular}
	\caption{\textbf{CUP18 - Invio Notifiche}}
\end{table}

\newpage
\paragraph{CUP19 - Gestione messaggi inopportuni \\}
L'\emph{Amministratore}  vuole visualizzare i messaggi segnalati come inopportuni all’interno delle \emph{chat}, così da poter decidere se nascondere o meno i messaggi che possono essere ritenuti offensivi o non adatti alla conversazione, decidendo inoltre anche se silenziare l’utente che ha scritto un messaggio offensivo.\\
\begin{table}
	%\normalsize % Dimensione testo normale
	\small
	
	\label{CUP19 - Gestione messaggi inopportuni}
	\begin{tabular}{| p{\useCaseLeft} | p{\useCaseNum} | p{\useCaseTwoCol} | p{\useCaseTwoCol} |}
		\hline
		\textbf{Nome caso d'uso} & \multicolumn{3}{p{\useCaseMulticol} |}{\textbf{CUP19 - Gestione messaggi inopportuni}} \\
		\hline
		\textbf{Attori partecipanti} & \multicolumn{3}{p{\useCaseMulticol} |}{Inizializzato da \textbf{\emph{Amministratore}}.} \\
		\hline
		\textbf{Condizioni d'ingresso} & \multicolumn{3}{p{\useCaseMulticol} |}{L'\emph{Amministratore} effettua il \emph{login}.
		} \\
		\hline
		\textbf{Flusso degli eventi} & \textbf{\#} & \textbf{\emph{Amministratore}} & \textbf{Sistema} \\
		\hline
		\textbf{} & \textbf{1} & Seleziona la voce “Gestione messaggi inopportuni”;    & \textbf{} \\
		\hline
		\textbf{} & \textbf{2} & \textbf{} & Mostra una lista di tutte le \emph{chat} dove è presente almeno un messaggio segnalato ancora da verificare riportando il numero di messaggi segnalati all’interno di ognuna; 
		Mostra la possibilità di effettuare una ricerca della lista;    \\
		\hline
		\textbf{} & \textbf{3}& Seleziona una \emph{chat} dove è presente almeno un messaggio da verificare;  & \textbf{} \\
		\hline
		\textbf{} & \textbf{4} & \textbf{} & Mostra la \emph{chat} evidenziando il o i messaggio / i da verificare; \\
		\hline
		
		
		\textbf{Eccezioni} & \multicolumn{3}{p{\useCaseMulticol} |}{1.1 Connessione assente: viene visualizzato un messaggio di errore;\newline 2.1 Errore di sistema: viene visualizzato un messaggio di errore;} \\
		\hline
		\textbf{Condizioni d'uscita} & \multicolumn{3}{p{\useCaseMulticol} |}{L'\emph{Amministratore}   visualizza il messaggio segnalato evidenziato nel canale.} \\
		\hline
	\end{tabular}
	\caption{\textbf{CUP19 - Gestione messaggi inopportuni}}
\end{table}

\clearpage